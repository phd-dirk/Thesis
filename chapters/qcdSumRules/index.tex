\documentclass[../../index.tex]{subfiles}

\begin{document}
\chapter{QCD Sum Rules}
\label{ch:theoreticalBackground}
The theory of \textsc{qcd} was formulated to find one single framework that
describes the many hadrons that exist. Unfortunately making use of
\define{pqcd}{perturbative \textsc{qcd}} is limited. \textsc{qcd} predicts a
large coupling constant for low energies. As a consequence we can only ever
observe hadrons, but our theoretical foundation is ruled by the \textsc{dof} of
quarks and gluons. To extract \textsc{qcd} parameters (the six quark masses and
the strong coupling) from hadrons we need to connect the quark\-/gluon picture
with the hadron picture. To do so we will introduce the framework of
\textsc{qcdsr}.

We will start by setting up the foundations of strong interaction with
introducing the \textsc{qcd} Lagrangian. The \textsc{qcd} Lagrangian is ruled by
the abelian gauge group \(SU(3)\). The group implies an energy dependence of the
coupling and thus limits the applicability of \textsc{pt} for low energies,
where the coupling is large. Next we will focus on the two\-/point function,
which plays a major role in the framework of \textsc{qcdsr}. The two\-/point
function is defined as vacuum\-/expectation values of the time ordered product
of two local fields
\begin{equation}
  \Pi(q^2) = \int\fdif{q} e^{iqx} \langle\Omega\vert T\{ \anti{q}(x)q(0) \} \vert\Omega\rangle.
\end{equation}
We can use it to theoretically describe processes, like \(\tau\) decays into
hadrons, by matching the quantum numbers of the fields, we choose in specifying
the two\-/point function, to the outgoing hadrons. We will see, that the
two\-/point function \(\Pi(q^2)\) is related to hadronic states, by poles for
\(q^2> 0\). Here \textsc{np} effects become important and we need to introduce
the \textsc{ope}, which handles \textsc{np} parts through the \textsc{qcd}
condensates. The condensates form part of the full physical vacuum and would not
exist regarding the perturbative vacuum solely. Consequently the condensates are
not accessible trough \textsc{pt} methods and have to be fitted from experiment
or calculated with the help of \textsc{np} tools, like \textsc{lqcd}. Finally we
will combine a dispersion relation and Cauchy's theorem to finalise the
discussion on the \textsc{qcdsr} with developing the \define{fesr}{finite energy
  sum rules}, which we will apply to extract the strong coupling from
\(\tau\) decays into hadrons.



\section{Quantum\-/Chromodynamics}
\label{sec:quantumchromodynamics}
Since the formulation of \textsc{qed} in the end of the forties it had been
attempted to construct a \textsc{qft} of the strong nuclear force, which has
been achieved in the 70's as \textsc{qcd}
\cite{GellMann1972,Fritzsch1973,Gross1973,Politzer1973,Weinberg1973}.
The fundamental fields of \textsc{qcd} are given by dirac spinors of
spin\-/\(1/2\), the so\-/called quarks, with a fractional electric charge of
\(\pm 1/3\) or \(\pm 2/3\). The theory furthermore contains gauge fields of spin
1. These gauge fields are called gluons, do not carry electric charge and are
massless. They are the force mediators, which interact with quarks and
themselves, because they carry colour charge, in contrast to photons of
\textsc{qed}, which interact only with
fermions. %(see \cref{fig:qcdFeynmanDiagrams}).

The corresponding gauge group of \textsc{qcd} is the non\-/abelian group
\(SU(3)\). Each of the quark flavours \(u,d,c,s,t\) and \(b\) belongs to the
fundamental representation of \(SU(3)\) and contains a triplet of fields
\(\Psi\).
\begin{equation}
  \Psi = \begin{pmatrix} \Psi_1 \\ \Psi_2 \\ \Psi_3 \end{pmatrix}
\end{equation}
The labels of the triplet are the colours red, green and blue, which play the
role of \textit{colour charge}, similar to the electric charge of \textsc{qed}.
The gluons belong to the adjoint representation of \(SU(3)\), contain an octet
of fields and can be expressed using the Gell-Mann matrices \(\lambda_a\)
\begin{equation}
  B_\mu = B_\mu^a \lambda_a \qquad a = 1,2,\dotsc 8
\end{equation}
% \begin{figure}
%   \centering
%   \includegraphics[width=\textwidth]{./images/qcdFeynmanDiagrams.eps}
%   \caption{Feynman diagrams of the strong interactions with corresponding
%   electromagnetic diagrams. We see that the gluons carry colour charge and
%   thus couple to other gluons, which is not the case for the photons.}
%   \label{fig:qcdFeynmanDiagrams}
% \end{figure}
\begin{table}
  \centering
  \begin{minipage}[c]{0.4\textwidth}
    \begin{tabular}{ll}
      \toprule
      Flavour & Mass\\
      \midrule
      \(u\) & \SI{2.50(17)}{\mega\eV} \\
      \(d\) & \SI{4.88(20)}{\mega\eV} \\
      \(s\) & \SI{93.44(68)}{\mega\eV} \\
      \(c\) & \SI{1.280(13)}{\giga\eV} \\
      \(b\) & \SI{4.198(12)}{\giga\eV} \\
      \(t\) & \SI{173.0(40)}{\giga\eV} \\
      \bottomrule 
    \end{tabular}
  \end{minipage}\hfill
  \begin{minipage}[c]{0.59\textwidth}
    \captionsetup{format=plain}
    \caption{List of quarks and their masses. The masses of the up, down and
      strange quark are quoted in the four\-/flavour theory (\(N_f=2+1+1\)) at
      the scale \(\mu=\SI{2}{\giga\eV}\) in the \(\overline{MS}\) scheme. The charm and
      bottom quark are also taken in the four\-/flavour theory and in the
      \(\overline{MS}\) scheme, but at the scales \(\mu=m_c\) and \(\mu=m_b\)
      correspondingly. All quarks except for the top quark are taken from the
      \defien{FLAG}{Flavour Lattice Averaging Group} \cite{FLAG2019}. The mass of the
      top quark is not discussed in \cite{FLAG2019} and has been taken
      from direct observations of top events \cite{PDG2018}.}
  \end{minipage}
  \label{table:quarkList}
\end{table}
The classical \textit{Lagrange density} of \textsc{qcd} is given by
\cite{Yndurain2006,Pascual1984}:

\begin{tcolorbox}[ams equation,myformula]
  \label{eq:qcdLagrangian}
  \mathcal{L}_{QCD}(x) = -\frac{1}{4}G_{\mu\nu}^a(x)G^{\mu\nu a}(x) + \sum_A
  \left[ \frac{i}{2} \bar{q}^A(x) \gamma^\mu \overleftrightarrow{D}_\mu q^A(x) -
    m_A\bar{q}^A(x) a^A(x) \right],
\end{tcolorbox}
with \(q^A(x)\) representing the quark fields and \(G_{\mu\nu}^a\) being the
\textit{gluon field strength tensor} given by:
\begin{equation}
  \label{eq:gluonField}
  G_{\mu\nu}^a(x) \equiv \partial_\mu B_\nu^a(x) - \partial_\nu B_\mu^a(x) + g f^{abc} B_\mu^b(x) B_\nu^c(x),
\end{equation}
with \(f^{abc}\) as \textit{structure constants} of the gauge group \(SU(3)\)
and \(\overleftrightarrow{D}_\mu\) as covariant derivative acting to the left
and to the right. Furthermore we have used \(A, B, \dotsc = 0, \dotsc 5\) as
flavour indices, \(a, b, \dotsc = 0, \dotsc, 8 \) as colour indices and \(\mu,
\nu, \dotsc = 0, \dotsc 3\) as Lorentz indices. Explicitly the Lagrangian
writes:
\begin{equation}
  \begin{split}
    \mathcal{L}_0(x) &=- \frac{1}{4} \left[ \partial_\mu G_\nu^a(x) - \partial_v G_\mu^a(x) \right] \left[ \partial^\mu G_a^\nu(x) - \partial^\nu G^\mu_a(x) \right] \\
    &\quad+ \frac{i}{2}\anti{q}_\alpha^A(x) \gamma^\mu \partial_\mu q_\alpha^A(x) - \frac{i}{2} \left[ \partial_\mu \anti{q}_\alpha^A(x) \right] \gamma^\mu q_\alpha^A(x) - m_A \anti{q}_\alpha^A(x) q_\alpha^A(x) \\
    &\quad+ \frac{g_s}{2} \anti{q}_\alpha^A(x) \lambda^a_{\alpha \beta}\gamma_\mu q_\beta^A(x) G^\mu_a(x) \\
    &\quad- \frac{g_s}{2}f_{abc}\left[ \partial_\mu G_\nu^a(x) - \partial_\nu G_\mu^a(x)\right] G_b^\mu(x) G_c^\nu(x) \\
    &\quad- \frac{g_s^2}{4} f_{abc}f_{ade} G_\mu^b(x) G_\nu^c(x) G_d^\mu(x)
    G_e^\nu(x)
  \end{split}
\end{equation}
The first term is the kinetic term for the massless gluons. The next three terms
are the kinetic terms for the quark field with different masses for each
flavour. The rest of the terms are the interaction terms. The fifth term
represents the interaction between quarks and gluons and the last two terms the
self interactions of gluon fields.

The corresponding Feynman rules have been displayed in
\cref{fig:qcdFeynmanRules}.
\begin{figure}
  \includegraphics[width=\textwidth]{./images/qcdFeynmanRules.eps}
  \caption{QCD Feynman rules.}
  \label{fig:qcdFeynmanRules}
\end{figure}
The rules are based on \textsc{pt}, but can be enhanced with the \textsc{qcd}
condensates, as we will see in the discussion of the \textsc{ope} in
\cref{sec:ope}

Having derived the Lagrangian leaves us with its quantisation. The
dirac\-/spinors can be quantised as in \textsc{qed} without any problems. The
\(\Psi(x)\) quantum field can be written as:
\begin{equation}
  \Psi(x) = \int \frac{\dif^3 p}{(2\pi)^3 2E(\vec p)} \sum_\lambda \left[ u(\vec p, \lambda) a(\vec p, \lambda) e^{-ipx} + v(\vec p, \lambda) b^\dagger (\vec p, \lambda) e^{ipx} \right],
\end{equation}
where the integration ranges over the positive sheet of the mass hyperboloid
\(\Omega_+(m) = \{p \vert p^2 = m^2, p^0 > 0 \}\). The four spinors \(u(\vec p,
\lambda)\) and \(v(\vec p, \lambda)\) are solutions to the dirac equations in
momentum space
\begin{equation}
  \begin{split}
    [\slashed p - m]u(\vec p, \lambda) &= 0 \\
    [\slashed p + m]v(\vec p, \lambda) &= 0,
  \end{split}
\end{equation}
with \(\lambda\) representing the helicity state of the spinors.

The quantisation of the gauge fields is more cumbersome. One is forced to
introduce supplementary non\-/physical fields, the so\-/called Faddeev\-/Popov
ghosts \(c^a(x)\) \cite{Faddeev1967}, to cancel unphysical helicity degrees of
freedom of the gluon fields.

The free propagators for the quark, the gluon and the ghost fields are then
given by
\begin{equation}
  \begin{split}
    i S^{(0)AB}_{\alpha\beta}(x-y) &\equiv \wick{ \c q_\alpha^A(x) \anti{\c
        q}_\beta^B(y)} \equiv \langle 0 \vert T \{ q_\alpha^A(x)
    \anti{q}_\beta^B \} \vert 0 \rangle
    = \delta_{AB} \delta_{\alpha\beta} i S^{(0)}(x-y) \\
    &= i \delta_{AB}\delta_{\alpha\beta}\int \fdif{p} \frac{\slashed p + m}{(p^2 - m^2 + i\epsilon)}\\
    i D^{(0)\mu\nu}_{ab}(x-y) &\equiv \wick{ \c B_a^\mu(x) \c B_b^\nu(y)} \equiv
    \langle 0 \vert T \{ B_a^\mu(x) B_b^\nu(y) \} \vert 0 \rangle
    \equiv \delta_{ab}i \int \fdif{k} D^{(0)\mu\nu}(k) e^{-ik(x-y)} \\
    &= i \delta_{ab} \int \fdif{k} \frac{1}{k^2+i\epsilon} \left[ -g_{\mu\nu} + (1-a) \frac{k_\mu k_\nu}{k^2+i\epsilon} \right] e^{-ik(x-y)} \\
    i \widetilde{D}_{ab}^{(0)}(x-y) & \equiv \wick{\c \phi_a(x) \anti{\c
        \phi}_b(y)} \equiv \langle 0 \vert T\{ \phi_a(x) \anti{\phi}_b(y) \}
    \vert 0 \rangle
    = i \delta_{ab} \int \fdif{q} \frac{-1}{q^2 + i\epsilon} e^{-q(x-y)} \\
    &\equiv i \delta_{ab} \int \fdif{q} \widetilde{D}^{(0)}(q)
    e^{-iq(x-y)},
  \end{split}
\end{equation}

The previously introduced Feynman rules and propagators all make use of the
perturbative vacuum \(\vert 0 \rangle\) and thus count as tools of
\textsc{pt}. Consequently they need a small coupling to approximate excitations
of full \textsc{qcd} vacuum. We will see in the following section, that the
strong coupling runs with energy and unfortunately is large for small energy
scales.

\subsection{Renormalisation Group}
Computing observables with the \textsc{qcd} Lagrangian (\cref{eq:qcdLagrangian})
lead to divergencies, which have to be \textit{renormalised}. To render these
divergent quantities finite we have to introduce a suitable parameter such
that the ``original divergent theory'' corresponds to a certain value of that
parameter. These procedure is referred to as \textit{regularisation} and there
are various approaches:
\begin{itemize}
\item \label{itm:lambdaRegularisation}\textbf{Cut\-/off regularisation:} In
  cut\-/off regularisation we limit the divergent momentum integrals by a
  cut\-/off \(\abs{\vec p} < \Lambda\). Here \(\Lambda\) has the dimension of
  mass. The cut\-/off regularisation breaks translational invariance, which can
  be guarded by making use of other regularisation methods.
\item \textbf{\define{P\-/V}{Pauli\-/Villars} regularisation:} \cite{Pauli1949}
  In \textsc{P-V} regularisation the propagator is forced to decrease faster
  than the divergence to appear. It replaces the nominator by
  \begin{equation}
    (\vec p^2 + m^2)^{-1} \to (\vec p^2 + m^2)^{-1} - (\vec p^2 + M^2)^{-1},
  \end{equation}
  where \(M\) has the dimension acts similar as the previously presented
  cut\-/off, but conserves translational invariance.
\item \textbf{Dimensional regularisation:}
  \cite{Bollini1972,tHooft1972,tHooft1973} Dimensional regularisation has been
  introduced in the beginning of the seventies to regularise non\-/abelian gauge
  theories (like \textsc{qcd}), where \(\Lambda\)- and
  \textsc{P-V}-regularisation failed. In dimensional regularisation we expand
  the four space\-/time dimensions to arbitrary \(D\) dimensions. To compensate
  for the additional dimensions we introduce an additional scale \(\mu^{D-4}\).
  A typical Feynman integral then has the following appearance:
  \begin{equation}
    \label{eq:dimRegFeynmanIntegral}
    \int \fdif{p} \frac{1}{\vec p^2 + m^2} \to \mu^{2\epsilon} \int \frac{\dif^D p}{(2\pi)^D}\frac{1}{\vec p^2+m^2}.
  \end{equation}
  Dimensional regularisation preserves all symmetries and allows an easy
  identification of divergences and naturally leads to the \define{ms}{minimal
    subtraction scheme} \cite{tHooft1973,Weinberg1973a}.
\end{itemize}
In all of the three regularisation schemes we introduced an arbitrary parameter
to regularise the divergence. This parameter causes scale dependence of the
strong coupling and the quark masses. As we are mainly concerned with the
non\-/abelian gauge theory \textsc{qcd} we will focus on dimensional
regularisation, which introduced the parameter \(\mu\). Measurable observables
(\textit{physical quantities}) cannot depend on the renormalisation scale
\(\mu\). Therefore the derivative by \(\mu\) of a general physical quantity has
to yield zero. A physical quantity \(R(q, a_s, m)\), that depends on the
external momentum q, the renormalised coupling \(a_s\equiv\alpha_s/\pi\) and the
renormalised quark mass \(m\) can then be expressed as
\begin{equation}
  \label{eq:RGE}
  \mu \od{}{\mu}R(q, a_s, m) = \left[ \mu \pd{}{\mu} + \mu \od{a_s}{\mu} \pd{}{m} + \mu \od{m}{\mu} \pd{}{m} \right] R(q, a_s, m) = 0.
\end{equation}
\Cref{eq:RGE} is referred to as a \define{rge}{renormalisation group equation}
and is the basis for defining the two \textit{renormalisation group functions}:
\begin{align}
  \beta(a_s) &\equiv -\mu \od{a_s}{\mu} = \beta_1 a_s^2 + \beta_2 a_s^3 + \dots & \beta\text{\-/function}
                                                                                  \label{eq:betaFunction} \\
  \gamma(a_s) &\equiv - \frac{\mu}{m} \od{m}{\mu} = \gamma_1 a_s + \gamma_2 a_s^2 + \dots & \text{anomalous mass dimension}.
                                                                                            \label{eq:anomalousMassDimension}
\end{align}
The \(\beta\)\-/function dictates the running of the strong coupling, whereas
the anomalous mass dimension describes the running of the quark masses. We have
a special interest in the running of the strong coupling, but will also shortly
sum up the running of the quark masses.

\subsubsection{Running gauge coupling}
Regarding the \(\beta\)-function we notice, that \(a_s(\mu)\) is not a constant,
but that it \textit{runs} by varying its scale \(\mu\). To better understand the
running of the strong coupling we integrate the \(\beta\)\-/function
\begin{equation}
  \int_{a_s(\mu_1)}^{a_s(\mu_2)}\frac{\dif a_s}{\beta(a_s)} = - \int_{\mu_1}^{\mu_2} \frac{\dif \mu}{\mu} = \log \frac{\mu_1}{\mu_2}.
\end{equation}
We analytically evaluate the above integral by approximating the
\(\beta\)-function to first order, with the known coefficient
\begin{equation}
  \label{eq:firstBetaCoefficient}
  \beta_1 = \frac{1}{6}(11 N_c - 2 N_f),
\end{equation}
which yields
\begin{equation}
  \label{eq:strongCouplingFirstOrder}
  a_s(\mu_2) = \frac{a_s(\mu_1)}{\left( 1 - a_s(\mu_1) \beta_1 \log\frac{\mu_1}{\mu_2} \right)}.
\end{equation}

\begin{figure}
  \centering \includegraphics[width=\textwidth]{./images/runningOfAs.eps}
  \caption{Running of the strong coupling \(\alpha_s(Q^2)\) at first order. The
    blue line represents the uncorrected coupling constant, with an
    \(\Lambda^{nf=5}\) chosen to match an experimental value of the coupling at
    \(Q^2=M_Z^2\). The quark\-/thresholds are shown by the black line and the
    corrected running is given by the red line. We additionally marked the
    breakdown of \textsc{pt} with a grey background for \(Q^2<1\). The image is
    taken from an recent review of the strong coupling \cite{Deur2016}.}
  \label{fig:runningOfAs}
\end{figure}

\Cref{eq:strongCouplingFirstOrder} has some important implications for the
strong coupling:
\begin{itemize}
\item The coupling at a scale \(\mu_2\) depends on \(a_s(\mu_1)\). Thus we have
  to take care of the scale \(\mu\), while comparing different values of
  \(\alpha_s\). In the literature (e.g. \cite{PDG2018}) \(\alpha_s\) is commonly
  compared at the \(Z\) boson scale of around \SI{91}{\giga\eV}. As we are
  extracting the strong coupling at the mass of the \(\tau\) lepton, around
  \SI{1.776}{\giga\eV} we need to run the strong coupling up to the desired
  scale. While running the coupling, we have to take care of the quark
  thresholds. Each quark gets active at a certain energy scale, which leads to a
  running of \(\alpha_s\) as shown in \cref{fig:runningOfAs}. Typically one runs
  the coupling with the aid of software packages like \textit{RunDec}
  \cite{Chetyrkin2000,Herren2017}, which has also been ported to support \(C\)
  (\textit{CRunDec}, \cite{Schmidt2012}) and Python \cite{Straub2016}.
\item As we have three colours (\(N_c=3\)) and six flavours (\(N_f=6\)) the
  \(\beta_1\) coefficient \ref{eq:betaFunction} is positive. Thus for the two
  scales \(\mu_2<\mu_1\) the strong coupling \(a_s(\mu_2)\) increases
  logarithmically and at a scale of \(\mu_2=\SI{1}{\giga\eV}\) reaches a value
  of
  \begin{equation}
    \alpha_s(\SI{1}{\giga\eV}) \approx 0.5,
  \end{equation}
  which questions the applicability of \textsc{pt} for energies lower than
  \SI{1}{\giga\eV} (as seen from the grey zone in \cref{fig:runningOfAs}).
\item A large coupling for small scales implies confinement. We are not able to
  separate quarks in a meson or baryon. No quark has been detected as single
  particle yet. This is qualitatively explained with the gluon field carrying
  colour charge. These gluons form so-called \textit{flux-tubes} between quarks,
  which cause a constant strong force between particles regardless of their
  separation. Consequently the energy needed to separate quarks is proportional
  to the distance between them and at some point there is enough energy to
  favour the creation of a new quark pair. Thus before separating two quarks we
  create a quark\-/antiquark pair. We will probably never be able to
  observe an isolated quark. This phenomenon is referred to as colour
  confinement or simply confinement.
\item With the first \(\beta\) coefficient being positive we notice that for
  increasing scales (\(\mu_2>\mu_1\)) the coupling decreases logarithmically.
  This leads to asymptotic freedom, which states, that for high energies (small
  distances), the strong coupling becomes diminishing small and quarks and
  gluons do not interact. Thus in isolated baryons and mesons the quarks are
  separated by small distances, move freely and do not interact. \end{itemize}

From the \textsc{rge} we have seen, that not only the coupling but also the
masses carry an energy dependencies.

\subsubsection{Running quark mass}
The mass dependence on energy is governed by the \textit{anomalous mass
  dimension} \(\gamma(a_s)\). Its properties of the running quark mass can be
derived similar to the gauge coupling. Starting from integrating the
\textit{anomalous mass dimension} \cref{eq:anomalousMassDimension}
\begin{equation}
  \log \frac{m(\mu_2)}{m(\mu_1)} = \int_{a_s(\mu_1)}^{a_s(\mu_2)} \dif a_s \frac{\gamma(a_s)}{\beta(a_s)}
\end{equation}
we can approximate the \textit{anomalous mass dimension} to first order and
solve the integral analytically \cite{Schwab2002}
\begin{equation}
  m(\mu_2) = m(\mu_1)\left( \frac{a(\mu_2)}{a(\mu_1)} \right)^{\frac{\gamma_1}{\beta_1}} \left( 1 + \mathcal{O}(\beta_2, \gamma_2) \right).
\end{equation}
As \(\beta_1\) and \(\gamma_1\) (see \ref{app:gammaCoefficients}) are positive
the quark mass decreases with increasing \(\mu\). The general relation between
different scales is given by
\begin{equation}
  m(\mu_2) = m(\mu_1) \exp \left( \int_{a_s(\mu_1)}^{a_s(\mu_2)} \dif a_s \frac{\gamma(a_s)}{\beta(a_s)}  \right)
\end{equation}
and can be solved numerically to run the quark mass to the needed scale
\(\mu_2\). Both, the \(\beta\)\-/function and the anomalous mass dimension are
currently known up to the 5\textsuperscript{th} order and listed in the appendix
\ref{sec:betaCoefficients}.

We will make use of the anomalous dimension while running the quark masses for
\textsc{np} contributions, which include the quark masses at different
energy scales.

\textsc{qcd} in general has a precision problem caused by uncertainties and
largeness of the strong coupling constant \(\alpha_s\). The fine-structure
constant (the coupling of \textsc{qed}) is known to eleven digits, whereas the
strong coupling is only known to about four. Furthermore for low energies the
strong coupling constant is much larger than the fine-structure constant. E.g.
at the \(Z\) mass, the standard mass to compare the strong coupling, we have an
\(\alpha_s\) of \(0.11\), whereas the fine structure constant would be around
\(0.007\). Consequently to use \textsc{pt} we have to calculate our results to
much higher orders, including tens of thousands of Feynman diagrams, in
\textsc{qcd} to achieve a precision equal to \textsc{qed}. For even lower
energies, around \SI{1}{\giga\eV}, the strong coupling reaches a critical value
of around \(0.5\) leading to a break down of \textsc{pt}.

In this work we try to achieve a higher precision in the value of \(\alpha_s\).
The framework we use to measure the strong coupling constant are the
\textsc{qcdsr}. A central object needed to describe hadronic states with the
help of \textsc{qcd} is the \textit{two-point function} for which we will devote
the following section.

\section{Two-Point Function}
\label{sec:twoPointFunction}
In analogy to the Green's function for elemental fields we can define a propagator
for composite currents, referred to as \textit{two\-/point function}
\begin{equation}
  \Pi(x) = \langle\Omega\vert T\{J(x)J(y)\} \vert\Omega\rangle,
\end{equation}
where \(T\{\cdots\}\) is the time\-/ordered product and \(\vert\Omega\rangle\)
is the ground state/vacuum of the interacting theory. Note that the fields are
in general given in the Heisenberg picture, which implies translational
invariance.
\begin{equation}
  \begin{split}
    \langle\Omega\vert \phi(x)\phi(y) \vert\Omega\rangle &= \langle\Omega\vert \phi(x) e^{i\hat P y}e^{-i\hat P y}\phi(y)e^{i\hat P y}e^{-i\hat P y} \vert\Omega\rangle \\
    &= \langle\Omega\vert \phi(x-y)\phi(0) \vert\Omega\rangle,
  \end{split}
\end{equation}
where we made use of the translation operator \(\hat T(x) = e^{-i \hat P x}\).

In this work we are especially concerned about the vacuum expectation value of
the Fourier transform of two time\-/ordered \textsc{qcd} quark Noether currents
\begin{tcolorbox}[ams equation,myformula]
  \label{eq:qcdCorrelator}
  \Pi_{\mu\nu}(p^2) \equiv \int \fdif{x} e^{ipx} \langle\Omega\vert
  T\left\{J_\mu(x)J_\nu(0)\right\} \vert\Omega\rangle,
\end{tcolorbox}
where the Noether current is given by
\begin{equation}
  \label{eq:qcdCurrent}
  J_\mu(x) = \anti{q}(x) \Gamma q(x).
\end{equation}
Here, \(\Gamma\) can be any of the following dirac matrices \(\Gamma \in \{ 1,
i\gamma_5, \gamma_\mu, \gamma_\mu\gamma_5\}\), specifying the quantum number of
the current (\define{s}{scalar}, \define{p}{pseudo-scalar}, \define{v}{vector}
and \define{a}{axial\-/vector}, respectively). By choosing the right quantum
numbers we can theoretically represent the processes we want to study, which
will be important when we want to theoretically describe the hadrons produced in
\(\tau\) decays.

\begin{wrapfigure}{r}{4cm}
  \centering
  \begin{tkizpicture}
    \feynmandiagram [small, vertical=a to b] { i1 [particle=\(\anti\tau\)] --
      [anti fermion] a -- [anti fermion] i2 [particle=\(\anti\nu_\tau\)], a --
      [scalar] b, b -- [fermion, half left, edge label=\(\anti{q}\)] c --
      [fermion, half left, edge label=\(q\)] b, c -- [scalar] d, f1
      [particle=\(\tau\)]-- [fermion] d -- [fermion] f2 [particle=\(\nu_\tau\)],
    };
  \end{tkizpicture}
  \captionsetup{format=plain}
  \caption{\(\tau\anti{\tau}\)\-/annihilation with a quark\-/antiquark pair.}
  \label{fig:tauAntiTauAnnihilation}
\end{wrapfigure}
From a Feynman diagram point of view we can illustrate the two\-/point function
as quark\-/antiquark pair, which is produced by an external source, e.g. the
virtual \(W\) boson of \(\tau\anti{\tau}\) annihilation as seen in
\cref{fig:tauAntiTauAnnihilation}. Here the quarks are propagating at
\textit{short distances}, which implies that we can make use of \textsc{pt},
thus avoiding \textit{long\-/distance} (\textsc{npt}) effects, that would
appear if the initial and final states where given by hadrons
\cite{Colangelo2000}. It is interesting to note, that the same process with the
help of the \textit{optical theorem} can be used to derive the total decay width
of hadronic tau decays.

\subsection{Short Distances vs. Long Distances}
If we want to calculate the two\-/point function in \textsc{qcd} we have to
differentiate short and long distances (large or small momenta). In
general when we talk about small distances we refer to large momenta. Large
momenta implies a small strong coupling. Consequently we can use
\textsc{pt} for short distances without problems. On the contrary long
distances involve small momenta, which implies a large coupling constant. Thus
for long distances the \textsc{np} effects become important and have to be dealt
with. To apply \textsc{pt} to the case of the \(\tau\anti{\tau}\) annihilation
we need the quark\-/antiquark pair of \cref{fig:tauAntiTauAnnihilation} to be
highly virtual\footnote{Which is the same of saying, that the quark\-/antiquark
  pair needs a high external momentum \(q\).}. To roughly separate
long distances from short distances using a length scale we can say that the
length scale should be smaller than the radius of a hadron.

\subsection{Relating the Two\-/Point Function to Hadrons}
The two\-/point function can be interpreted physically as the amplitude of
propagating single- or multi\-/particle states and their excitations. The
possible states, in our case, the hadrons we describe through the correlator,
are fixed by the quantum numbers of the current, we define for the vacuum
expectation value. For example the neutral \(\rho\) meson is a spin\-/1 vector
meson with a quark content of \((u\anti{u} - d\anti{d})/\sqrt{2}\). Consequently
by choosing a current
\begin{equation}
  J_\mu(x) = \frac{1}{2} ( \anti{u}(x)\gamma_\mu u(x) - \anti{d}(x)\gamma_\mu d(x) )
\end{equation}
the two\-/point function contains the same quantum numbers as the \(\rho\) meson
and is said to materialise to it. A list of some ground\-/state mesons for
combinations of the light quarks \(u, d\) and \(s\) is given in
\cref{table:groundStateMesons}.
\begin{table}
  \centering
  \begin{tabular*}{\textwidth}{lccc @{\extracolsep{\fill}}c}
    \toprule
    Symbol & Quark content & Isospin & \(J\) & Current \\
    \midrule
    \(\pi^+\)  & \(u\anti{d}\) & \(1\) & \(0\)
                                             & \(:u\gamma_\mu\gamma_5\anti{d}:\) \\
    \(\pi^0\)  & \(\frac{u\anti{u} - d\anti{d}}{2}\) & \(1\) & \(0\)
                                             & \(:\anti{u}\gamma_\mu\gamma_5 u + \anti{d}\gamma_\mu\gamma_5 d:\) \\
    \(\eta\)   & \(\frac{u\anti{u} + d\anti{d} - 2s\anti{s}}{\sqrt{6}}\) & \(0\)
                                     & \(0\) & \(:\anti{u}\gamma_\mu\gamma_5 u + \anti{d}\gamma_\mu\gamma_5d
                                               - 2\anti{s}\gamma_\mu\gamma_5 s:\) \\
    \(\eta\prime\) & \(\frac{u\anti{u} + d\anti{d} + s\anti{s}}{\sqrt{3}}\)
                           & \(0\) & \(0\) & \(:\anti{u}\gamma_\mu\gamma_5 u +
                                             \anti{d}\gamma_\mu\gamma_5 d + \anti{s}\gamma_\mu\gamma_5 s:\) \\
    \(\rho^0\) & \(\frac{u\anti{u} - d\anti{d}}{\sqrt{2}}\) & \(1\) & \(1\)
                                             & \(:\anti{u}\gamma_\mu u - \anti{d}\gamma_\mu d:\) \\
    \(\omega\) & \(\frac{u\anti{u}+d\anti{d}}{\sqrt{2}}\) & \(0\) & \(1\)
                                             & \(:\anti{u}\gamma_\mu u + \anti{d}\gamma_\mu d:\) \\
    \(\phi\) & \(s \anti{s}\) & \(0\) & \(1\)
                                             & \(:\anti{s}\gamma_\mu\gamma_5 s\): \\                                        
    \(K^+\) & \(u \anti{s}\) &  \(\frac{1}{2}\) & \(0\) &
                                                          \(:u\gamma_\mu\gamma_5\anti{s}:\) \\
    \(K^0\) & \(d \anti{s}\) &  \(\frac{1}{2}\) & \(0\) &
                                                          \(:d\gamma_\mu\gamma_5\anti{s}:\) \\
    \bottomrule
  \end{tabular*}
  \caption{Ground\-/state vector and pseudoscalar mesons for the light quarks
    \(u, d\) and \(s\) with their corresponding currents in the two\-/point
    function. Note that we use \(\gamma_\mu\) for vector and
    \(\gamma_\mu\gamma_5\) for the pseudoscalar mesons.}
  \label{table:groundStateMesons}
\end{table}

The correlator is materialising into a spectrum of hadrons. Thus if we insert a
complete set of states of hadrons we can make use of the unitary relation
\begin{equation}
  \langle\Omega\vert J_\mu(x)J_\nu(0) \vert\Omega\rangle = \sum_X \langle\Omega\vert J_\mu(x) \vert X\rangle \langle X\vert J_\nu(0) \vert\Omega\rangle
\end{equation}
to represent the two\-/point correlator via a spectral function \(\rho(t)\)
\begin{tcolorbox}[ams equation,myformula]
  \label{eq:KallenLehmanSpectralDecomposition}
  \Pi(p^2) = \int_0^\infty \dif s \frac{\rho(s)}{s-p^2-i\epsilon}.
\end{tcolorbox}
The above relation is referred to as \textit{Källén-Lehmann spectral
  representation} \cite{Kallen1952,Lehmann1954} or \textit{dispersion relation}.
It relates the two\-/point function to the spectral function \(\rho\), which can
be represented as sum over all possible hadronic states
\begin{equation}
  \rho(s) = (2\pi)^3 \sum_X \int \dif \Pi_X \abs{\langle\Omega\vert J_{\mu}(0) \vert X\rangle}^2 \delta^4(s-p_X).
\end{equation}
Note that the analytic properties of the two\-/point function are in
one\-/to\-/one correspondence with the newly introduced spectral function and
thus determined by the possible hadrons states, which only form on the positive
real axis. A full derivation of the \textit{Källén\-/Lehmann spectral
  representation} can be found in \cite{Rafael1997}. The spectral function is
interesting to us for two reasons. First, it is experimentally measurable and
second it carries a ``branch cut'', which we want to discuss now.

\subsection{Analytic Structure of the Two\-/Point Function}
The general two\-/point function \(\rho(s)\) has some interesting analytic
properties. It has poles for single particle states and a continuous branch
cut for multi particle states. The single and multi particle states, for a
general correlator, can mathematically be separated by
\begin{equation}
  \rho(s) = Z \delta(s-m^2) + \theta(s-s_0)\sigma(s),
\end{equation}
where the second term is the contribution from multi particle states.
\(\sigma(s)\) is zero till we reach the threshold, where we have sufficient
energy to form multi particle states. The analytic structure is depicted by
\cref{fig:analyticStructureCorrelator} and we can see that the spectral function
has \(\delta\) spikes for single particle states and a continuous
contribution for \(s\geq 4m\) resulting from multi particle states. These lead
to poles and a continuous branch cut of the two\-/point function.
\begin{figure}
  \centering
  \includegraphics[width=\textwidth]{./images/analyticStructureCorrelator.eps}
  \caption{Analytic structure in the complex \(q^2\)-plane of the Fourier
    transform of the two-point function. The hadronic final states are
    responsible for poles appearing on the real-axis. The single particle states
    contribute as isolated pole and the multi particle states contribute as
    bound states poles or a continues ``discontinuity cut''
    \cite{Peskin1995,Zwicky2016}.}
  \label{fig:analyticStructureCorrelator}
\end{figure}

\subsection{Decompositions}
Apart the spectral decomposition we can also Lorentz decompose the two\-/point
function or write it in terms of \textsc{v}, \textsc{a}, \textsc{s} and
\textsc{p} contributions.

\subsubsection{Lorentz decomposition}
Due to the Lorentz invariance of the two\-/point function, and by assuming the
conservation of the Noether current, we can apply the Ward identity to decompose
the correlator \(\Pi_{\mu\nu}\) into its scalar contribution \(\Pi\).

There exist only two possible terms that can guard the structure of the second
order tensor: \(q_\mu q_\nu\) and \(q^2 g_{\mu\nu}\). The sum of both multiplied
with two arbitrary functions \(A(q^2)\) and \(B(q^2)\) yields
\begin{equation}
  \Pi_{\mu\nu}(q^2) = q_\mu q_\nu A(q^2) + q^2 g_{\mu\nu} B(q^2).
\end{equation}
By assuming that we deal with equal quark flavours and that the vector current
is conserved, i.e. \(\partial^\mu j_\mu(x) = 0\), we can make use of the
\textit{Ward identity}
\begin{equation}
  \label{eq:wardIdentity}
  q^\mu \Pi_{\mu\nu} = 0
\end{equation} 
% \begin{equation}
%   \label{eq:wardIdentity}
%   \begin{split}
%     q^\mu \Pi_{\mu\nu} &= \int \dif x q^\mu e^{iqx} \langle 0 \vert  J_\mu(x) J_\nu(0) \vert 0 \rangle \\
%     &= -i \int \dif x i q^\mu e^{i q^\nu x_\nu} \langle 0 \vert J_\mu(x) J_\nu(0) \vert 0 \rangle \\
%     &= i \int \dif x e^{iqx} \langle 0 \vert \partial_\mu[J_\mu(x)] J_\nu(0)
%     \vert 0 \rangle \\
%     &= 0, \quad \text{with} \quad \partial_\mu J_\mu(x) = 0,
%   \end{split}
% \end{equation}
% where we used \(i q^\mu e^{i q^\nu x_\nu} = \partial_\mu e^{i q^\nu x_\nu}\)
% in the second and integration by parts in the third line.
to demonstrate, that the two arbitrary functions are related
\begin{equation}
  \begin{split}
    q^\mu q^\nu \Pi_{\mu\nu} &= q^4 A(q^2) + q^4 B(q^2) = 0 \\
    &\quad \implies A(q^2) = -B(q^2).
  \end{split}
\end{equation}
Thus redefining \(A(q^2) \equiv \Pi(q^2)\) we expressed the correlator as a
scalar function of spin \(1\)
\begin{equation}
  \Pi_{\mu\nu}(q^2) = (q_\mu q_\nu - q^2 g_{\mu\nu})\Pi^{(1)}(q^2).
\end{equation}
In case of a current of different quark flavours, the current will not be
conserved and we cannot apply the Ward identity. Consequently the standard
Lorentz decomposition into transversal and longitudinal components reads
\begin{tcolorbox}[ams equation,myformula]
  \label{eq:standardLorentzDecomposition}
  \Pi^{\mu\nu}(q^2) = (q^\mu q^\nu - g^{\mu\nu}q^2) \Pi^{(1)}(q^2) + q^\mu q^\nu
  \Pi^{(0)}(q^2).
\end{tcolorbox}

\subsubsection{Transversal and Longitudinal Relations}
By comparing the standard Lorentz decomposition
(\cref{eq:standardLorentzDecomposition}) with the decomposition into
\textsc{v}/\textsc{a} and \textsc{s}/\textsc{p} parts we can identify the
longitudinal components of the correlator as being purely scalar. The latter
decomposition can be written as \cite{Broadhurst1981,Jamin1992}
\begin{equation}
  \label{eq:correlatorVectorScalarDecomposition}
  \begin{split}
    q^2\Pi^{\mu\nu}(q^2) &= (q^\mu q^\nu - q^2 g^{\mu\nu}) \Pi^{V,A}(q^2) + g^{\mu\nu} (m_i \mp m_j)\Pi^{S,P}(q^2) \\
    &+ g^{\mu\nu}(m_i \mp m_j) \left[ \langle\Omega\vert \anti{q}_i q_i
      \vert\Omega\rangle \mp \langle\Omega\vert \anti{q}_j q_j
      \vert\Omega\rangle\right],
  \end{split}
\end{equation}
where the third term is a correction arising due to the physical vacuum
\(|\Omega\rangle\). By multiplying \cref{eq:correlatorVectorScalarDecomposition}
by two four-momenta and making use of the Ward identity \cref{eq:wardIdentity}
we can write
\begin{equation}
  q_\mu q_\nu \Pi^{\mu\nu}(q^2) = (m_i \mp m_j)^2 \Pi^{S,P}(q^2) + (m_i \mp m_j)[\langle \anti{q}_i q_i \rangle \mp \langle \anti{q}_j q_j \rangle],
\end{equation}
which then can be related to the longitudinal component of
\cref{eq:standardLorentzDecomposition} by comparison with
\begin{equation}
  q_\mu q_\nu \Pi^{\mu\nu}(q^2) = q^4 \Pi^{(0)}(q^2) = s^2 \Pi^{(0)}(s) \quad \text{with} \quad s\equiv q^2,
\end{equation}
leading to
\begin{equation}
  \label{eq:longitudinalTwoPointFunction}
  s^2 \Pi^{(0)}(s) = (m_i \mp m_j)^2 \Pi^{(S,P)}(s) + (m_i \mp m_j)[ \langle \anti{q}_i q_i \rangle \mp \langle \anti{q}_j q_j \rangle].
\end{equation}
Note that all appearing mass terms are related to the longitudinal component.

As the \(\tau\) decays, with the limiting factor of the tau mass, can only decay into
light quarks we will often neglect the quark masses and work in the so\-/called
\textit{chiral limit} (\(m_q \to 0\)), in which the longitudinal component is
going to vanish.

By defining a combination of the transversal and longitudinal correlator
\begin{tcolorbox}[ams equation,myformula]
  \label{eq:correlatorCombination}
  \Pi^{(1+0)}(s) \equiv \Pi^{(1)}(s) + \Pi^{(0)}(s)
\end{tcolorbox}
we can additionally relate the transversal and vectorial components via
\begin{equation}
  \label{eq:longitudinalCorrelator}
  \begin{split}
    \Pi^{\mu\nu}(s) &= \underbrace{(q^\mu q^\nu - g^{\mu\nu}q^2)\Pi^{(1)}(s) +
      (q^\mu q^\nu - g^{\mu\nu} q^2)\Pi^{(1)}(s)}_{=(q^\mu q^\nu - g^{\mu\nu}
      q^2) \Pi^{(1+0)}(s)} + \frac{g^{\mu\nu}s^2}{q^2}\Pi^{(0)}(s),
  \end{split}
\end{equation}
such that
\begin{equation}
  \Pi^{(V,A)}(s) = \Pi^{(1)}(s) + \Pi^{(0)}(s) = \Pi^{(1+0)}(s),
\end{equation}
where the vector/axial\-/vector component of the correlator is now related to the
newly defined transversal and longitudinal combination of the correlator.

Having dealt exclusively with the perturbative part of the theory, we have to
discuss \textsc{np} contributions. These arise due to non negligible long
distance effects. Thus to complete the needed ingredients for the \textsc{qcdsr}
we need a final ingredient the \textsc{ope}, which
treats the \textsc{np} contributions of our theory.



\section{Operator Product Expansion}
\label{sec:ope}
The \textsc{ope} was introduced by Wilson in 1969 \cite{Wilson1969} as an
alternative to the in this time commonly used current algebra. The expansion
states that products of operators at different space\-/time points can be
rewritten into a sum of composite local operators and their corresponding
coefficients:
\begin{tcolorbox}[ams equation,myformula]
  \label{eq:ope}
  \lim_{x\to y} A(x) B(y) = \sum_n C_n(x-y)\mathcal{O}_n(x),
\end{tcolorbox}
where \(C_n(x-y)\) are the so-called \textit{Wilson coefficients} and \(A, B\)
and \(\mathcal{O}_n\) are operators.

The \textsc{ope} lets us separate short distances from long distances. In pure
\textsc{pt} we can only amount for short distances, which are equal to high
energies, where the strong coupling \(\alpha_s\) is small. The \textsc{ope} on
the other hand accounts for long\-/distance effects with higher dimensional
operators. Applying the \textsc{ope} to the two\-/point function we get a sum
over the vacuum expectation values
\begin{equation}
  \Pi_{OPE}(q^2) = -\frac{1}{3 q^2} \sum_n \langle\Omega\vert \mathcal{O}_n(0) \vert\Omega\rangle
  \int \dif^4 x e^{iqx} C_n(x).
  \label{eq:opeTwoPointFunction}
\end{equation}

The form of the composite operators are dictated by gauge and Lorentz symmetry.
For the two\-/point function in \cref{eq:opeTwoPointFunction} we only have to
consider operators \(\mathcal{O}_n\) of dimension
\begin{equation}
  d \left( \mathcal{O}_n \right) \leq (D - 4) + 2N.
\end{equation}
The scalar operators up to dimension six are then given by \cite{Pascual1984}
\begin{equation}
  \begin{array}{ll}
    \text{Dimension 0:} & \mathbb{1} \\
    \text{Dimension 4:} & :m_i \anti{q} q: \\
                        & :G_a^{\mu\nu}(x) G_{\mu\nu}^a(x): \\
    \text{Dimension 6:} & :\anti{q} \Gamma q \anti{q} \Gamma q: \\
                        & :\anti{q} \Gamma \frac{\lambda^a}{2} q_\beta(x) \anti{q} \Gamma \frac{\lambda^a}{2} q: \\
                        & :m_i \anti{q} \frac{\lambda^a}{2} \sigma_{\mu\nu} q G^{\mu\nu}_a: \\
                        & :f_{abc} G_a^{\mu\nu} G_b^{\nu\delta} G_c^{\delta\mu}:,
  \end{array}
\end{equation}
where \(\Gamma\) stands for one of the possible dirac matrices (as seen
\cref{eq:qcdCurrent}). Note, that the \(D=2\) operator violates gauge symmetry
and is consequently excluded from our list. Within \textsc{pt} only the unit
operator would exist, as the higher dimensional operators would appear as normal
ordered products of fields and vanish by being sandwiched into the perturbative
vacuum. On the contrary, in \textsc{np qcd} they appear as \textit{condensates}.
Condensates are the vacuum expectation values of non\-/vanishing normal ordered
fields by applying the full \textsc{qcd} vacuum, which contribute to all strong
processes. For example the condensates of dimension four are the
quark-condensate \(m_i\langle \anti{q} q \rangle\) and the gluon-condensate
\(\langle GG \rangle\).

As long as working with dimensionless functions (e.g. the correlator \(\Pi\) in
\cref{eq:opeTwoPointFunction}), the \define{rhs}{right\-/hand side} of
\cref{eq:ope} has to be dimensionless. As a result the Wilson coefficients have
to cancel the dimension of the operator with their inverse mass dimension. To
account for the dimensions we can make the inverse momenta explicit
\begin{equation}
  \Pi_{V/A}^{OPE}(s) = \sum_{D=0,2,4\dots} \frac{C^{(D)}
    \langle\Omega\vert \mathcal{O}^{(D)}(x) \vert\Omega\rangle}{(-q^2)^{D/2}},
\end{equation}
where we used \(c^{(D)}=C^{(D)}/(-s)^{D/2}\) with \(D\) being the dimension.
Thus the \textsc{ope} should converge with increasing dimension for sufficiently
large momenta \(s\).

\subsection{A practical example}
Let us show how the \textsc{ope} contributions are calculated with a standard
example \cite{Shifman1978, Pascual1984}. We will compute the perturbative and
quark condensate Wilson coefficients for the \(\rho\) meson. To do so we have to
evaluate Feynman diagrams using standard \textsc{pt}.

The \(\rho\) meson is a vector meson of isospin one composed of \(u\) and \(d\)
quarks. As a result (see. \cref{table:groundStateMesons}) we can match its
quantum numbers with the current
\begin{equation}
  J^\mu(x) = \frac{1}{2}\left(:[\anti{u} \gamma^\mu u](x) - [\anti{d} \gamma^\mu d](x):\right).
\end{equation}
\begin{figure}
  \centering
  \includegraphics[width=\textwidth]{./images/condensateFeynmanDiagram.eps}
  \caption{Feynman diagrams of the perturbative (a) and the quark-condensate (b)
    contribution. The upper part of the right diagram is not Wick contracted and
    responsible for the condensate.}
  \label{fig:OPEFeynmanDiagram}
\end{figure}
Pictorial the dimension zero contribution is given by the quark\-/antiquark loop
Feynman diagram in \cref{fig:OPEFeynmanDiagram}. The higher dimension contributions
are given by the same Feynman diagram, but with non contracted fields. These non
contracted fields contain the condensates. Thus not contracting the
quark\-/antiquark field (see. \cref{fig:OPEFeynmanDiagram} b) will give us
access to the Wilson coefficient of the dimension four quark condensate \(m_i
\langle \anti{q}q \rangle\).

The perturbative part (the Wilson coefficient of dimension zero) can than be
taken from the mathematical expression for the scalar correlator
\begin{equation}
  \label{eq:rhoScalarCorrelator}
  \begin{split}
    \Pi(q^2) &= - \frac{i}{4 q^2 (D -1)} \int \dif^D x e^{iqx} \langle \Omega \vert T \{:\anti{u}(x) \gamma^\mu u(x) - \anti{d}(x) \gamma^\mu d(x): \\
    &\quad\times :\anti{u}(0 \gamma_\mu u(0) - \anti{d}(0)\gamma_\mu d(0): \}
    \rangle.
  \end{split}
\end{equation}
To extract the dimension zero Wilson coefficient we apply Wick's theorem to
contract all of the fields, which represents the lowest order of the
perturbative contribution. The calculation is solely using standard \textsc{pt}
and we will restrict ourselves in displaying the result and omitting the
calculation\footnote{The interested reader can follow \cite{Pascual1984} for a
  detailed calculation.}
\begin{equation}
  \begin{split}
    \Pi(q^2) &= \frac{i}{4q^2(D-1)} (\gamma^\mu)_{ij} (\gamma_\mu)_{kl} \int \dif^D x e^{iqx} \\
    &\times\quad\left[ \wick{ \c u_{j \alpha}(x) \anti{\c u}_{k \beta}(0)} \cdot
      \wick{\c u_{l \beta}(0) \anti{\c u}_{i \alpha}(x)} + (u \to d)
    \right] \\
    &= \frac{3}{8\pi^2} \left[ \frac{5}{3} - \log\left( -\frac{q^2}{\nu^2}
      \right) \right].
  \end{split}
\end{equation}

To calculate the higher dimensional contributions of the \textsc{ope} we use the
same techniques as before. We apply Wick's theorem, but in this case, due to the
\textsc{np} vacuum, we have non\-/vanishing vacuum expectation value of normal
ordered products of fields. Thus some of the fields are left uncontracted, as
can be graphically seen in \cref{fig:OPEFeynmanDiagram}. For leaving the quark
field uncontracted in \cref{eq:rhoScalarCorrelator} we get
\begin{equation}
  \begin{split}
    \Pi(q^2) &= \frac{i}{4q^2(D-1)} (\gamma^\mu)_{ij} (\gamma_\mu)_{kl} \int \dif^D x e^{iqx} \left[ \phantom{\wick{ \c u_{j \alpha}(x) \anti{\c u}_{k \beta}(0)}} \right.\\
    &\quad+\,\wick{ \c u_{j \alpha}(x) \anti{\c u}_{k \beta}(0)} \cdot \langle \Omega \vert : \anti{u}_{i \alpha}(x) u_{l \beta} (0) :\vert \Omega \rangle \\
    &\quad\,\left.+\wick{\c u_{l \beta}(0) \anti{\c u}_{i \alpha}(x)} \cdot
      \langle \Omega \vert: \anti{u}_{k \beta}(0) u_{j \alpha}(x) : \vert \Omega
      \rangle + (u \to d) \right],
  \end{split}
\end{equation}
where \((u\to d)\) is representing the previous expressions with \(u\) and \(d\)
interchanged. Here we can observe the condensates as non\-/vanishing vacuum
values of normal ordered product of fields:
\begin{equation}
  \langle\Omega_{QCD}\vert: \anti{q}(x) q(0) :\vert\Omega_{QCD}\rangle \neq 0.
\end{equation}
We emphasised the \textsc{qcd} vacuum \(\Omega_{QCD}\), which is responsible for
vacuum expectation values different than zero. E.g. for a vacuum of \textsc{qed}
this contributions would vanish by definition. Pictorial the condensates take
form of unconnected propagators, sometimes marked with an \(\times\), as seen in
\cref{fig:OPEFeynmanDiagram}.

To make the non\-/contracted fields local, we can expanded them in \(x\)
\begin{equation}
  \begin{split}
    \langle \Omega \vert: \anti{q}(x) q(0):\vert \Omega \rangle &= \langle\Omega\vert: \anti{q}(0) q(0): \vert\Omega\rangle\\
    &\quad+ \langle\Omega\vert:\left[ \partial_\mu \anti{q}(0) \right]
    q(0):\vert\Omega\rangle x^\mu + \dots,
  \end{split}
\end{equation}
where terms with derivatives lead to higher dimensional operators, which can be
seen by applying the equation of motions. We then can focus on the first term
and introduce a standard notation for the localised condensate
\begin{equation}
  \langle \anti{q}q \rangle \equiv \langle\Omega\vert: \anti{q}(0) q(0):\vert\Omega\rangle.
\end{equation}
Finally, the contribution to the \(\rho\) scalar correlator is then given by the
following expression
\begin{equation}
  \Pi_{(\rho)}(q^2) = \frac{1}{2} \frac{1}{\left(Q\right)^2}
  \left[ m_u\langle \anti{u} u \rangle + m_d \langle \anti{d} d \rangle\right],
\end{equation}
where we defined \(Q \equiv -q^2\). Here we can clearly see that for dimension four we
get a factor of \(1/(Q)^2\), which is responsible for the suppression of the
series. The condensates \(\langle\anti{u}u\rangle\) and
\(\langle\anti{d}d\rangle\) are numbers, that have to be derived by
phenomenological fits or computed from \textsc{lqcd}. Fortunately once found,
the value of the condensate can be used for any process.

In summary we note that the usage of the \textsc{ope} and its validity is far
from obvious. Until today there is no analytic proof of the \textsc{ope}.
Furthermore we are deriving the \textsc{ope} from matching the Wilson
coefficients to Feynman graph analyses. These Feynman graphs are calculated
perturbatively but the coefficients with dimension \(D>0\) correspond to
\textsc{np} condensates! The condensates by themselves have to be gathered from
external, \textsc{np} methods.

Now that we have a tool to deal with the \textsc{qcd} vacuum and \textsc{npt}
effects we are left with two problems. First, we still do not know how to deal
with hadronic states in the quark\-/gluon picture. This will be tackled by
duality. Secondly, we have seen that we can access the two\-/point function
theoretically on the physical sheet except for the positive real axis, due to
its analytic properties. Unfortunately the experimental measurable spectral
function is solely be defined on this positive real axis, which is theoretically
not accessible. To match the theory with the experiment we will have to apply
Cauchy's theorem. In the final section of this chapter we will combine
the two\-/point function, the \textsc{ope}, duality and Cauchy's theorem to
formulate the \textsc{qcdsr}.



\section{Sum Rules}
\label{sec:sumRules}
The \textsc{qcdsr} are a method to connect the \define{dof}{degrees of freedom}
of \textsc{qcd}, the quarks and gluon fields, to the \textsc{dof} of the vacuum
spectrum of hadrons, thereby allowing for the determination of the strong
coupling. To do so we have to treat the in \cref{sec:twoPointFunction}
introduced two\-/point function \textsc{np} with the help of the \textsc{ope}
\begin{equation}
  \Pi(s) \to \Pi_{OPE}(s).
\end{equation}
\textsc{qcdsr} furthermore introduce an ad hoc assumption, namely
\textit{quark\-/hadron duality}, stating that the observable hadron picture can
be equally described by the \textsc{qcd} quark\-/gluon picture and that both
pictures are equally valid. As the experimentally measured hadronic states are
represented in poles and cuts on the positive real axis of the two\-/point
function, which we have encountered in the analytic properties of its spectral
decomposition, we will follow the prescription of \textsc{qcdsr} to apply
\textit{Cauchy's theorem} and weight functions to take care of perturbative
complications close to the positive real axis.


\subsection{The Dispersion Relation}
We have already seen the Källén-Lehmann spectral representation in
\cref{eq:dispersionRelation}. The general dispersion relation is defined to have
an additional polynomial function \(P(s)\)
\begin{tcolorbox}[ams equation,myformula]
  \label{eq:dispersionRelation}
  \Pi(s) = \int_0^\infty \frac{\rho(s^\prime)}{s^\prime-s-i\epsilon} + P(s),
\end{tcolorbox}
which accounts for the fact, that the two\-/point function increases for large
\(s\), but the integral on the \textsc{rhs} cannot reproduce this behaviour. For
example the vector correlator carries only a constant and the scalar correlator
a linear polynom. The two\-/point function is in general an unphysical quantity,
whereas the spectral function \(\rho(s)\) is a physical quantity. As a result
the polynomial accounts carries the unphysical scale dependency of the
two\-/point function.

\subsection{Duality}
\label{sec:duality}
\textsc{qcd} treats quarks and gluon as its fundamental \textsc{dof}, but due to
confinement we are only ever able to observe hadrons. The mechanism that
connects the two worlds is the \textit{quark\-/hadron duality} (or simply
duality), which implies that physical quantities can be described equally good
in the hadronic as in the quark-gluon picture. Thus we can connect experimental
detected with theoretically calculated values from the two\-/point function in
the dispersion relation \cref{eq:dispersionRelation} as
\begin{equation}
  \Pi_{th}(s) = \int_0^\infty \frac{\rho(s^\prime)_{exp}}{s^\prime-s-i\epsilon} + P(s),
\end{equation}
where we connected the theoretical correlator \(\Pi_{th}\) with the experimental
measurable spectral function \(\rho_{exp}\). We can represent duality as,
substituting the two\-/point function \cite{Cata2005}
\begin{equation}
  \Pi(s) \to \Pi_{OPE}(s).
\end{equation}
If this approximation carries no error, we would say that the experimental
spectral function is dual to the \textsc{ope}. On the contrary if the
substitution is not exact we are missing contributions, which are represented by
so\-/called \textsc{dv}.


\subsubsection{Duality Violations}
There exist situations where we cannot make use of duality as an assumption.
These situation are referred to as \textsc{dv} and belong to the \textsc{np}
part of the theory. It is often assumed that by applying the \textsc{ope} to all
orders we account for all \textsc{np} effects, including \textsc{dv}.
Unfortunately this assumption is only partly right. Even if we could compute the
\textsc{ope} to all orders, we would still experience discrepancies to our
theoretical results. In general it is said, that if we have deviations beyond
the natural uncertainty of the \textsc{ope} we call them \textsc{dv}
\cite{Shifman2000}. E.g. if we compute \(\Pi(s)\) to orders of \(\alpha^2\) and
\(\frac{1}{Q^4}\), while we cutoff higher orders (\(\alpha^3\) and
\(\frac{1}{Q^6}\)) we get a natural error, because we have not calculated the
full series. Values of the hadronic spectral density, out of range of the
natural error, are then referred to as \textsc{dv}.

A detailed discussion of duality has been given by the Shifman in
\cite{Shifman2000}.


\subsection{Finite Energy Sum Rules}
To theoretically calculate the two\-/point function we have to integrate the
experimental data \(\rho_{exp}(s)\) from zero to infinity. No experiment will
ever take data for an infinite momentum \(s\). For \(\tau\) decays we are
limited to energies around the \(\tau\) mass of \SI{1.776}{\giga\eV}. To deal
with the upper integration limit several approaches have been made. One of them,
the \textit{Borel transform}, is to exponentially suppress higher energy
contributions (see \cite{Weinberg1996,Rafael1997}). The technique we are
focusing on is called \define{fesr}{finite energy sum rules} and introduces a
energy cut\-/off. We thus integrate the experimental data \(\rho(s)\) only to a
certain energy \(s_0\). Furthermore we have to theoretically evaluate the
integral over the spectral function of the dispersion relation
(\cref{eq:dispersionRelation}), which includes singularities caused by the
hadronic spectrum. As a result we have to apply Cauchy's theorem
\begin{equation}
  \oint_{\mathcal{C}} f(z) = 0,
\end{equation}
which states that any integral over an analytic function \(f(z)\) on a closed
contour \(\mathcal{C}\) has to be zero. Thus we can construct a contour to avoid
the positive problematic real axis. Pictorial the contour is drawn in
\cref{fig:theoreticalTwoPointFunction}
\begin{figure}
  \centering
  \includegraphics[width=0.8\textwidth]{./images/rTauCauchysTheorem.eps}
  \caption{Visualisation of the usage of Cauchy's theorem to transform
    \cref{eq:dispersionRelation} into a closed contour integral over a circle of
    radius \(s_0\).}
  \label{fig:theoreticalTwoPointFunction}
\end{figure}
and mathematically we can express it as
\begin{equation}
  \label{eq:sumRulesCauchysTheorem}
  \oint \Pi(s) = \int_0^{s_0} \Pi(s+i\epsilon)-\Pi(s-i\epsilon)\dif s
  + \int_{0+\alpha(\epsilon)}^{2\pi-\alpha(\epsilon)}\Pi(s_0e^{i\theta})\dif \theta + \int_{3\pi/2}^{\pi/2}\Pi(\epsilon e^{i\theta})\dif \theta.
\end{equation}
If we make to use of \textit{Schwartz reflection principle}:
\begin{equation}
  f(\overline{z}) = \overline{f(z)},
\end{equation}
which can be applied if \(f(z)\) is analytic and maps only to real values on the
positive real axis, we can express the integrand of the first integral of
\cref{eq:sumRulesCauchysTheorem} as the imaginary part of the two\-/point
function
\begin{equation}
  \Pi(s+i\epsilon)-\Pi(s-i\epsilon) = \Pi(s+i\epsilon)-\Pi^*(s+i\epsilon) = 2i\Ima\Pi(s+i\epsilon),
\end{equation}
which is by definition equal to the spectral function
\begin{equation}
  \label{eq:spectralFunction}
  \rho(s) \equiv \frac{\Ima\Pi(s)}{\pi}.
\end{equation}
After taking the limit of small \(\epsilon\) we can relate the line integral
with the lower limit zero and the upper limit \(s_0\) and the experimental
spectral function as integrand to a theoretical accessible circular contour
integral of radius \(s_0\)
\begin{equation}
  \int_0^{s_0} \rho(s) = \frac{-1}{2\pi i}\oint_{\abs{s}=s_0} \Pi(s) \dif s, \quad \text{where we applied} \quad \epsilon \to 0.
\end{equation}
Note that the unphysical contribution of the polynomial in
\cref{eq:dispersionRelation} cancel in the contour integral.

We are free to multiply the upper equation with an analytic function
\(\omega(s)\), which completes the \textsc{fesr}
\begin{tcolorbox}[ams equation,myformula]
  \label{eq:qcdSumRules}
  \int_0^{s_0} \omega(s) \rho(s) = \frac{-1}{2\pi i}\oint_{\abs{s}=s_0}\omega(s)
  \Pi_{OPE}(s) \dif s
\end{tcolorbox}
where the \define{lhs}{left\-/hand side} can be taken from experiment and the
\textsc{rhs} by the theoretically evaluated correlator \(\Pi_{OPE}(s)\). The
analytic function \(\omega(s)\) plays the role of a weight. It can be used to
further suppress the non\-/perturbative contributions coming from \textsc{dv}
and also enhance or suppress different contributions of the \textsc{ope} as we
will see.


\subsection{Weighting OPE dimensions}
We have seen that the perturbative part of the two\-/point function carries a
discontinuity on the positive real axis. Consequently we applied Cauchy's
theorem to avoid the non\-/analytic part of the two\-/point function. This left
us with non\-/closed contour integral for the perturbative part of the
\textsc{ope}, which will always contribute. On the other hand, the strength of
the higher dimension contributions of the \textsc{ope} can be modified. We can
use different weights to control the dimensions of the \textsc{ope} that
contribute. The weights we are using have to be analytic, so that we can make
use of Cauchy's theorem. Thus they can be represented as polynomials
\begin{equation}
  \omega(x) = \sum_i a_i x^i,
\end{equation}
every contributing monomial is responsible for a dimension of the \textsc{ope}.
Dimensions that are not represented in the weight polynomial do not contribute
at all or are very suppressed as we will demonstrate now.

The residue of a monomial \(x^k\) is only different from zero if its power
\(k=-1\):
\begin{equation}
  \label{eq:monomialWeights}
  \oint_{C} x^k \dif x = i \int_0^{2\pi}\left(e^{i \theta}\right)^{k+1} \dif \theta
  = \begin{cases} \mbox{\(2 \pi i\)} & \mbox{if } k=-1, \\ \mbox{0} & \mbox{otherwise} \end{cases}.
\end{equation}
We will see in discussing the total \(\tau\) decay ratio, that the integrand of
the closed\-/contour integral in \cref{eq:qcdSumRules} for the different
\textsc{ope} contributions is the weight function divided by a term proportional
to \(x^{D/2}\), where \(D\) is the dimension of the contributing \textsc{ope}
operator. If we regard solely a monomial as weight and neglect all terms of no
interest to us we can write
\begin{equation}
  \label{eq:monomialInclusiveRatio}
  \begin{split}
    \left. R^\prime(x) \right\rvert_{D=0,2,4\dots} &= \oint_{\abs{x}=1} \dif x \frac{x^k}{x^{\frac{D}{2}}}C^{D}(x) \\
    &= \oint_{\abs{x}=1} \dif x \,x^{k-D/2} \,C^{D}(x),
  \end{split}
\end{equation}
where \(C^{D}\) are the \(D\) dimensional Wilson coefficients. Thus combining
\cref{eq:monomialWeights} with \cref{eq:monomialInclusiveRatio} we see that only
Dimension which fulfil
\begin{equation}
  k - D/2 = -1 \quad \implies \quad  D = 2(k+1)
\end{equation}
contribute to the \textsc{ope}. For example the polynomial of the kinematic
weight
\begin{equation}
  \label{eq:kinematicWeight}
  \omega_\tau(s) \equiv \left( 1 - \frac{s}{m_\tau^2} \right) \left( 1 + 2 frac{s}{m_\tau^2} \right)
\end{equation},
which will appear naturally in the context of the total \(\tau\) decay ratio,
is given by
\begin{equation}
  (1 - x)^2 (1 + 2x) = \underbrace{1}_{D=2} - 3\underbrace{x^2}_{D=6} + 2\underbrace{x^3}_{D=8},
\end{equation}
where the underbraced monomials express the active dimensions. A list of
monomials and their corresponding dimensions up to dimension 14 can be found in
\cref{table:monomialDimensions}.
\begin{table}
  \centering
  \begin{tabular}{l|ccccccc}
    \toprule
    \textbf{monomial:} & \(x^0\) & \(x^1\) & \(x^2\) & \(x^3\) & \(x^5\) & \(x^6\) & \(x^7\)\\
    \textbf{dimension:} & \(D^{(2)}\) & \(D^{(4)}\) & \(D^{(6)}\) & \(D^{(8)}\) & \(D^{(10)}\) & \(D^{(12)}\) & \(D^{(14)}\)\\
    \bottomrule 
  \end{tabular}
  \caption{List of monomial and their corresponding ``active'' dimensions in the
    \textsc{ope}. Note that the perturbative contributions of the \textsc{ope}
    are always present.}
  \label{table:monomialDimensions}
\end{table}
This behaviour enables us to bring out different dimensions of the \textsc{ope}
and suppress contributions of higher order (\(D\geq10\)) for which less is
known.


For the interested reader we gathered several introduction texts to the
\textsc{qcdsr}, which where of great use to us
\cite{Narison1989,Rafael1997,Colangelo2000,Dominguez2013}.
\end{document}
% LocalWords:  qcdFeynmanDiagrams qcdFeynmanRules dispersionRelation sav itm
% LocalWords:  qcdCurrent twoPointFunction anomalousMassDimension sumRules lccc
% LocalWords:  OPEFeynmanDiagram correlatorComplexContour qcdSumRules ams iqx
% LocalWords:  LocalWords simpleTwoPointFunction twoPointFunctionSelfEnergy ipx
% LocalWords:  cuttingRules qcdLagrangian electronElectronScattering myformula
% LocalWords:  lambdaRegularisation groundStateMesons strongCouplingFirstOrder
% LocalWords:  runningOfAs tauAntiTauAnnihilation analyticStructureCorrelator
% LocalWords:  kallenLehmannSpectralRepresentation rhoScalarCorrelator ccccccc
% LocalWords:  kallenLehmanSpectralDecomposition theoreticalTwoPointFunction
% LocalWords:  sumRulesCauchysTheorem standardLorentzDecomposition wardIdentity
% LocalWords:  correlatorVectorScalarDecomposition inclusiveRatio
% LocalWords:  monomialWeights monomialInclusiveRatio monomialDimensions
% LocalWords:  opeTwoPointFunction
