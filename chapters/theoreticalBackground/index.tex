\documentclass[../../index.tex]{subfiles}

\begin{document}
\chapter{QCD Sum Rules}
\label{ch:theoreticalBackground}
It is most remarkably that we can describe the properties of quarks and gluons
via a local \textsc{qft} based on the gauge group $SU(3)$. However \textsc{qcd}
only applies to coloured particles and not to colourless particles like hadrons.
Due to confinement we can only ever observe hadrons, but our theoretical
foundation is ruled by the \textsc{dof} of quarks and gluons. To extract
\textsc{qcd}\-/parameters (the six quark\-/masses and the strong\-/coupling)
from hadrons we need bridge the quark\-/gluon picture with the hadron picture.
To do so we will introduce the framework of \textsc{qcdsr}. Setting up the
foundations of strong interaction we will discuss the standard tool of particle
physics, namely the \textsc{qcd}\-/Lagrangian. The \textsc{qcd}\-/Lagrangian
describes the quark\-/gluon picture solely and is ruled by the abelian gauge
group $SU(3)$. The group has important implications on the strong coupling and the
applicability of \textsc{pt}. Next we will focus on the two\-/point function,
which plays a major role in the framework of \textsc{qcdsr}. The two\-/point
function is defined as vacuum\-/expectation values of two local Noether\-/currents. We can use
it to theoretically describe processes, like $\tau$\-/decays into hadrons,
by matching the quantum numbers of the Noether\-/current, we choose in
specifying the two\-/point function, to the outgoing hadrons. We will see, that
the two\-/point function $\Pi(q^2)$ is related to hadronic states, by poles for
$q^2 > 0$. Here \textsc{np}\-/effects become important and we need to introduce
the \textsc{ope}, which handles \textsc{np} parts as
\textsc{qcd}\-/condensates. These condensates are remainders of the
\textsc{qcd}\-/vacuum $\Omega_{QCD}$, which in contrast to the normal\-/ordered
products of field in $\textsc{qed}$, do not vanish, but remain as parameters and have to
be phenomenological fitted or calculated by other \textsc{np} tools, like
\textsc{lqcd}. \textcolor{red}{...}

\textcolor{red}{edit} \\
have been discussed in many articles
\cite{Narison1989,Rafael1997,Colangelo2000,Dominguez2013} Duality first
introduced Poggio, Quinn and Weinberg \cite{Poggio1975}

\section{Quantum\-/chromodynamics}
\label{sec:quantumchromodynamics}
Since the formulation of \textsc{qed} in the end of the 40's it has been
attempted to describe the strong nuclear force as a \textsc{qft}, which has been
achieved in the 70's as \textsc{qcd}
\cite{GellMann1972,Fritzsch1973,Gross1973,Politzer1973,Weinberg1973}.
\textsc{qcd} is a renormalisable \textsc{qft} of the strong interaction, which
fundamental fields are given by dirac spinors of spin\-/$1/2$, the so\-/called
quarks, with a fractional electric charge of $\pm 1/3$ or $\pm 2/3$. The theory
furthermore contains gauge\-/fields of spin 1, which are chargeless, massless
and referred to as gluons. The gluons are the force\-/mediators, which interact
with quarks and themselves, in contrast to photons of \textsc{qed}, which
interact only with fermions (see \cref{fig:qcdFeynmanDiagrams}).

The corresponding gauge\-/group of \textsc{qcd} is the non\-/abelian group
$SU(3)$. Each of the quark flavours $u,d,c,s,t$ and $b$ belongs to the
fundamental representation of $SU(3)$ and contains a triplet of fields $\Psi$.
\begin{equation}
  \Psi = \begin{pmatrix} \Psi_1 \\ \Psi_2 \\ \Psi_3 \end{pmatrix}
  = \begin{pmatrix} \textcolor{red}{red} \\ \textcolor{green}{green} \\ \textcolor{blue}{blue} \end{pmatrix}
\end{equation}
The components of the triplet are titled colours\footnote{The colour
  denomination is not gauge\-/invariant. After a colour gauge transformation the
  new colours are a linear combination of the old colours, which breaks
  gauge\-/symmetry.} red, green and blue, which play the role of
\textit{colour\-/charge}, similar to the electric charge of \textsc{qed}. The
gluons belong to the adjoint representation of $SU(3)$, contain an octet of
fields and can be expressed using the Gell-Mann matrices $\lambda_a$
\begin{equation}
  B_\mu = B_\mu^a \lambda_a \qquad a = 1,2,\dotsc 8
\end{equation}
\begin{figure}
  \centering \includegraphics[width=\textwidth]{./images/qcdFeynmanDiagrams.eps}
  \caption{Feynman diagrams of the strong interactions with corresponding
    electromagnetic diagrams. We see that the gluons carry colour charge and
    thus couple to other gluons, which is not the case for the photons.}
  \label{fig:qcdFeynmanDiagrams}
\end{figure}
\begin{table}
  \centering
  \begin{minipage}[c]{0.4\textwidth}
    \begin{tabular}{ll}
      \toprule
      Flavour & Mass\\
      \midrule
      $u$ & \SI{3.48(24)}{\mega\eV} \\
      $d$ & \SI{6.80(29)}{\mega\eV} \\
      $s$ & \SI{130.0(18)}{\mega\eV} \\
      $c$ & \SI{1.523(18)}{\giga\eV} \\
      $b$ & \SI{6.936(57)}{\giga\eV} \\
      $t$ & \SI{173.0(40)}{\giga\eV} \\
      \bottomrule 
    \end{tabular}
  \end{minipage}\hfill
  \begin{minipage}[c]{0.59\textwidth}
    \caption{List of Quarks and their masses. The masses of the up, down,
      strange, charm and bottom quark are the renormalisation group invariant
      (RGI) quark masses and are quoted in the four-flavour theory ($N_f=2+1$)
      at the scale $\mu=\SI{2}{\giga\eV}$ in the $\overline{MS}$ scheme and are
      taken from the \textit{Flavour Lattice Averaging Group} \cite{FLAG2019}.
      The mass of the top quark is not discussed in \cite{FLAG2019} and has been
      taken from \cite{PDG2018} from direct observations of top events.}
  \end{minipage}
  \label{table:quarkList}
\end{table}
The classical \textit{Lagrange density} of \textsc{qcd} is given by
\cite{Jamin2006,Pascual1984}:

\begin{tcolorbox}[ams equation,myformula]
  \label{eq:qcdLagrangian}
  \mathcal{L}_{QCD}(x) = -\frac{1}{4}G_{\mu\nu}^a(x)G^{\mu\nu a}(x) + \sum_A \left[ \frac{i}{2} \bar{q}^A(x) \gamma^\mu \overleftrightarrow{D}_\mu q^A(x) - m_A\bar{q}^A(x) a^A(x) \right],
\end{tcolorbox}
with $q^A(x)$ representing the quark fields and $G_{\mu\nu}^a$ being the
\textit{gluon field strength tensor} given by:
\begin{equation}
  \label{eq:gluonField}
  G_{\mu\nu}^a(x) \equiv \partial_\mu B_\nu^a(x) - \partial_\nu^a(x) + g f^{abc} B_\mu^b(x) B_\nu^c(x),
\end{equation}
with $f^{abc}$ as \textit{structure constants} of the gauge\-/group \(SU(3)\) and
\(\overleftrightarrow{D}_\mu\) as covariant derivative acting to the left and to
the right. Furthermore we have used $A, B, \dotsc = 0, \dotsc 5$ as flavour
indices, $a, b, \dotsc = 0, \dotsc, 8 $ as colour indices and $\mu, \nu, \dotsc
= 0, \dotsc 3$ as lorentz indices. Explicitly the Lagrangian writes:
\begin{equation}
  \begin{split}
    \mathcal{L}_0(x) &=- \frac{1}{4} \left[ \partial_\mu G_\nu^a(x) - \partial_v G_\mu^a(x) \right] \left[ \partial^\mu G_a^\nu(x) - \partial^\nu G^\mu_a(x) \right] \\
    &\quad+ \frac{i}{2}\anti{q}_\alpha^A(x) \gamma^\mu \partial_\mu q_\alpha^A(x) - \frac{i}{2} \left[ \partial_\mu \anti{q}_\alpha^A(x) \right] \gamma^\mu q_\alpha^A(x) - m_A \anti{q}_\alpha^A(x) q_\alpha^A(x) \\
    &\quad+ \frac{g_s}{2} \anti{q}_\alpha^A(x) \lambda^a_{\alpha \beta}\gamma_\mu q_\beta^A(x) G^\mu_a(x) \\
    &\quad- \frac{g_s}{2}f_{abc}\left[ \partial_\mu G_\nu^a(x) - \partial_\nu G_\mu^a(x)\right] G_b^\mu(x) G_c^\nu(x) \\
    &\quad- \frac{g_s^2}{4} f_{abc}f_{ade} G_\mu^b(x) G_\nu^c(x) G_d^\mu(x)
    G_e^\nu(x)
  \end{split}
\end{equation}
The first term is the kinetic term for the massless gluons. The next three terms
are the kinetic terms for the quark field with different masses for each
flavour. The rest of the terms are the interaction terms. The fifth term
represents the interaction between quarks and gluons and the last two terms the
self\-/interactions of gluon fields.

Having derived the Lagrangian leaves us with its quantisation. The
dirac\-/spinors can be quantised as in \textsc{qed} without any problems. The
$\Psi(x)$ quantum field can be written as:
\begin{equation}
  \Psi(x) = \int \frac{\dif^3 p}{(2\pi)^3 2E(\vec p)} \sum_\lambda \left[ u(\vec p, \lambda) a(\vec p, \lambda) e^{-ipx} + v(\vec p, \lambda) b^\dagger (\vec p, \lambda) e^{ipx} \right],
\end{equation}
where the integration ranges over the positive sheet of the mass hyperboloid
$\Omega_+(m) = \{p \vert p^2 = m^2, p^0 > 0 \}$. The four spinors $u(\vec p,
\lambda)$ and $v(\vec p, \lambda)$ are solutions to the dirac equations in
momentum space
\begin{equation}
  \begin{split}
    [\slashed p - m]u(\vec p, \lambda) &= 0 \\
    [\slashed p + m]v(\vec p, \lambda) &= 0,
  \end{split}
\end{equation}
with $\lambda$ representing the helicity state of the spinors.

The quantisation of the gauge\-/fields are more cumbersome. One is forced to
introduce supplementary non\-/physical fields, the so\-/called Faddeev\-/Popov
ghosts $c^a(x)$ \cite{Faddeev1967}.

The free propagators for the quark\-/, the gluon\-/ and the ghost\-/fields are
then given by
\begin{equation}
  \begin{split}
    i S^{(0)AB}_{\alpha\beta}(x-y) &\equiv \wick{ \c q_\alpha^A(x) \anti{\c
        q}_\beta^B(y)} \equiv \langle 0 \vert T \{ q_\alpha^A(x)
    \anti{q}_\beta^B \} \vert 0 \rangle
    = \delta_{AB} \delta_{\alpha\beta} i S^{(0)}(x-y) \\
    &= i \delta_{AB}\delta_{\alpha\beta}\int \fdif{p} \frac{\slashed p + m}{(p^2 - m^2 + i\epsilon)}\\
    i D^{(0)\mu\nu}_{ab}(x-y) &\equiv \wick{ \c B_a^\mu(x) \c B_b^\nu(y)} \equiv
    \langle 0 \vert T \{ B_a^\mu(x) B_b^\nu(y) \} \vert 0 \rangle
    \equiv \delta_{ab}i \int \frac{\dif^4 k}{(2\pi)^4} D^{(0)\mu\nu}(k) e^{-ik(x-y)} \\
    &= i \delta_{ab} \int \frac{\dif^4 k}{(2\pi)^4} \frac{1}{k^2+i\epsilon} \left[ -g_{\mu\nu} + (1-a) \frac{k_\mu k_\nu}{k^2+i\epsilon} \right] e^{-ik(x-y)} \\
    i \widetilde{D}_{ab}^{(0)}(x-y) & \equiv \wick{\c \phi_a(x) \anti{\c
        \phi}_b(y)} \equiv \langle 0 \vert T\{ \phi_a(x) \anti{\phi}_b(y) \}
    \vert 0 \rangle
    = \frac{i}{(2\pi)^4}\delta_{ab} \int \dif^4 q \frac{-1}{q^2 + i\epsilon} e^{-q(x-y)} \\
    &\equiv \frac{i}{(2\pi)^4} \delta_{ab} \int \dif^4 q \widetilde{D}^{(0)}(q)
    e^{-iq(x-y)},
  \end{split}
\end{equation}
and the corresponding Feynman\-/rules have been displayed in
\cref{fig:qcdFeynmanRules}.
\begin{figure}
  \includegraphics[width=\textwidth]{./images/qcdFeynmanRules.eps}
  \caption{QCD Feynman rules.}
  \label{fig:qcdFeynmanRules}
\end{figure}

\subsection{Renormalisation Group}
The perturbations of the QCD Lagrangian in \cref{eq:qcdLagrangian} lead to
divergencies, which have to be \textit{renormalised}. Making these divergencies
finite is referred to as \textit{regularisation} and there are various
approaches:
\begin{itemize}
\item \label{itm:lambdaRegularisation}\textbf{$\bm{\lambda}$ regularisation:} In
  $Lambda$ regularisation we limit the divergent momentum integrals by a cutoff
  $\abs{\vec p} < \Lambda$. Here $\Lambda$ has the dimension of mass. In
  \textsc{qcd} $\Lambda$ marks the separation between short\-/ and
  long\-/distance effects. For momenta smaller than the cutoff ($\abs{\vec p} <
  \Lambda$) we probe short\-/distances. On the contrary for large momenta
  $\abs{\vec p} > \Lambda$ we have to include long\-/distance effects. We will
  see that we can make use of \textsc{pt} only for high\-/momenta
  (short\-/distances). The cutoff regularisation breaks translational
  invariance, which can be guarded by making use of other regularisation
  methods.
\item \textbf{\define{Pauli-Villars}{P-I} regularisation:} \cite{Pauli1949} In
  \textsc{P-I} regularisation the propagator is forced to decrease faster than
  the divergence to appear. It replaces the nominator by
  \begin{equation}
    (\vec p^2 + m^2)^{-1} \to (\vec p^2 + m^2)^{-1} - (\vec p^2 + M^2)^{-1},
  \end{equation}
  where $M$ has the dimension acts similar as the previously presented cutoff,
  but conserves translational invariance.
\item \textbf{Dimensional regularisation:}
  \cite{Bollini1972,tHooft1972,tHooft1973} Dimensional regularisation has been
  introduced in the beginning of the seventies to regularise non\-/abelian gauge
  theories (like \textsc{qcd}), where $\Lambda$\-/ and
  \textsc{P-V}\-/regularisation failed. In dimensional regularisation we expand
  the four space\-/time dimensions to arbitrary $D$-dimensions. To compensate
  for the additional dimensions we introduce an additional scale $\mu^{D-4}$. A
  typical Feynman\-/integral then has the following appearance:
  \begin{equation}
    \label{eq:dimRegFeynmanIntegral}
    \int \fdif{p} \frac{1}{\vec p^2 + m^2} \to \mu^{2\epsilon} \int \frac{\dif^D p}{(2\pi)^D}\frac{1}{\vec p^2+m^2},
  \end{equation}
  Dimensional regularisation preserves all symmetries, it allows an easy
  identification of divergences and naturally leads to the \define{ms}{minimal
    subtraction scheme} \cite{tHooft1973,Weinberg1973a}.
\end{itemize}
In all of the three regularisation schemes we introduced an arbitrary parameter
to regularise the divergence. This parameter causes an scale dependence of the
strong coupling and the quark masses. As we are mainly concerned with the
non\-/abelian gauge theory \textsc{qcd} we will focus on dimensional
regularisation, which introduced the parameter $\mu$. Measurable observables
(\textit{Physical quantities}) cannot depend on the renormalisation scale $\mu$.
In the Therefore the derivative by $\mu$ of a general physical quantity has to
yield zero. The physical quantity $R(q, a_s, m)$, that depends on the external
momentum q, the renormalised coupling $a_s\equiv\alpha_s/\pi$ and the
renormalised quark mass $m$ can then be expressed as
\begin{equation}
  \label{eq:RGE}
  \mu \od{}{\mu}R(q, a_s, m) = \left[ \mu \pd{}{\mu} + \mu \od{a_s}{\mu} \pd{}{m} + \mu \od{m}{\mu} \pd{}{m} \right] R(q, a_s, m) = 0.
\end{equation}
\Cref{eq:RGE} is referred to as \textbf{renormalisation group equation} and is
the basis for defining the two \textit{renormalisation group functions}:
\begin{align}
  \beta(a_s) &\equiv -\mu \od{a_s}{\mu} = \beta_1 a_s^2 + \beta_2 a_s^3 + \dots & \beta-\text{function}
                                                                                  \label{eq:betaFunction} \\
  \gamma(a_s) &\equiv - \frac{\mu}{m} \od{m}{\mu} = \gamma_1 a_s + \gamma_2 a_s^2 + \dots & \text{anomalous mass dimension}.
                                                                                            \label{eq:anomalousMassDimension}
\end{align}
The coefficients of the $\beta$-function and the mass anomalous dimension are
currently known up to the 5\textsuperscript{th} order and listed in the appendix
\ref{sec:betaCoefficients}.

\subsubsection{Running gauge coupling}
The $\beta$-function and the anomalous mass dimension are responsible for the
running of the strong coupling and the running of the quark mass respectively.
In this section we will shortly review the $\beta$-function and its implications
on the strong coupling, whereas in the following section we will discuss the
anomalous-mass dimension.

Regarding the $\beta$-function we notice, that $a_s(\mu)$ is not a constant, but
\textit{runs} by varying its scale $\mu$. Lets observe the running of the strong
coupling constant by integrating the $\beta$-function
\begin{equation}
  \int_{a_s(\mu_1)}^{a_s(\mu_2)}\frac{\dif a_s}{\beta(a_s)} = - \int_{\mu_1}^{\mu_2} \frac{\dif \mu}{\mu} = \log \frac{\mu_1}{\mu_2}.
\end{equation}
To analytically evaluate the above integral we can approximate the
$\beta$-function to first order, with the known coefficient
\begin{equation}
  \label{eq:firstBetaCoefficient}
  \beta_1 = \frac{1}{6}(11 N_c - 2 N_f),
\end{equation}
yielding
\begin{equation}
  a_s(\mu_2) = \frac{a_s(\mu_1)}{\left( 1 - a_s(\mu_1) \beta_1 \log\frac{\mu_1}{\mu_2} \right)}.
\end{equation}
As we have three colours ($N_c=3$) and six flavours ($N_f=6$) the first
$\beta$-function \ref{eq:betaFunction} is positive. Thus for $\mu_2>\mu_1$
$a_s(\mu_2)$ decreases logarithmically and vanishes for $\mu_2 \to \infty$. This
behaviour is known as \textit{asymptotic freedom} and leads to
\textit{confinement}.

Asymptotic freedom states, that for high energies (small distances), the strong
coupling becomes diminishing small and quarks and gluons do not interact. Thus
in isolated baryons and mesons the quarks are separated by small distances, move
freely and do not interact. On the other hand we are not able to separate the
quarks in a meson or baryon. No quark has been detected as single particle yet.
This is qualitatively explained with the gluon field carrying colour charge.
These gluons form so-called \textit{flux-tubes} between quarks, which cause a
constant strong force between particles regardless of their separation.
Consequently the energy needed to separate quarks is proportional to the
distance between them and at some point there is enough energy to favour the
creation of a new quark pair. Thus before separating two quarks we create a
quark\-/antiquark pair. As a result we will probably never be able to observe an
isolated quark. This phenomenon is referred to as colour confinement or simply
confinement.

\subsubsection{Running quark mass}
Not only the coupling but also the masses carry an energy dependencies, which is
governed by the \textit{anomalous mass dimension} $\gamma(a_s)$.

The properties of the running quark mass can be derived similar to the gauge
coupling. Starting from integrating the \textit{anomalous mass dimension}
\cref{eq:anomalousMassDimension}
\begin{equation}
  \log \frac{m(\mu_2)}{m(\mu_1)} = \int_{a_s(\mu_1)}^{a_s(\mu_2)} \dif a_s \frac{\gamma(a_s)}{\beta(a_s)}
\end{equation}
we can approximate the \textit{anomalous mass dimension} to first order and
solve the integral analytically \cite{Schwab2002}
\begin{equation}
  m(\mu_2) = m(\mu_1)\left( \frac{a(\mu_2)}{a(\mu_1)} \right)^{\frac{\gamma_1}{\beta_1}} \left( 1 + \mathcal{O}(\beta_2, \gamma_2) \right).
\end{equation}
As $\beta_1$ and $\gamma_1$ (see \ref{app:gammaCoefficients}) are positive the
quark mass decreases with increasing $\mu$. The general relation between
different scales is given by
\begin{equation}
  m(\mu_2) = m(\mu_1) \exp \left( \int_{a_s(\mu_1)}^{a_s(\mu_2)} \dif a_s \frac{\gamma(a_s)}{\beta(a_s)}  \right)
\end{equation}
and can be solved numerically to run the quark mass to the needed scale $\mu_2$.

\textsc{qcd} in general has a precision problem caused by uncertainties and
largeness of the strong coupling constant $\alpha_s$. The fine-structure
constant (the coupling \textsc{qed}) is known to eleven digits, whereas the
strong coupling is only known to about four. Furthermore for low energies the
strong coupling constant is much larger than the fine-structure constant. E.g.
at the $Z$-mass, the standard mass to compare the strong coupling, we have an
$\alpha_s$ of $0.11$, whereas the fine structure constant would be around
$0.007$. Consequently to use \textsc{pt} we have to calculate our results to
much higher orders, including tens of thousands of Feynman diagrams, in
\textsc{qcd} to achieve a precision equal to \textsc{qed}. For even lower
energies, around \SI{1}{\giga\eV}, the strong coupling reaches a critical value
of $\approx 0.5$ leading to a break down of \textsc{pt}.

In this work we try to achieve a higher precision in the value of $\alpha_s$.
The framework we use to measure the strong coupling constant are the
\textsc{qcdsr}. A central object needed to describe hadronic states with the
help of \textsc{qcd} is the \textit{two-point function} for which we will devote
the following section.

\section{Two-Point function}
\label{sec:twoPointFunction}
A lot of particle physics is dedicated of calculating the \textit{S\-/matrix},
which contains all the information about how initial states evolve in time. One
important tool for obtaining the S\-/matrix is the \textit{LSZ
  (Lehmann\-/Symanzik\-/Zimmermann)-reduction formula}
\cite{Lehmann1954a,Schwartz2013}
\begin{equation}
  \begin{split}
    \langle f\vert S \vert i\rangle = \left[ i\int_0^\infty \fdif{x_1}
      e^{-ip_1x_1} (\Box^2 + m^2)\right] \cdots
    \left[ i\int_0^\infty \fdif{x_n} e^{ip_nx_n} (\Box^2 + m^2)\right] \\
    \times \langle\Omega\vert T\{\phi(x_1)\cdots\phi(x_n)\} \vert\Omega\rangle,
  \end{split}
\end{equation}
with the $-i$ in the exponent applying for initial states and the $+i$ for final
states. The LSZ\-/reduction formula relates the S\-/matrix to the
\textit{correlator} (also referred to as \textit{n\-/point function})
\begin{equation}
  \langle\Omega\vert T\{\phi(x_1)\phi(x_2)\cdots\phi(x_n)\} \vert\Omega\rangle,
\end{equation}
where $T\{\cdots\}$ is the time\-/ordered product and $\vert\Omega\rangle$ is
the ground state/ vacuum of the interacting theory. Note that the fields are in
general given in the Heisenberg picture, which implies translational invariance.
\begin{equation}
  \begin{split}
    \langle\Omega\vert \phi(x)\phi(y) \vert\Omega\rangle &= \langle\Omega\vert \phi(x) e^{i\hat P y}e^{-i\hat P y}\phi(y)e^{i\hat P y}e^{-i\hat P y} \vert\Omega\rangle \\
    &= \langle\Omega\vert \phi(x-y)\phi(0) \vert\Omega\rangle,
  \end{split}
\end{equation}
where we made use of the translation operator $\hat T(x) = e^{-i \hat P x}$.

In this work we are solely concerned about the \textit{two\-/point function},
especially in the vacuum expectation value of the Fourier transform of two time\-/ordered \textsc{qcd}
quark Noether\-/currents
\begin{tcolorbox}[ams equation,myformula]
  \label{eq:qcdCorrelator}
  \Pi_{\mu\nu}(q^2) \equiv \int \fdif{q} e^{iqx} \langle\Omega\vert J_\mu(x)J_\nu(0) \vert\Omega\rangle,
\end{tcolorbox}
where the Noether current is given by
\begin{equation}
  \label{eq:qcdCurrent}
  J_\mu(x) = \anti{q}(x) \Gamma q(x).
\end{equation}
Here, $\Gamma$ can be any of the following dirac matrices $\Gamma \in \{ 1,
i\gamma_5, \gamma_\mu, \gamma_\mu\gamma_5\}$, specifying the quantum number of
the current (S: \textit{scalar}, P: \textit{pseudo-Scalar}, V:
\textit{vectorial}, A: \textit{axial-vectorial}, respectively). By choosing the
right quantum numbers we can theoretically represent the processes we want to
study, which will be important when we want to match the hadrons produced in $\tau$\-/decays.

\begin{wrapfigure}{r}{4cm}
  \centering
  \begin{tkizpicture}
    \feynmandiagram [small, vertical=a to b] { i1 [particle=\(\tau\)] -- [anti fermion]
      a -- [anti fermion] i2 [particle=\(\nu_\tau\)], a -- [scalar] b, b -- [fermion, half
      left, edge label=\(\anti{q}\)] c -- [fermion, half left, edge label=\(q\)]
      b, c -- [scalar] d, f1 [particle=\(\tau\)]-- [fermion] d -- [fermion] f2
      [particle=\(\nu_\tau\)], };
  \end{tkizpicture}
  \caption{\(\tau\anti{\tau}\)\-/annihilation with a quark\-/antiquark pair.}
  \label{fig:tauAntiTauAnnihilation}
\end{wrapfigure}
From a Feynman diagram point of view we can illustrate the two\-/point function
as quark\-/antiquark pair, which is produced by an external source, e.g. the
virtual \(W\)\-/boson of \(\tau\anti{\tau}\)\-/annihilation as seen in
\cref{fig:tauAntiTauAnnihilation}. Here the quarks are propagating at
\textit{short\-/distances}, which implies that we can make use of \textsc{pt},
thus avoiding \textit{long\-/distance} (\textsc{npt}\-/) effects, that would
appear if the initial and final states where given by hadrons
\cite{Colangelo2000}.

\subsection{Short\-/Distances vs. Long\-/Distances}
If we want to calculate the two\-/point function in \textsc{qcd} we have to
differentiate short\-/ and long\-/distances or large or small momenta. In
general when we talk about small distances we refer to large momenta. Large
momenta implies a small strong coupling constant. Consequently we can use
\textsc{pt} for short\-/distances. On the contrary long distances involve small
momenta, which implies a large coupling constant. Thus for long distances the
perturbation theory has broken down and cannot be used. We can make use of
\(\Lambda\)\-/regularisation (\cref{itm:lambdaRegularisation}) to define the
differentiation of short\-/ and long\-/distances. Thus for \(q <
\Lambda\)\footnote{$\Lambda$ is a momentum\-/cutoff, so we have to compare
  momenta and not distances.} we have short\-/distances and for \(q > \Lambda\)
we have long\-/distances. In our case we need the quark\-/antiquark pair of
\cref{fig:electronElectronScattering} to be highly virtual \footnote{Which is
  the same of saying, that the quark\-/antiquark pair needs a high external
  momentum \(q\).}. To separate long\-/distances from short\-/distances using a
length scale we can say that the length scale should be smaller than the radius
of a hadron.

\subsection{Relating Two\-/Point Function and Hadrons}
The two\-/point function can be interpreted physically as the amplitude of
propagating single- or multi\-/particle states and their excitations. The
possible states, in our case hadrons we describe through the correlator is fixed
by the quantum numbers of the current we set for the vacuum expectation value.
For example the neutral $\rho$\-/meson has a quark content of \((u\anti{u} -
d\anti{d})/\sqrt{2}\) and is a spin\-/1 vector meson. Consequently by choosing a
current
\begin{equation}
  J_\mu(x) = \frac{1}{2} ( \anti{u}(x)\gamma_\mu u(x) - \anti{d}(x)\gamma_\mu d(x) )
\end{equation}
the two\-/point function contains the same quantum numbers as the
\(\rho\)\-/meson and is said to materialise to it. A list of some ground\-/state
mesons for combinations of the light\-/quarks \(u, d\) and \(s\) is given in
\cref{table:groundStateMesons}.
\begin{table}
  \centering
  \begin{tabular*}{\textwidth}{lccc @{\extracolsep{\fill}}c}
    \toprule
    Symbol & Quark content & Isospin & \(J\) & Current \\
    \midrule
    \(\pi^0\)  & \(\frac{u\anti{u} - d\anti{d}}{2}\) & \(1\) & \(0\)
                                             & \(\anti{u}\gamma_\mu\gamma_5 u + \anti{d}\gamma_\mu\gamma_5 d\) \\
    \(\eta\)   & \(\frac{u\anti{u} + d\anti{d} - 2s\anti{s}}{\sqrt{6}}\) & \(0\)
                                     & \(0\) & \(\anti{u}\gamma_\mu\gamma_5 u + \anti{d}\gamma_\mu\gamma_5d
                                               - 2\anti{s}\gamma_\mu\gamma_5 s\) \\
    \(\eta\prime\) & \(\frac{u\anti{u} + d\anti{d} + s\anti{s}}{\sqrt{3}}\)
                           & \(0\) & \(0\) & \(\anti{u}\gamma_\mu\gamma_5 u +
                                             \anti{d}\gamma_\mu\gamma_5 d + \anti{s}\gamma_\mu\gamma_5 s\) \\
    \(\rho^0\) & \(\frac{u\anti{u} - d\anti{d}}{\sqrt{2}}\) & \(1\) & \(1\)
                                             & \(\anti{u}\gamma_\mu u - \anti{d}\gamma_\mu d\) \\
    \(\omega\) & \(\frac{u\anti{u}+d\anti{d}}{\sqrt{2}}\) & \(0\) & \(1\)
                                             & \(\anti{u}\gamma_\mu u + \anti{d}\gamma_\mu d\) \\
    \(\phi\) & \(s \anti{s}\) & \(0\) & \(1\)
                                             & \(\anti{s}\gamma_\mu\gamma_5 s\)\\                                        
    \bottomrule
  \end{tabular*}
  \caption{Ground\-/state vector and pseudoscalar mesons for the light\-/quarks
    \(u, d\) and \(s\) with their corresponding currents in the two\-/point
    function. Note that we use \(\gamma_\mu\) for vector and
    \(\gamma_\mu\gamma_5\) for the pseudoscalar mesons.}
  \label{table:groundStateMesons}
\end{table}

The correlator is materialising into a spectrum of hadrons. Thus if we insert a
complete set of states of hadrons we can make use of the unitary relation
\begin{equation}
  \langle\Omega\vert J_\mu(x)\J_\nu(0) \vert\Omega\rangle = \sum_X \langle\Omega\vert J_\mu(x) \vert X\rangle \langle X\vert J_\nu(0) \vert\Omega\rangle.
\end{equation}
to represent the two\-/point correlator via a spectral function \(\rho(t)\)
\begin{tcolorbox}[ams equation,myformula]
  \label{eq:KallenLehmannSpectralDecomposition}
  \Pi(q^2) = \int_0^\infty \dif s \frac{\rho(s)}{s-p^2-i\epsilon}.
\end{tcolorbox}
The above relation is referred to as \textit{Källén-Lehmann spectral
  representation} and has been discovered early on by
\cite{Kallen1952,Lehmann1954}. It relates the two\-/point function to the
spectral function $\rho$, which can be represented as sum over all possible
hadronic states
\begin{equation}
  \rho(s) = (2\pi)^3 \SumInt_X \abs{\langle\Omega\vert J_{\mu}(0) \vert X\rangle}^2 \delta^4(s-p_X).
\end{equation}
Note that the analytic properties of the two\-/point are in one\-/to\-/one
correspondence with the newly introduced spectral function and thus determined
by the possible hadrons states, which only form on positive real axis. A full
derivation of the \textit{Källén\-/Lehmann spectral representation} can be found
in the appendix The spectral function is interesting to us for two reasons.
First it is experimentally measurable and second it carries a problematic
``branch cut'', which we want to discuss now.

\subsection{Analytic Structure of the Two\-/Point Function}
The general two\-/point function $\rho(s)$ has some interesting, but problematic
analytic properties. It has poles for single\-/particle states and a continuous
branch cut for multi\-/particle states. The single and multi\-/particle states,
for a general correlator, can be mathematically separated by
\begin{equation}
  \rho(s) = Z \delta(s-m^2) + \theta(s-s_0)\sigma(s)
  + \sigma(s),
\end{equation}
where the second term is the contribution from multi\-/particle states.
\(\sigma(s)\) is zero till we reach the threshold, where we have sufficient
energy to form multi\-/particle states. The analytic structure is depicted by
\cref{fig:analyticStructureCorrelator} and we can see that the spectral function
has \(\delta\)\-/spikes for single\-/particle states and a continuous
contribution for \(s\geq 4m\) resulting from multi\-/particle states. These lead
to poles and a continuous branch cut of the two\-/point function.
\begin{figure}
  \centering
  \includegraphics[width=\textwidth]{./images/analyticStructureCorrelator.eps}
  \caption{Analytic structure in the complex $q^2$-plane of the Fourier
    transform of the two-point function. The hadronic final states are
    responsible for poles appearing on the real-axis. The one-particle states
    contribute as isolated pole and the multi-particle states contribute as
    bound-states poles or a continues ``discontinuity cut''
    \cite{Peskin1995,Zwicky2016}.}
  \label{fig:analyticStructureCorrelator}
\end{figure}


\textcolor{red}{correct sum up} is referred to as \textit{dispersion relation}
analogous to similar relations which arise for example in electrodynamics and
defines the \textit{spectral function}
\begin{equation}
  \label{eq:spectralFunction}
  \rho(s) = \frac{1}{\pi} \Ima \Pi(s).
\end{equation}


\subsection{Lorentz Decomposition}
Apart the spectral decomposition we can Lorentz decompose the correlator to a
scalar function $\Pi(q^2)$. There are only two possible terms that can reproduce
the second order tensor $q_\mu q_\nu$ and $q^2 g_{\mu\nu}$. The sum of both
multiplied with two arbitrary functions $A(q^2)$ and $B(q^2)$ yields
\begin{equation}
  \Pi_{\mu\nu}(q^2) = q_\mu q_\nu A(q^2) + q^2 g_{\mu\nu} B(q^2).
\end{equation}
By making use of the \textit{Ward\-/identity}
\begin{equation}
  \label{eq:wardIdentity}
  \begin{split}
    q^\mu \Pi_{\mu\nu} &= \int \dif x q^\mu e^{iqx} \langle 0 \vert  J_\mu(x) J_\nu(0) \vert 0 \rangle \\
    &= -i \int \dif x i q^\mu e^{i q^\nu x_\nu} \langle 0 \vert J_\mu(x) J_\nu(0) \vert 0 \rangle \\
    &= i \int \dif x e^{iqx} \langle 0 \vert \partial_\mu[J_\mu(x)] J_\nu(0)
    \vert 0 \rangle \\
    &= 0, \quad \text{with} \quad \partial_\mu J_\mu(x) = 0,
  \end{split}
\end{equation}
where we used $i q^\mu e^{i q^\nu x_\nu} = \partial_\mu e^{i q^\nu x_\nu}$ in
the second and integration by parts in the third line. The Ward identity is
dependent on the conserved Noether\-/current $J_\mu$ and thus only holds for
same flavour quarks. With the Ward\-/identity we are able to demonstrate, that
the two arbitrary functions are related
\begin{equation}
  \begin{split}
    q^\mu q^\nu \Pi_{\mu\nu} &= q^4 A(q^2) + q^4 B(q^2) = 0 \\
    &\quad \implies A(q^2) = -B(q^2).
  \end{split}
\end{equation}
Thus redefining $A(q^2) \equiv \Pi(q^2)$ we expressed the correlator as a scalar
function
\begin{equation}
  \Pi_{\mu\nu}(q^2) = (q_\mu q_\nu - q^2 g_{\mu\nu})\Pi(q^2).
\end{equation}





These discontinuities can be tackled with \textit{Cauchy's theorem}, which we
will apply in \cref{sec:sumRules}.

Having dealt exclusively with the perturbative part of the theory, we have to
discuss non\-/perturbative contributions, as \textsc{qcd} is known to have
non\-/negligible contributions. Thus before continuing with the \textit{Sum
  Rules} we need a final ingredient the \define{ope}{Operator Product
  Expansion}, which treats the non\-/perturbative contributions of our theory.




\section{Operator Product Expansion}
The \textsc{ope} was introduced by Wilson in 1969 \cite{Wilson1969} as an
alternative to the in this time commonly used current\-/algebra. The expansion
states that non\-/local operators can be rewritten into a sum of composite local
operators and their corresponding coefficients:
\begin{tcolorbox}[ams equation,myformula]
  \label{eq:ope}
  \lim_{x\to y} A(x) B(y) = \sum_n C_n(x-y)\mathcal{O}_n(x),
\end{tcolorbox}
where $C_n(x-y)$ are the so-called \textit{Wilson-coefficients} and \(A, B\) and
\(\mathcal{O}_n\) are operators.

The OPE lets us separate short-distance from long-distance effects. In
\textsc{pt} we can only amount for short-distances, which are equal to high
energies, where the strong-coupling $\alpha_s$ is small. Consequently the
\textsc{ope} decodes the long-distance effects in the higher dimensional
operators.

The form of the composite operators are dictated by gauge- and Lorentz symmetry.
Thus we can only make use of operators of even dimension. The scalar operators
up to dimension six are given by \cite{Pascual1984}
\begin{equation}
  \begin{array}{ll}
    \text{Dimension 0:} & \mathbb{1} \\
    \text{Dimension 4:} & :m_i \anti{q} q: \\
                        & :G_a^{\mu\nu}(x) G_{\mu\nu}^a(x): \\
    \text{Dimension 6:} & :\anti{q} \Gamma q \anti{q} \Gamma q: \\
                        & :\anti{q} \Gamma \frac{\lambda^a}{2} q_\beta(x) \anti{q} \Gamma \frac{\lambda^a}{2} q: \\
                        & :m_i \anti{q} \frac{\lambda^a}{2} \sigma_{\mu\nu} q G^{\mu\nu}_a: \\
                        & :f_{abc} G_a^{\mu\nu} G_b^{\nu\delta} G_c^{\delta\mu}:,
  \end{array}
\end{equation}
where $\Gamma$ stands for one of possible dirac matrices (as seen
\cref{eq:qcdCurrent}). The operator of dimension zero is the identity and its
Wilson\-/coefficient is solely \textsc{pt}. The higher dimension operators
appear as normal ordered products of fields and vanish by definition in
\textsc{pt}. On the contrary, in \textsc{npt qcd} they appear as
\textit{condensates}. Condensates are remainders of the \textsc{qcd} vacuum,
which contribute to all strong processes. For example the condensates of
dimension four are the quark-condensate $\langle \anti{q} q \rangle$ and the
gluon-condensate $\langle a GG \rangle$.

As we work with dimensionless functions (e.g. the correlator $\Pi$), the r.h.s.
of \cref{eq:ope} has to be dimensionless. As a result the Wilson-coefficients
have to cancel the dimension of the operator with their inverse mass dimension.
To account for the dimensions we can make the inverse momenta explicit
\begin{equation}
  \Pi_{V/A}^{OPE}(s) = \sum_{D=0,2,4\dots} \frac{c^{(D)} \langle \mathcal{O}^{(D)}(x) \rangle}{(-q^2)^{D/2}},
\end{equation}
where we used $C^{(D)}=c/(-s)^{D/2}$ with $D$ being the dimension. Thus the
\textsc{ope} should converge with increasing dimension for sufficiently large
momenta $s$.

\subsection{A practical example}
Let's show how the \textsc{ope} contributions are calculated \cite{Shifman1978,
  Pascual1984}. We will compute the perturbative and quark-condensate
Wilson-coefficients for the rho meson. To do that we have to evaluate Feynman
diagrams using standard \textsc{pt}.

The rho meson is a vector meson of isospin one composed of \(u\) and \(d\)
quarks. As a result (see. \cref{table:groundStateMesons}) we can match its
quantum numbers with the current
\begin{equation}
  J^\mu(x) = \frac{1}{2}\left(:[\anti{u} \gamma^\mu u](x) - [\anti{d} \gamma^\mu d](x):\right).
\end{equation}
\begin{figure}
  \centering
  \includegraphics[width=\textwidth]{./images/condensateFeynmanDiagram.eps}
  \caption{Feynman diagrams of the perturbative (a) and the quark-condensate (b)
    contribution. The upper part of the right diagram is not wick-contracted and
    responsible for the condensate.}
  \label{fig:OPEFeynmanDiagram}
\end{figure}
Pictorial the dimension zero contribution is given by the quark\-/antiquark loop
Feynman diagram \cref{fig:OPEFeynmanDiagram}. The higher dimension contributions
are given by the same Feynman diagram, but with non contracted fields. These non
contracted fields are yielding the condensates. Thus not contracting the
quark\-/antiquark field (see. \cref{fig:OPEFeynmanDiagram} b) will give us
access to the Wilson coefficient of the dimension four quark\-/condensate
\(\langle \anti{q}q \rangle\).

The perturbative part (the Wilson coefficient of dimension zero) can than be
taken from the mathematical expression for the scalar correlator
\begin{equation}
  \label{eq:rhoScalarCorrelator}
  \begin{split}
    \Pi(q^2) &= - \frac{i}{4 q^2 (D -1)} \int \dif^D x e^{iqx} \langle \Omega \vert T \{:\anti{u}(x) \gamma^\mu u(x) - \anti{d}(x) \gamma^\mu d(x): \\
    &\quad\times :\anti{u}(0 \gamma_\mu u(0) - \anti{d}(0)\gamma_\mu d(0): \}
    \rangle.
  \end{split}
\end{equation}
To extract the dimension zero Wilson coefficient we apply Wick's theorem to
contract all of the fields, which represents the lowest order of the
perturbative contribution. The calculation is only using standard \textsc{pt}
and we will restrict ourselves in displaying the result and omitting the
calculation\footnote{The interested reader can follow \cite{Pascual1984} for a
  detailed calculation.}.
\begin{equation}
  \begin{split}
    \Pi(q^2) &= \frac{i}{4q^2(D-1)} (\gamma^\mu)_{ij} (\gamma_\mu)_{kl} \int \dif^D x e^{iqx} \\
    &\times\quad\left[ \wick{ \c u_{j \alpha}(x) \anti{\c u}_{k \beta}(0)} \cdot
      \wick{\c u_{l \beta}(0) \anti{\c u}_{i \alpha}(x)} + (u \to d)
    \right] \\
    &= \frac{3}{8\pi^2} \left[ \frac{5}{3} - \log\left( -\frac{q^2}{\nu^2}
      \right) \right].
  \end{split}
\end{equation}

To calculate the higher dimensional contributions of the OPE we use the same
techniques as before, but leave some of the fields uncontracted. Thus instead of
applying Wick's theorem for all possible contractions fields, we leave some
fields uncontracted. For leaving the quark field uncontracted in
\cref{eq:rhoScalarCorrelator} we get
\begin{equation}
  \begin{split}
    \Pi(q^2) &= \frac{i}{4q^2(D-1)} (\gamma^\mu)_{ij} (\gamma_\mu)_{kl} \int \dif^D x e^{iqx} \left[ \phantom{\wick{ \c u_{j \alpha}(x) \anti{\c u}_{k \beta}(0)}} \right.\\
    &\quad+\,\wick{ \c u_{j \alpha}(x) \anti{\c u}_{k \beta}(0)} \cdot \langle \Omega \vert : \anti{u}_{i \alpha}(x) u_{l \beta} (0) :\vert \Omega \rangle \\
    &\quad\,\left.+\wick{\c u_{l \beta}(0) \anti{\c u}_{i \alpha}(x)} \cdot
      \langle \Omega \vert: \anti{u}_{k \beta}(0) u_{j \alpha}(x) : \vert \Omega
      \rangle + (u \to d) \right].
  \end{split}
\end{equation}
Here we can observe the condensates as non\-/vanishing vacuum values of normal
ordered product of fields:
\begin{equation}
  \langle\Omega_{QCD}\vert \anti{q}(x) q(0) \vert\Omega_{QCD}\rangle \neq 0.
\end{equation}
We emphasised the \textsc{qcd} vacuum \(\Omega_{QCD}\), which is responsible for
vacuum expectation values different than zero. E.g. for a vacuum of \textsc{qed}
this contributions would vanish by definition. Pictorial the condensates take
form of unconnected propagators, sometimes marked with an \(x\), as seen in
\cref{fig:OPEFeynmanDiagram}.

To make the non\-/contracted fields local, we can expanded them in $x$
\begin{equation}
  \begin{split}
    \langle \Omega \vert: \anti{q}(x) q(0):\vert \Omega \rangle &= \langle\Omega\vert: \anti{q}(0) q(0): \vert\Omega\rangle\\
    &\quad+ \langle\Omega\vert:\left[ \partial_\mu \anti{q}(0) \right]
    q(0):\vert\Omega\rangle x^\mu + \dots.
  \end{split}
\end{equation}
and introduce a standard notation for the localised condensate
\begin{equation}
  \langle \anti{q}q \rangle \equiv \langle\Omega\vert: \anti{q}(0) q(0):\vert\Omega\rangle.
\end{equation}
Finally, the contribution to the rho scalar correlator is then given by the
following expression
\begin{equation}
  \Pi_{(\rho)}(q^2) = \frac{1}{2} \frac{1}{\left(-q^2\right)^2} \left[ m_u\langle \anti{u} u \rangle + m_d \langle \anti{d} d \rangle\right].
\end{equation}
Here we can clearly see that for dimension four we get a factor of
\(1/(-q^2)^2\), which is responsible for the suppression of the series. The
condensates \(\langle\anti{u}u\rangle\) and \(\langle\anti{d}d\rangle\) are
numbers, that have to be derived by phenomenological fits or \textsc{lqcd}.
Fortunately once found, the value of the condensate can be used for any process.

In summary we note that the usage of the \textsc{ope} and its validity is far
from obvious. We are deriving the \textsc{ope} from matching the
Wilson-coefficients to Feynman-graph analyses. These Feynman-graphs are
calculated perturbatively but the coefficients with dimension $D>0$ correspond
to \textsc{npt} condensates! The condensates by themselves have to be gathered
from external, \textsc{npt} methods.

Now that we have a tool to deal with the problematic \textsc{qcd} vacuum and
\textsc{npt}\-/effects we are left with two problems. First we still do not know
how to deal with hadronic states in the quark\-/gluon picture. This will be
tackled by Duality. Secondly we have seen that we can access the two\-/point
function theoretically on the physical sheet except for the positive real axis, due to its
analytic properties, but that the experimental measurable spectral function is
solely be defined on the positive real axis. Thus we need to make use of
Cauchy's Theorem. In total we will bring together the two\-/point function, the
\textsc{ope}, Duality and Cauchy's theorem to formulate the \textsc{qcdsr}.

\section{Sum Rules}
\label{sec:sumRules}
To relate the measurable hadronic final states of a \textsc{qcd} process (e.g.
$\tau$-decays into hadrons) to a theoretical calculable \textsc{qcdsr} have been
employed by Shifman in the late seventies \cite{Shifman1978}.

The sum rules are a combination of the two-point function, its analyticity, the
\textsc{ope}, a dispersion relation, the optical theorem and quark hadron
duality.

The previously introduced two-point function \cref{eq:twoPointFunction} is
generally described by the \textsc{ope} to account for \textsc{npt} effects.
\begin{equation}
  \Pi(q^2) = \Pi^{OPE}(q^2).
\end{equation}
Furthermore it is related to the theoretical spectral function $\rho(s)$ via a
dispersion relation (\cref{eq:dispersionRelation}). Using \textsc{qcd} we are
computing interactions based on quarks and gluons, but as we have seen
(confinement), we are only able to observe hadrons. Consequently to connect the
theory to the experiment we have to assume \textbf{quark-hadron
  duality}\footnote{Or simply duality.}, which implies that physical quantities
can be described equally good in the hadronic or in the quark-gluon picture.
Thus we rewrite the dispersion relation \cref{eq:dispersionRelation} as
\begin{equation}
  \Pi^{OPE}_{th}(q^2) = \int_0^\infty \frac{\rho_{exp}(q^2)}{(s-q^2-i\epsilon)},
\end{equation}
where we connected the theoretical correlator $\Pi_{th}$ with the experimental
measurable spectral function $\rho_{exp}$.

We have seen, that the theoretical description of the correlator $\Pi_{th}$
contains poles on the real axis. Unfortunately the experimental data
$\rho_{exp}$ is solely accessible on the positive real axis. As a result we have
to make use of Cauchy's theorem to access the theoretical values of the
two-point function close to the positive real axis (see
\cref{fig:correlatorComplexContour}), which is given by
\begin{equation}
  \label{eq:cauchysTheorem}
  \int_{\mathcal{C}} f(z) \dif z = 0,
\end{equation}
where $f(z)$ is an analytic function on a closed contour $\mathcal{C}$.
\begin{figure}[h]
  \centering
  \label{fig:correlatorComplexContour}
  \includegraphics[width=0.6\textwidth]{./images/correlatorComplexContour.eps}
  \caption{Analytical structure of $\Pi(s)$ with the used contour $\mathcal{C}$
    for the final QCD Sum Rule expression \cref{eq:qcdSumRules}.}
\end{figure}

The final ingredient of the \textsc{qcd} sum rules is the \textit{optical
  theorem}, relating experimental data with the imaginary part of the correlator
(the spectral function $\rho(s)$).

In total, with the help Cauchy's theorem, the \textsc{qcd} sum rules can be
summed up in the following expression
\begin{equation}
  \label{eq:qcdSumRules}
  \frac{1}{\pi}\int_0^\infty \frac{\rho_{exp}(t)}{t - s}\dif t = \frac{1}{\pi} \oint_{\mathcal{C}} \frac{\Ima \Pi_{OPE}(t)}{t -s}\dif t,
\end{equation}
where the l.h.s. is given by the experiment and the r.h.s. can be theoretically
evaluated by applying the \textsc{ope} of the correlator $\Pi_{OPE}(s)$.
\end{document}
% LocalWords:  qcdFeynmanDiagrams qcdFeynmanRules dispersionRelation sav itm
% LocalWords:  qcdCurrent twoPointFunction anomalousMassDimension sumRules lccc
% LocalWords:  OPEFeynmanDiagram correlatorComplexContour qcdSumRules
% LocalWords:  LocalWords simpleTwoPointFunction twoPointFunctionSelfEnergy
% LocalWords:  cuttingRules qcdLagrangian electronElectronScattering
% LocalWords:  lambdaRegularisation groundStateMesons
