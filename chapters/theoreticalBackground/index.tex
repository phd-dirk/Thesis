\documentclass[../../index.tex]{subfiles}

\begin{document}
\chapter{Theoretical Background}

\section{Quantum\-/chromodynamics}
\label{sec:quantumchromodynamics}
Since the formulation of \textsc{qed} in the end of the 40's it has been
attempted to describe the strong nuclear force as a \textsc{qft}, which has been
achieved in the 70's as \textsc{qcd}
\cite{GellMann1972,Fritzsch1973,Gross1973,Politzer1973,Weinberg1973}.
\textsc{qcd} is a renormalisable \textsc{qft} of the strong interaction, which
fundamental fields are given by dirac spinors of spin\-/$1/2$, the so\-/called
quarks, with a fractional electric charge of $\pm 1/3$ or $\pm 2/3$. The theory
furthermore contains gauge\-/fields of spin 1, which are chargeless, massless
and referred to as gluons. The gluons are the force\-/mediators, which interact
with quarks and themselves, in contrast to photons of \textsc{qed}, which
interact only with fermions (see \cref{fig:qcdFeynmanDiagrams}).

The corresponding gauge\-/group of \textsc{qcd} is the non\-/abelian group
$SU(3)$. Each of the quark flavours $u,d,c,s,t$ and $b$ belongs to the
fundamental representation of $SU(3)$ and contains a triplet of fields $\Psi$.
\begin{equation}
  \Psi = \begin{pmatrix} \Psi_1 \\ \Psi_2 \\ \Psi_3 \end{pmatrix}
  = \begin{pmatrix} \textcolor{red}{red} \\ \textcolor{green}{green} \\ \textcolor{blue}{blue} \end{pmatrix}
\end{equation}
The components of the triplet are the so\-/called colours\footnote{The colour
  denomination is not gauge\-/invariant. After a colour gauge transformation the
  new colours are a linear combination of the old colours, which breaks
  gauge\-/symmetry.} red, green and blue, which is referred to as
\textit{colour\-/charge}. The gluons belong to the adjoint representation of
$SU(3)$, contain an octet of fields and can be expressed using the Gell-Mann
matrices $\lambda_a$
\begin{equation}
  B_\mu = B_\mu^a \lambda_a \qquad a = 1,2,\dotsc 8
\end{equation}
\begin{figure}
  \centering \includegraphics[width=\textwidth]{./images/qcdFeynmanDiagrams.eps}
  \caption{Feynman diagrams of the strong interactions with corresponding
    electromagnetic diagrams. We see that the gluons carry colour charge and thus
    couple to other gluons, which is not the case for the photons.}
  \label{fig:qcdFeynmanDiagrams}
\end{figure}
\begin{table}
  \centering
  \begin{minipage}[c]{0.4\textwidth}
    \begin{tabular}{ll}
      \toprule
      Flavour & Mass\\
      \midrule
      $u$ & \SI{3.48(24)}{\mega\eV} \\
      $d$ & \SI{6.80(29)}{\mega\eV} \\
      $s$ & \SI{130.0(18)}{\mega\eV} \\
      $c$ & \SI{1.523(18)}{\giga\eV} \\
      $b$ & \SI{6.936(57)}{\giga\eV} \\
      $t$ & \SI{173.0(40)}{\giga\eV} \\
      \bottomrule 
    \end{tabular}
  \end{minipage}\hfill
  \begin{minipage}[c]{0.59\textwidth}
    \caption{List of Quarks and their masses. The masses of the up, down,
      strange, charm and bottom quark are the renormalisation group invariant
      (RGI) quark masses and are quoted in the four-flavour theory ($N_f=2+1$)
      at the scale $\mu=\SI{2}{\giga\eV}$ in the $\overline{MS}$ scheme and are
      taken from the \textit{Flavour Lattice Averaging Group} \cite{FLAG2019}.
      The mass of the top quark is not discussed in \cite{FLAG2019} and has been
      taken from \cite{PDG2018} from direct observations of top events.}
  \end{minipage}
  \label{table:quarkList}
\end{table}
The classical \textit{Lagrange density} of \textsc{qcd} is given by
\cite{Jamin2006,pascual1984}:
\begin{equation}
  \label{eq:qcdLagrangian}
  \mathcal{L}_{QCD}(x) = -\frac{1}{4}G_{\mu\nu}^a(x)G^{\mu\nu a}(x) + \sum_A \left[ \frac{i}{2} \bar{q}^A(x) \gamma^\mu \overleftrightarrow{D}_\mu q^A(x) - m_A\bar{q}^A(x) a^A(x) \right],
\end{equation}
with $q^A(x)$ representing the quark fields and $G_{\mu\nu}^a$ being the
\textit{gluon field strength tensor} given by:
\begin{equation}
  \label{eq:gluonField}
  G_{\mu\nu}^a(x) \equiv \partial_\mu B_\nu^a(x) - \partial_\nu^a(x) + g f^{abc} B_\mu^b(x) B_\nu^c(x),
\end{equation}
with $f^{abc}$ as \textit{structure constants} of the gauge\-/group $SU(3)$ and
$overleftrightarrow{D}_\mu$ as covariant derivative acting to the left and to
the right.
Furthermore we have used $A, B, \dotsc = 0, \dotsc 5$ as flavour indices, $a, b,
\dotsc = 0, \dotsc, 8 $ as colour indices and $\mu, \nu, \dotsc = 0, \dotsc 3$
as lorentz indices. Explicitly the Lagrangian writes:
\begin{equation}
  \begin{split}
    \mathcal{L}_0(x) &=- \frac{1}{4} \left[ \partial_\mu G_\nu^a(x) - \partial_v G_\mu^a(x) \right] \left[ \partial^\mu G_a^\nu(x) - \partial^\nu G^\mu_a(x) \right] \\
    &\quad+ \frac{i}{2}\anti{q}_\alpha^A(x) \gamma^\mu \partial_\mu q_\alpha^A(x) - \frac{i}{2} \left[ \partial_\mu \anti{q}_\alpha^A(x) \right] \gamma^\mu q_\alpha^A(x) - m_A \anti{q}_\alpha^A(x) q_\alpha^A(x) \\
    &\quad+ \frac{g_s}{2} \anti{q}_\alpha^A(x) \lambda^a_{\alpha \beta}\gamma_\mu q_\beta^A(x) G^\mu_a(x) \\
    &\quad- \frac{g_s}{2}f_{abc}\left[ \partial_\mu G_\nu^a(x) - \partial_\nu G_\mu^a(x)\right] G_b^\mu(x) G_c^\nu(x) \\
    &\quad- \frac{g_s^2}{4} f_{abc}f_{ade} G_\mu^b(x) G_\nu^c(x) G_d^\mu(x)
    G_e^\nu(x)
  \end{split}
\end{equation}
The first term is the kinetic term for the massless gluons. The next three terms
are the kinetic terms for the quark field with different masses for each
flavour. The rest of the terms are the interaction terms. The fifth term
represents the interaction between quarks and gluons  and the last two terms the
self\-/interactions of gluon fields. The equations of motion resulting from the
above Lagrangian can then be given, in compact form, by
\begin{equation}
  \begin{split}
    (i \gamma_\mu D^\mu - m_A) q^A(x) &= 0 \\
    \left[ D^\mu, G_{\mu\nu}(x) \right] &= -i g^2 T_a \sum_A \anti{q}^A(x) T_a \gamma_\nu q^A(x).
  \end{split}
\end{equation}

\subsection{Renormalisation Group}
The perurbations of the QCD Lagrangian \ref{eq:qcdLagrangian} lead to
divergencies, which have to be \textit{renormalized}. There are different
aproaches to 'make' these divergencies finite. The most popular one is
\textbf{dimensional regularisation}.

In \textit{dimensional regularisation} we expand the four space-time dimensions
to arbitrary dimensions. Consequently the in QCD calculations appearing
$\textit{Feyman integrals}$ have to be continued to $D$-dimensions like
\begin{equation}
  \label{eq:dimRegFeynmanIntegral}
  \mu^{2\epsilon} \int \frac{\dif^D p}{(2\pi)^D}\frac{1}{[p^2-m^2+i0][(q-p)^2=m^2+i0]},
\end{equation}
where we introduced the scale parameter $\mu$ to account for the extra
dimensions and conserve the mass dimension of the non continued integral.

In addition \textit{physical quantities}\footnote{Observables that can be
  measured.} cannot depend on the renormalisation scale $\mu$. Thus the
derivative by $\mu$ of a general \textit{physical quantity} $R(q, a_s, m)$ that
depends on the external momentum q, the renormalised coupling
$a_s\equiv\alpha_s/\pi$ and the renormalized quark mass $m$ has to yield zero
\begin{equation}
  \label{eq:RGE}
  \mu \od{}{\mu}R(q, a_s, m) = \left[ \mu \pd{}{\mu} + \mu \od{a_s}{\mu} \pd{}{m} + \mu \od{m}{\mu} \pd{}{m} \right] R(q, a_s, m) = 0.
\end{equation}
\cref{eq:RGE} is referred to as \textbf{renormalization group equation} and is
the basis for defining the \textit{renormalisation group functions}:
\begin{align}
  \beta(a_s) &\equiv -\mu \od{a_s}{\mu} = \beta_1 a_s^2 + \beta_2 a_s^3 + \dots & \beta-\text{function}
                                                                                  \label{eq:betaFunction} \\
  \gamma(a_s) &\equiv - \frac{\mu}{m} \od{m}{\mu} = \gamma_1 a_s + \gamma_2 a_s^2 + \dots & \text{anomalous mass dimension}.
                                                                                            \label{eq:anomalousMassDimension}
\end{align}

\subsubsection{Running gauge coupling}
The $\beta$-function and the anomalous mass dimension are responsible for the
running of the strong coupling and the running of the quark mass respektively.
In this section we will shortly review the $\beta$-function and its implications
on the strong coupling, whereas in the following section we will discuss the
anomalous-mass dimension.

Regarding the $\beta$-function we notice, that $a_s(\mu)$ is not a constant, but
\textit{runs} by varying its scale $\mu$. Lets observe the running of the strong
coupling constant by integrating the $\beta$-function
\begin{equation}
  \int_{a_s(\mu_1)}^{a_s(\mu_2)}\frac{\dif a_s}{\beta(a_s)} = - \int_{\mu_1}^{\mu_2} \frac{\dif \mu}{\mu} = \log \frac{\mu_1}{\mu_2}.
\end{equation}
To analytically evaluate the above integral we can approximate the
$\beta$-function to first order, with the known coefficient
\begin{equation}
  \label{eq:firstBetaCoefficient}
  \beta_1 = \frac{1}{6}(11 N_c - 2 N_f),
\end{equation}
yielding
\begin{equation}
  a_s(\mu_2) = \frac{a_s(\mu_1)}{\left( 1 - a_s(\mu_1) \beta_1 \log\frac{\mu_1}{\mu_2} \right)}.
\end{equation}
As we have three colours $N_c=3$ and six flavours $N_f=6$ the first
$\beta$-function \ref{eq:betaFunction} is positive. Thus for $\mu_2>\mu_1$
$a_s(\mu_2)$ decreases logarithmically and vanishes for $\mu_2 \to \infty$. This
behaviour is known as \textit{asymptotic freedom}. The coefficients of the
$\beta$-function are currently known up to the 5th order and listed in the
appendix \ref{sec:betaCoefficients}.

As QCD we have three colors and six flavours the beta function carries a
negative sign (see \cref{eq:firstBetaCoefficient}). This implies that the
strength of the coupling is \textbf{decreasing} for increasing energies. In QED
we have a beta-function with a positive sign and thus its coupling strength
increases with energy. This behaviour of QCD leads to \textit{confinement} and
\textit{asymptotic freedom}.

Color confinement, or simply confinement, means that color charged quarks cannot
be isolated. They always appear in composite hadrons. Until today no quark has
been measured as single particle. There is no analytic proof of confinement, but
until today no single quark has been observed. It is qualitativly explained with
the gluon field carrying color charge. These gluons form so-called
\textit{flux-tubes} between quarks, which cause a constant strong force between
particles regardless of their separation. Consequently the energy needed to
separate quarks is proportional to the distance between them and at some point
there is enough energy to favour the creation of a new quark pair. Thus before
separating two quarks we create a two new quarks and we will probably never be
able to observe an isolated quark.

For high energies, or very close quarks, the QCD coupling becomes weak and the
quarks and gluons are essentially free. This phenomen is refered to as
asymptotic freedom.


\subsubsection{Running quark mass}
Not only the coupling but also the masses carry an energy dependencies, which is
governed by the \textit{anomalou mass dimension} $\gamma(a_s)$.

The properties of the running quark mass can be derived similar to the gauge
coupling. Starting from integrating the \textit{anomalous mass dimension}
\ref{eq:anomalousMassDimension}
\begin{equation}
  \log \frac{m(\mu_2)}{m(\mu_1)} = \int_{a_s(\mu_1)}^{a_s(\mu_2)} \dif a_s \frac{\gamma(a_s)}{\beta(a_s)}
\end{equation}
we can approximate the \textit{anomalous mass dimension} to first order and
solve the integral analytically \cite{Schwab2002}
\begin{equation}
  m(\mu_2) = m(\mu_1)\left( \frac{a(\mu_2)}{a(\mu_1)} \right)^{\frac{\gamma_1}{\beta_1}} \left( 1 + \mathcal{O}(\beta_2, \gamma_2) \right).
\end{equation}
As $\beta_1$ and $\gamma_1$ (see \ref{app:gammaCoefficients}) are positive the
quark mass decreases with increasing $\mu$. The general relation between
different scales is given by
\begin{equation}
  m(\mu_2) = m(\mu_1) \exp \left( \int_{a_s(\mu_1)}^{a_s(\mu_2)} \dif a_s \frac{\gamma(a_s)}{\beta(a_s)}  \right)
\end{equation}
and can be solved numerically to run the quark mass to the needed scale $\mu_2$.

QCD in general has a precision problem caused by uncertainties and largness of
the strong coupling constant $\alpha_s$. The fine-structure constant (the
coupling constant of QED) is known to eleven digits, whereas the strong coupling
is only known to about four. Furthermore for low energies the strong coupling
constant is much larger than the fine-structure constant. E.g. at the $Z$-mass,
the standard mass to compare the strong coupling, we have an $\alpha_s$ of
$0.11$, whereas the fine structure constant would be around $0.007$.
Consequently to use PT we have to calculate our results to much heigher orders,
including tens of thousends of Feynman diagrams, in QCD to achieve a precision
equal to QED. For even lower energies, around \SI{1}{\giga\eV}, the strong
coupling reaches a critical value of $\approx 0.5$ leading to a break down of
PT.

In this work we try to achieve a higher precision in the value of $\alpha_s$.
Our method to measure the strong coupling is called \textbf{QCD sum rules},
which by itself is based on a concept called the \textit{two-point function} for
which we will devote the following section.

\subsection{Two-Point function}
\label{sec:twoPointFunction}
The vacuum expectation value of the product of the conserved noether current
$J_\mu(x)$ at different space-times points $x$ and $y$ is known as the
\textbf{two-point function} (or simply \textbf{correlator})
\begin{equation}
  \label{eq:twoPointFunction}
  \Pi_{\mu\nu}(q^2) = \langle  0 | J_\mu(x) J_\nu(y) | 0 \rangle,
\end{equation}
where the noether current is given by
\begin{equation}
  J_\mu(x) = \anti{q}(x) \Gamma q(y)
\end{equation}
, where $\Gamma$ stands for one of the dirac matrices $\Gamma \in \{ 1,
i\gamma_5, \gamma_\mu, \gamma_\mu\gamma_5\}$, specifying the quantum number of
the current (S: \textit{scalar}, P: \textit{pseudo-Scalar}, V:
\textit{vectorial}, A: \textit{axial-vectorial}, respectively).

The correlator tensor $\Pi_{\mu\nu}(q^2)$ can be lorentz decomposed to a scalar
function $\Pi(q^2)$. There are only two possible terms that can reproduce the
second order tensor $q_\mu q_\nu$ and $q^2 g_{\mu\nu}$. The sum of both
multiplied with two arbitrary functions $A(q^2)$ and $B(q^2)$ yields
\begin{equation}
  \Pi_{\mu\nu}(q^2) = q_\mu q_\nu A(q^2) + q^2 g_{\mu\nu} B(q^2).
\end{equation}
By making use of the \textbf{Ward-identity} \cite{Peskin1995}
\begin{equation}
  \label{eq:wardIdentity}
  q^\mu \Pi_{\mu\nu}(q^2) = q^\nu \Pi_{\mu\nu} = 0
\end{equation}
we can demonstrate, that the two arbitrary functions are related
\begin{equation}
  \begin{split}
    q^\mu q^\nu \Pi_{\mu\nu} &= q^4 A(q^2) + q^4 B(q^2) = 0 \\
    &\quad \implies A(q^2) = -B(q^2).
  \end{split}
\end{equation}
Thus redefining $A(q^2) \equiv \Pi(q^2)$ we expressed the correlator as a scalar
function
\begin{equation}
  \Pi_{\mu\nu}(q^2) = (q_\mu q_\nu - q^2 g_{\mu\nu})\Pi(q^2).
\end{equation}

The scalar QCD two point function can then be related to the spectrum of
hadronic states. The correlator is then related to an integral over the
\textbf{spectral function} $\rho(s)$ via the \textit{Källén-Lehmann spectral
  representation} \cite{Kallen1952,Lehmann1954}, which is known since the early
fities
\begin{equation}
  \label{eq:dispersionRelation}
  \Pi(q^2) = \int_0^\infty \dif s \frac{\rho(s)}{s - q^2 - i \epsilon}.
\end{equation}

Equation \ref{eq:dispersionRelation} is refered to as \textbf{dispersion
  relation} analogous to similar relations which arise for example in
electrodynamics and defines the \textbf{spectral function} (a derivation can be
found in \cite{Rafael1997})
\begin{equation}
  \label{eq:spectralFunction}
  \rho(s) = \frac{1}{\pi} \Ima \Pi(s).
\end{equation}

Until know we connected theoretical correlators with the measurable hadronic
spectrum. Nevertheless the analytic properties of the correlators have to be
discussed as the function has discontinuities.

The main contribution from the spectral function given in
\cref{eq:dispersionRelation} are the hadronic final states
\begin{equation}
  2 \pi \rho(m^2) = \sum_n \langle  0 | J_\mu(x) | n \rangle \langle n | J_\nu(y) \rangle (2 \pi^2)^4 \delta^{(4)}(p - p_n),
\end{equation}
which lead to a series of continuous poles on the positive real axis for the
two-point function, see Fig. \ref{fig:analyticStructureCorrelator}.
\begin{figure}[h]
  \centering
  \includegraphics[width=0.8\textwidth]{./images/analyticStructureCorrelator.eps}
  \caption{Analytic structure in the complex $q^2$-plane of the Fourier
    transform of the two-point function. The hadronic final states are
    responsible for poles appearing on the real-axis. The one-particle states
    contribute as isolated pole and the multi-particle states contribute as
    bound-states poles or a continues ``discontinuity cut'' \cite{Peskin1995}.}
  \label{fig:analyticStructureCorrelator}
\end{figure}
These discontinuities can be tackled with \textit{Cauchy's theorem}, which we
will apply in \cref{secSumRules}.

Until now we exclusively dealt with the perturbative (PT) part of the theory,
but QCD is known to have not negligible non-perturbative (NPT) contributions.
Thus before continuing with the \textit{Sum Rules} we need a final ingredient
the operator product expansion, which implements NPT cotributions to our theory.


\subsection{Operator Product Expansion}
The \textbf{Operator Product Expansion} (OPE) was introduced by Wilson in 1969
\cite{Wilson1969}. The expansion states that non-local operators can be
rewritten into a sum of composite local operators and their corresponding
coefficients:
\begin{equation}
  \label{eq:OPE}
  \lim_{x\to y} \mathcal{O}_1(x) \mathcal{O}_2(y) = \sum_n C_n(x-y)\mathcal{O}_n(x),
\end{equation}
where $C_n(x-y)$ are the so-called \textit{Wilson-coefficients}.

The OPE lets us separate \textit{short-distance} from \textit{long-distance}
effects. In perturbation theory (PT) we can only amount for
\textit{short-distances}, which are equal to hight energies, where the
strong-coupling $\alpha_s$ is small. Consequently the OPE decodes the
long-distance effects in the higher dimensionsional operators.

The form of the composite operators are dictated by Gauge- and Lorentz symmetry.
Thus we can only make use of operators of even dimension. The operators up to
dimension six are given by \cite{Pascual1984}
\begin{equation}
  \begin{array}{ll}
    \text{Dimension 0:} & \mathbb{1} \\
    \text{Dimension 4:} & :m_i \anti{q} q: \\
                        & :G_a^{\mu\nu}(x) G_{\mu\nu}^a(x): \\
    \text{Dimension 6:} & :\anti{q} \Gamma q \anti{q} \Gamma q: \\
                        & :\anti{q} \Gamma \frac{\lambda^a}{2} q_\beta(x) \anti{q} \Gamma \frac{\lambda^a}{2} q: \\
                        & :m_i \anti{q} \frac{\lambda^a}{2} \sigma_{\mu\nu} q G^{\mu\nu}_a: \\
                        & :f_{abc} G_a^{\mu\nu} G_b^{\nu\delta} G_c^{\delta\mu}:,
  \end{array}
\end{equation}
where $\Gamma$ stands for one of the dirac matrices $\Gamma \in \{1, i \gamma_5,
\gamma^\mu, \gamma^\mu \gamma_5\}$, specifying the quantum number of the current
(S, P, A, respectively). As all the operators appear normal ordered they vanish
by definition in PT. Consequently they appear as \textbf{Condensates} in
Non-perturbative (NPT) QCD like quark-condensate $\langle \anti{q} q \rangle$ or
the gluon-condensate $\langle a GG \rangle$ (both of dimension four). These
non-vanishing condensats characterize the QCD-vacuum.

As we work with dimensionless functions (e.g. $\Pi$) in Sum Rules, the r.h.s. of
\cref{eq:ope} has to be dimensionless. Consequently the Wilson-coefficients have
to cancel the dimension of the operator with their inverse mass dimension. To
account for the dimensions we can make the inverse momenta explicit
\begin{equation}
  \Pi_{V/A}^{OPE}(s) = \sum_{D=0,2,4\dots} \frac{c^{(D)} \langle \mathcal{O}^{(D)}(x) \rangle}{-s^{D/2}},
\end{equation}
where we used $C^{(D)}=c/(-s)^{D/2}$ with $D$ being the dimension. Consequently
the OPE should converge with increasing dimension for suficienty large momenta
$s$.

Let's show how the OPE contributions are calculated with a the ``standard
example'' (following \cite{Pascual1986}), where we compute the perturbative and
quark-condensate Wilson-coefficients for the $\rho$-meson. For the $\rho$-meson,
which is composed of u and d quarks, the current of \cref{eq:twoPointFunction}
takes the following form
\begin{equation}
  j^\mu(x) = \frac{1}{2}\left(:[\anti{u} \gamma^\mu u](x) - \anti{d} \gamma^\mu d](x)\right).
\end{equation}
\begin{figure}
  \centering
  \includegraphics[width=\textwidth]{./images/condensateFeynmanDiagram.eps}
  \caption{Feynman diagrams of the perturbative (a) and the quark-condensate (b)
    contribution. The upper part of the right diagram is not wick-contracted and
    responsible for the condensate.}
  \label{fig:OPEFeynmanDiagram}
\end{figure}
In \cref{fig:OPEFeynmanDiagram} we draw the Feynman-diagram, from which we can
take the uncontracted mathematical expression for the scalar correlator
\begin{equation}
  \begin{split}
    \Pi(q^2) &= - \frac{i}{4 q^2 (D -1)} \int \dif^D x e^{iqx} \langle \Omega | T \{:\anti{u}(x) \gamma^\mu u(x) - \anti{d}(x) \gamma^{mu} d(x): \\
    &\quad\times :\anti{u}(0 \gamma_\mu u(0) - \anti{d}(0)\gamma_\mu d(0): \}
    \rangle.
  \end{split}
\end{equation}
Using Wick's theorem we can contract all of the fields and calculate the first
term of the OPE ($\mathbb{1}$), which represents the perturbative contribution
of the OPE ($\mathbb{1}$)
\begin{equation}
  \begin{split}
    \Pi(q^2) &= \frac{i}{4q^2(D-1)} (\gamma^\mu)_{ij} (\gamma_\mu)_{kl} \int \dif^D x e^{iqx} \\
    &\times\quad\left[ \wick{ \c u_{j \alpha}(x) \anti{\c u}_{k \beta}(0)} \cdot
      \wick{\c u_{l \beta}(0) \anti{\c u}_{i \alpha}(x)} + (u \to d)
    \right] \\
    &= \frac{3}{8\pi^2} \left[ \frac{5}{3} - \log\left( -\frac{q^2}{\nu^2}
      \right) \right].
  \end{split}
\end{equation}

To calculate the higher dimensional contributions of the OPE we use the same
techniques as before, but leave some of the fields uncontracted. For the
quark-condensate, which we want to derive for tree-level, we leave two fields
uncontracted
\begin{equation}
  \begin{split}
    \Pi(q^2) &= \frac{i}{4q^2(D-1)} (\gamma^\mu)_{ij} (\gamma_\mu)_{kl} \int \dif^D x e^{iqx} \left[ \phantom{\wick{ \c u_{j \alpha}(x) \anti{\c u}_{k \beta}(0)}} \right.\\
    &\quad+\,\wick{ \c u_{j \alpha}(x) \anti{\c u}_{k \beta}(0)} \cdot \langle \Omega \vert : \anti{u}_{i \alpha}(x) u_{l \beta} (0) :\vert \Omega \rangle \\
    &\quad\,\left.+\wick{\c u_{l \beta}(0) \anti{\c u}_{i \alpha}(x)} \cdot
      \langle \Omega \vert: \anti{u}_{k \beta}(0) u_{j \alpha}(x) : \vert \Omega
      \rangle + (u \to d) \right].
  \end{split}
\end{equation}
The non contracted fields can then be expanded in x
\begin{equation}
  \begin{split}
    \langle \Omega \vert: \anti{q}(x) q(0):\vert \Omega \rangle &= \langle\Omega\vert: \anti{q}(0) q(0): \vert\Omega\rangle\\
    &\quad+ \langle\Omega\vert:\left[ \partial_\mu \anti{q}(0) \right]
    q(0):\vert\Omega\rangle x^\mu + \dots
  \end{split}
\end{equation}
and redefined to a more elegant notation
\begin{equation}
  \langle \anti{q}q \rangle \equiv \langle\Omega\vert: \anti{q}(0) q(0):\vert\Omega\rangle.
\end{equation}
The finally result can be taken from \cite{Pascual1984} and yields
\begin{equation}
  \Pi_{(\rho)}(q^2) = \frac{1}{2} \frac{1}{\left(-q^2\right)^2} \left[ m_u\langle \anti{u} u + m_d \langle \anti{d} \rangle\right].
\end{equation}

The usage of the OPE and its validity is far from obvious. We are deriving the
OPE from matching the Wilson-coefficients to Feynman-graph analyses. These
Feynman-graphs are calculated perturbatively but the coefficients with dimension
$D>0$ correspond to NPT condensates!

Having gathered all of the necessary concepts we can close the gap between the
theory and experiment in the last section of the introduction: QCD Sum Rules.

\subsection{Sum Rules}
\label{sec:sumRules}
To relate the measurable hadronic final states of a QCD process (e.g.
$\tau$-decays into Hadrons) to a theoretical calculable \textbf{QCD sum rules}
have been empliyed by Shifman in the late sevent \cite{Shifman1978}.

The sum rules are a combination of the two-point function and its analyticity,
the OPE, a dispersion relation, the optical theorem and quark hadron duality.

The previously introduced two-point function \cref{eq:twoPointFunction} is
generally descriped by the OPE to account for NPT effects.
\begin{equation}
  \Pi(q^2) = \Pi^{OPE}(q^2).
\end{equation}
Furthermore it is related to the theoretical spectral function $\rho(s)$ via a
dispersion relation \cref(eq:dispersionRelation). Using QCD we are computing
interactions based on quarks and gluons, but due to confinement, we are only
able to observe Hadrons. Consequently to connect the theory to the experiment we
have to assume \textbf{quark-hadron duality}\footnote{Or simply duality.}, which
implies that physical quantities can be described equally good in the hadronic
or in the quark-gluon picture. Thus we can rewrite the dispersion relation
\cref{eq:dispersionRelation} as
\begin{equation}
  \Pi^{OPE}_{th}(q^2) = \int_0^\infty \frac{\rho_{exp}(q^2)}{(s-q^2-i\epsilon)},
\end{equation}
where we connected the theoretical correlator $\Pi_{th}$ with the experimental
measurable spectral function $\rho_{exp}$.

We have seen that the theoretical description of the correlator $\Pi_{th}$
contains poles on the real axis, but the experimental data $\rho_{exp}$ is
solely accesible on the positive real axis. Thus we have to make use of Cauchy's
theorem to access the theoretical values of the two-point function close to the
postive real axis (see \cref{fig:correlatorComplexContour}) given by
\begin{equation}
  \label{eq:cauchysTheorem}
  \int_{\mathcal{C}} f(z) \dif z = 0,
\end{equation}
where $f(z)$ is an analytic function on a closed contour $\mathcal{C}$.
\begin{figure}[h]
  \centering
  \label{fig:correlatorComplexContour}
  \includegraphics[width=0.6\textwidth]{./images/correlatorComplexContour.eps}
  \caption{Analytical structure of $\Pi(s)$ with the used contour $\mathcal{C}$
    for the final QCD Sum Rule expression \cref{eq:qcdSumRules}.}
\end{figure}

The final ingredient of the QCD sum rules is the \textit{optical theorem},
relating experimental data with the imaginary part of the correlator (the
spectral function $\rho(s)$).

In total, with the help Cauchy's theorem, the QCD sum rules can be sumed up in
the following expression
\begin{equation}
  \label{eq:qcdSumRules}
  \frac{1}{\pi}\int_0^\infty \frac{\rho_{exp}(t)}{t - s}\dif t = \frac{1}{\pi} \oint_{\mathcal{C}} \frac{\Ima \Pi_{OPE}(t)}{t -s}\dif t,
\end{equation}
where the l.h.s. is given by the experiment and the r.h.s. can be theoretically
evaluated with by applying the OPE of the correlator $\Pi_{OPE}(s)$.
\end{document}
% LocalWords:  qcdFeynmanDiagrams
