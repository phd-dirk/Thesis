\documentclass[../../index.tex]{subfiles}

\begin{document}
\pdfbookmark[0]{Abstract}{abstract} % Sets a PDF bookmark for the abstract
\chapter*{Abstract}
In this work we perform a \textsc{qcd} analysis on hadronic \(\tau\) lepton
decays. We make use of the \textsc{aleph} data to fit the strong coupling and
higher order \textsc{ope} contributions. Our approach is based on the
\textsc{qcdsr}, especially the framework of \textsc{fopt}, which we apply for
the \(V+A\) channel of the inclusive Cabibbo\-/allowed hadronic \(\tau\) decay
data. We perform this fits using a new set of analytic weight functions to shed
light on the discussion of the importance of \textsc{dv} while determining the
strong coupling \(\alpha_s\) at the \(m_\tau^2\) scale. We found a value of
\(\alpha_s(m_\tau^2) = 0.3261(51) \) for the strong coupling. For the
\textsc{ope} dimension six and eight contributions we found \(C_6 = -0.68(20)\)
and \(C_8 = -0.80(38) \). While performing fits using differently pinched
weights we showed that \textsc{dv} play only a minor role, even for single
pinched weights, if the fits are performed in the \(V+A\) channel, using
\textsc{fopt} with an \(s_{min} \gtrsim \SI{2.1}{\giga\eV}\). In comparing
fits performed in the framework of the \textsc{bs} with fits performed in
\textsc{fopt} we found that these fits highly agree. In the discussion of
\textsc{fopt} vs \textsc{cipt} we consequently favour \textsc{fopt} and
discourage the use of \textsc{cipt}.

\end{document}