\documentclass[../../index.tex]{subfiles}

\begin{document}
\pdfbookmark[0]{Abstract}{abstract} % Sets a PDF bookmark for the abstract
\chapter*{Abstract}
In this work, we perform a \define{qcd}{quantum chromodynamics} analysis on
hadronic \(\tau\) lepton decays. We make use of the \textsc{aleph} data to fit
the strong coupling and higher order \define{ope}{operator product expansion}
contributions. Our approach is based on the \define{qcdsr}{\textsc{qcd} sum
  rules}, especially the framework of \define{fopt}{fixed\-/order perturbation
  theory}, which we apply for the \(V+A\) channel of the inclusive
Cabibbo\-/allowed hadronic \(\tau\) decay data. We perform this fits using a new
set of analytic weight functions to shed light on the discussion of the
importance of \define{dv}{duality violation}. Since the inclusion of a model to
parametrise contributions of \textsc{dv} by Boito et al. \cite{2011a} there has
been an ongoing discussion, especially with the group around Pich
\cite{Pich2016}, which disfavours the usage of the \textsc{dv} model. Within
this work, we want to give a third opinion arguing that \textsc{dv} are not
present in double pinched weights. Even for single pinched weights, we find that
\textsc{dv} are sufficiently suppressed for high precision measurements of the
strong coupling. Another unsolved topic is the discussion of \textsc{fopt} vs
\define{cipt}{contour\-/improved perturbation theory}. Beneke et al.
\cite{Beneke2008} have found that \textsc{cipt} cannot reproduce the
\define{bs}{Borel summation}, while the creators of \textsc{cipt}
\cite{Pivovarov1991, LeDiberder1992a} are in favour of the framework. To
investigate the validity of \textsc{fopt} we apply the \textsc{bs}. The
parameters we obtain from both frameworks are in high agreement. Performing fits
in the framework of \textsc{cipt} lead to different results. Consequently, in
the discussion of \textsc{fopt} vs \textsc{cipt} we argue for \textsc{fopt}
being the favoured framework. For our final result of the strong coupling we
perform fits for ten different weights. For each weight, we further fit 20
different moments by varying the energy limit \(s_0\). We select the best fit of
each weight in a final comparison. The fits are in high agreement and the
average of the parameters we obtain yields a value of \(\alpha_s(m_\tau^2) =
0.3261(51) \) for the strong coupling, \(\rho^{(6)} = -0.68(20)\) for the
dimension six \textsc{ope} contribution and \(\rho^{(8)} = -0.80(38) \) for the
dimension eight \textsc{ope} contribution.

\end{document}