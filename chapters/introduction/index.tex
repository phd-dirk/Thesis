\documentclass[../../index.tex]{subfiles}

\begin{document}
\chapter{Introduction}
In particle physics we are concerned about small objects and their interactions.
Their dynamics are currently best described by the Standard Model (SM).

The SM contains two groups of fermionic, Spin 1/2 particles. The former group,
the Leptons consist of: the electron ($e$), the muon ($\mu$), the tau ($\tau$)
and their corresponding neutrinos $\nu_e$, $\nu_\mu$ and $\nu_\tau$. The latter
group, the Quarks contain: $u$, $d$ (up and down, the so called light quarks ),
$s$ (strange), $c$ (charm), $b$ (beauty or beauty) and $t$ (top or truth). The SM
furthermore differenciates between three fundamental forces (and its carriers):
the electromagnetic ($\gamma$ photon), weak ($Z$- or $W$-Boson) and strong ($g$
gluon) interactions. The before mentioned Leptons solely interact through the
electromagnetic and the weak force (also refered to as electroweak interaction),
whereas the quarks additionally interact through the strong force.

The strong force is also refered to as Quantumchromodynamics
(QCD). As the name suggest\footnote{Chromo is the greek word for color.} the
force is characterized by the color charge. Every quark has next to its type one
of the three colors blue, red or green. The color force is mediated through
eight gluons, which each being bi-colored\footnote{Each gluon carries a color
  and an anti-color.}, interact with quarks and each other. The strength of the
strong force is given by the coupling constant $\alpha_s$. The coupling
constants are a function of energy $E$ and $\alpha_s(E)$ increases with 
energy\footnote{In contrast to the electromagnetic force, where $\alpha(E)$
  decreases!}. This is exclusive for QCD and leads to \textit{asymptotic freedom} an
\textit{confinement}. The former phenomen describes the decreasing strong force
between quarks and gluons, which become asymptotically free at large
energies. The latter expresses the fact, that no isolated quark has been found
until today. Quarks appear confined as \textit{Hadrons}, the so called
\textit{Mesons}\footnote{Composite of a quark and an anti-quark.} and
\textit{Baryons}\footnote{Composite of three quarks or three anti-quarks.}.
As we measure \textit{Hadrons} in our experiments but calculate with quarks
within our theoretical QCD model we have to assume \textit{Quark-Hadron
  Duality}, which states that QCD is still valid for Hadrons for energies
sufficently heigh energies. There exist \textit{Duality Violations} (DV), which
will be investigated within this work.


\section{$\tau$-Decays}
\label{sec:tauDecays}


\section{Quantumchromodynamics}
\label{sec:quantumchromodynamics}
QCD describes the strong interaction, which occur between \textit{quarks} and
are transmitted through \textit{gluons}. A list of quarks can be found in
\ref{table:quarkList}.
\begin{table}
  \label{table:quarkList}
  \centering
  \begin{tabular}{l l c}
    \toprule
    Flavour & Mass & comment\\
    \midrule
    $u$ & $2.2_{-0.4}^{+0.5}$ \SI{}{\mega\eV} & $\overline{\text{MS}}$ \\
    $d$ & $4.7_{-0.3}^{+0.5}$ \SI{}{\mega\eV} \\
    $s$ & $95_{-3}^{+9}$ \SI{}{\mega\eV}  \\
    $c$ & $1.275_{-0.035}^{+0.025}$ \SI{}{\giga\eV} \\
    $b$ & $4.18_{-0.03}^{+0.04}$ \SI{}{\giga\eV} \\
    $t$ & \SI{173.0 \pm 4}{\giga\eV} \\
    \bottomrule 
  \end{tabular}
  \caption{List of Quarks and their masses\cite{PDG2018}.}
\end{table}

The QCD Lagrange density is similar to that of QED\cite{Jamin2006},
\begin{equation}
  \label{eq:qcdLagrangian}
  \mathcal{L}_{QCD}(x) = -\frac{1}{4}G_{\mu\nu}^a(x)G^{\mu\nu a}(x) + \sum_A \left[ \frac{i}{2} \bar{q}^A(x) \gamma^\mu \overleftrightarrow{D}_\mu q^A(x) - m_A\bar{q}^A(x) a^A(x) \right],
\end{equation}
where $q^A(x)$ represents the quark fields and $G_{\mu\nu}^a$ being the \textit{gluon field strength tensor} given by:
\begin{equation}
  \label{eq:gluonField}
  G_{\mu\nu}^a(x) \equiv \partial_\mu B_\nu^a(x) - \partial_\nu^a(x) + g f^{abc} B_\mu^b(x) B_\nu^c(x),
\end{equation}
where $B_\mu^a$ are the \textit{gluon fields}, given in the \textit{adjoint
  representation} of the SU(3) gauge group with $f^{abc}$ as \textit{structure
  constants}. Furthermore we have used $A, B, \dots$ as flavour indeces, $a,
b, \dots$ as color indeces and $\mu, \nu, \dots$ as lorentz indeces.


\subsection{Renormalisation Group}
The perurbations of the QCD Lagrangian \ref{eq:qcdLagrangian} lead to divergencies, which have to be
\textit{renormalized}. There are different aproaches to 'make' these
divergencies finite. The most popular one is \textit{dimensional
  regularisation}.
In \textit{Dimensional regularisation} we expand the four space-time dimensions
to arbitrary dimensions. Consequently the in QCD calculations appearing
$\textit{Feyman integrals}$ have to be continued to $D$-dimensions like
\begin{equation}
  \label{eq:dimRegFeynmanIntegral}
  \mu^{2\epsilon} \int \frac{\dif^D p}{(2\pi)^D}\frac{1}{[p^2-m^2+i0][(q-p)^2=m^2+i0]},
\end{equation}
where we introduced the scale parameter $\mu$ to account for the extra
dimensions and conserve the mass dimension of the non continued integral.

In addition \textit{physical quantities}\footnote{Observables that can be
  measured.} cannot depend on the renormalisation scale $\mu$. Thus examining a \textit{physical quantity} $R(q, a_s, m)$ that depends on the
external momentum q, the renormalised coupling $a_s=\alpha_s/\pi$ and the renormalized quark mass $m$ 
\begin{equation}
  \mu \od{}{\mu}R(q, a_s, m) = \left[ \mu \pd{}{\mu} + \mu \od{m}{\mu} \pd{}{m} \right] R(q, a_s, m) = 0
\end{equation}
we can define the \textit{renormalisation group functions}:
\begin{align}
  \beta(a_s) &\equiv -\mu \od{a_s}{\mu} = \beta_1 a_s^2 + \beta_2 a_s^3 + \dots & \beta-\text{function}
  \label{eq:betaFunction} \\
  \gamma(a_s) &\equiv - \frac{\mu}{m} \od{m}{\mu} = \gamma_1 a_s + \gamma_2 a_s^2 + \dots & \text{anomalous mass dimension}.
  \label{eq:anomalousMassDimension}
\end{align}

\subsubsection{Running gauge coupling}
Regarding the $\beta$-function we notice, that $a_s(\mu)$ is not a constant, but
\textit{runs} by varying the scale $\mu$. Integrating the $\beta$-function yields 
\begin{equation}
  \int_{a_s(\mu_1)}^{a_s(\mu_2)}\frac{\dif a_s}{\beta(a_s)} = - \int_{\mu_1}^{\mu_2} \frac{\dif \mu}{\mu} = \log \frac{\mu_1}{\mu_2}.
\end{equation}
To analytically evaluate the above integral we can approximate the $\beta$-function to first order, with the known
coefficient
\begin{equation}
  \beta_1 = \frac{1}{6}(11 N_c - 2 N_f),
\end{equation}
yielding
\begin{equation}
  a_s(\mu_2) = \frac{a_s(\mu_1)}{\left( 1 - a_s(\mu_1) \beta_1 \log\frac{\mu_1}{\mu_2} \right)}.
\end{equation}
As we have three colours $N_c=3$ and six flavours $N_f=6$ the first
$\beta$-function \ref{eq:betaFunction} is positive. Thus for $\mu_2>\mu_1$ $a_s(\mu_2)$ decreases
logarithmically and vanishes for $\mu_2 \to \infty$. This behaviour is known as
\textit{asymptotic freedom}.
The coefficients of the $\beta$-function are currently known up to the 5th
order, which are displayed in the appendix \ref{sec:betaCoefficients}.

\subsubsection{Running quark mass}
The properties of the running quark mass can be derived similar to the gauge
coupling. Starting from integrating the \textit{anomalous mass dimension} \ref{eq:anomalousMassDimension}
\begin{equation}
  \log \frac{m(\mu_2)}{m(\mu_1)} = \int_{a_s(\mu_1)}^{a_s(\mu_2)} \dif a_s \frac{\gamma(a_s)}{\beta(a_s)}
\end{equation}
we can approximate the \textit{anomalous mass dimension} to first order and
solve the integral analytically \cite{Schwab2002}
\begin{equation}
  m(\mu_2) = m(\mu_1)\left( \frac{a(\mu_2)}{a(\mu_1)} \right)^{\frac{\gamma_1}{\beta_1}} \left( 1 + \mathcal{O}(\beta_2, \gamma_2) \right).
\end{equation}
As $\beta_1$ and $\gamma_1$ (see \ref{app:gammaCoefficients}) are positive the
quark mass decreases with increasing $\mu$.
The general relation between different scales is given by
\begin{equation}
  m(\mu_2) = m(\mu_1) \exp \left( \int_{a_s(\mu_1)}^{a_s(\mu_2)} \dif a_s \frac{\gamma(a_s)}{\beta(a_s)}  \right)
\end{equation}
and can be solved numerically to run the quark mass to the needed scale $\mu_2$.
\end{document}