\documentclass[../../index.tex]{subfiles}

\begin{document}
\chapter{Introduction}
In particle physics we are concerned about small objects and their interactions.
The smallest of these objects are referred to as \textit{elemental particles}.
Their dynamics are governed by the laws of nature. These laws are organised
through symmetries, which are currently best described by the \textit{Standard
  Model} (\textsc{sm}).

The \textsc{sm} classifies all known elementary particles and describes three of
the four fundamental forces: the electromagnetic, weak and strong force. The
particles representing matter are contained in two groups of fermionic,
spin\-/1/2 particles. The former group, the leptons consist of: the electron
($e$), the muon ($\mu$), the tau ($\tau$) and their corresponding neutrinos
$\nu_e$, $\nu_\mu$ and $\nu_\tau$. The latter group, the quarks contain: $u$,
$d$ (up and down, the so called light quarks ), $s$ (strange), $c$ (charm), $b$
(bottom or beauty) and $t$ (top or truth). The three fundamental forces, the
\textsc{sm} differentiates, are described through their carrier particles, the
so\-/called bosons: the photon for the electromagnetic, the $Z$\-/ or
$W$\-/Boson for the weak and the gluon ($g$) for the strong interaction. and
strong ($g$ gluon) interactions. The before mentioned Leptons solely interact
through the electromagnetic and the weak force (also referred to as electroweak
interaction), whereas the quarks additionally interact through the strong force.
A short summary of the taxonomy of the \textsc{sm} can be seen in
\cref{fig:SMTaxonomy}
\begin{figure}
  \centering
  \includegraphics[width=\textwidth]{./images/standardModelTaxonomy.eps}
  \caption{Taxonomy of the Standard Model.}
  \label{fig:SMTaxonomy}
\end{figure}

From a more mathematical point of view the \textsc{sm} is a gauge \textit{
  quantum field theory} (\textsc{qft}). \textsc{qft} is the combination of
\textit{classical field theory}, \textit{special relativity} and \textit{quantum
  mechanics}. Its fundamental objects are ruled through its gauge\-/group
$SU(3)\times SU(2)\times U(1)$. Each of its subgroups introduces a global and a
local gauge symmetry. The global symmetry introduces the charges, which the
fields are carrying. The local symmetry introduce the gauge-fields, which
represent the previously mentioned force carriers. Naively every
subgroup\footnote{Actually $U(1)$ and $SU(2)$ have to be regarded as combined
  group to be mapped to the electromagnetic\-/ and weak\-/force in form of the
  electroweak interaction.} of the gauge\-/group of the standard model is
responsible for one of the three forces:
\begin{description}
\item[$\bm{U(1)}$] the \textit{abelian} gauge group governs the representation
  of \textit{quantum electrodynamics} (\textsc{qed}), which is commonly known as
  the electric force. Its global and local symmetry introduces the electric
  charge and the photon\-/field.
\item[$\bm{SU(2)}$] Is the \textit{non\-/abelian} symmetry group responsible for
  the weak\-/interaction. It introduces the $W^+,W^-$ and $Z$ bosons and the
  weak charge. The gauge groups $U(1)$ and $SU(2)$ have been combined to the
  \textit{electroweak interaction}.
\item[$\bm{SU(3)}$] The $SU(3)$\-/group is also \textit{non-abelian} and governs
  the strong interactions, which are summarised in the theory of \textit{quantum
    chromodynamics} (\textsc{qcd}). The group yields the three colour charges
  and due to its eight\-/dimensional adjoint\-/representation, eight different
  gluons.
\end{description}
Unfortunately we are still not able to include gravity, the last of the four
forces, into the \textsc{sm}. There have been attempts to describe gravity
through \textsc{qft} with the graviton, a spin\-/2 boson, as mediator, but there
are unsolved problems with the renormalisation of general relativity
(\textsc{gr}). Until now \textsc{gr} and quantum mechanics (\textsc{qm}) remain
incompatible.

Apart from gravity no being included, the \textsc{sm} has a variety of flaws.
One of them is being dependent on many parameters, which have to be measured
accurately to perform high\-/precision physics. In total the Lagrangian of the
\textsc{sm} contains 19 parameters. These parameters are represented by ten
masses, four \textsc{ckm}\-/matrix parameters, the \textsc{qcd}\-/vacuum angle,
the Higgs\-/vacuum expectation value and three gauge coupling constants. Highly
accurate values with low errors are crucial for theoretical calculated
predictions. One of the major error inputs of every theoretical output are
uncertainties in these parameters. In this work we will focus on one of the
parameters, namely the strong coupling $\alpha_s$.

\begin{wrapfigure}{r}{4cm}
  \includegraphics[width=0.3\textwidth]{./images/alphasDetermination.png}
  \caption{The six different subfields and their results for measuring the
    strong coupling $\alpha_s$ \cite{PDG2018}.}
  \label{fig:alphaSDetermination}
\end{wrapfigure}
The strong coupling is currently measured in six different ways: through
$\tau$\-/decays, \textsc{qcd}\-/lattice computations, deep inelastic collider
results and electroweak precision fits \cite{PSG2018}. We have plotted the
values of each of the methods in \cref{fig:alphaSDetermination}. During this
work we will focus on the subfield of $\tau$\-/decays to measure the value of
the strong coupling $\alpha_s(m_\tau)$ at the $\tau$\-/scale. We will see that
in \textsc{qcd} the value of the coupling ``constant'' depends upon the scale.
The $\tau$ is an elementary particle with negative electric charge and a spin of
1/2. Together with the lighter electron and muon it forms the group of charged
Leptons\footnote{Leptons do not interact via the strong force.}. Even though it
is an elementary particle it decays via the weak interaction with a lifetime of
$\tau_\tau=\SI{2.9e-13}{\second}$ and has a mass of
$\SI{1776.86\pm0.12}{\mega\electronvolt}$\cite{PDG2018}. It is furthermore the
only lepton massive enough to decay into hadrons, thus of interest for our
\textsc{qcd} study. The final states of a decay are limited by conservation
laws. In case of a $\tau$-decay they must conserve the electric charge ($-1$)
and invariant mass of the system. Thus, we can see from the corresponding
Feynman diagram \cref{fig:tauDecay} \footnote{The $\tau$-particle can also decay
  into strange quarks or charm quarks, but these decays are rather uncommon due
  to the heavy masses of s and c.}, that the $\tau$ decays by the emission of a
$W$\-/boson and a tau-neutrino $\nu_\tau$ into pairs of $(e^-, \bar\nu_e),
(\mu^-, \bar\nu_\mu)$ or $(q, \bar q)$.
\begin{figure}
  \centering \includegraphics[width=0.4\textwidth]{images/tauDecay.eps}
  \caption{Feynman diagram of common decay of a $\tau$-lepton into pairs of
    lepton-antineutrino or quark-antiquark by the emission of a \textit{W
      boson}.}
  \label{fig:tauDecay}
\end{figure}
We are foremost interested into the hadronic decay channels, meaning
$\tau$-decays that have quarks in their final states. Quarks have never been
measured isolated, but appear always in combination of \textit{mesons} and
\textit{baryons}. Due to its mass of $m_\tau \approx
\SI{1.8}{\giga\electronvolt}$ the $\tau$-particle decays into light mesons
(pions-$\pi$, kaons-$K$, and eta-$\eta$, see \cref{table:lightMesons}), which
can be experimentally detected.
\begin{table}
  \centering
  \begin{tabular}{lccc}
    \toprule
    Name & Symbol & Quark content & Rest mass ($\SI{}{\mega\electronvolt}$) \\
    \midrule
    Pion & $\pi^-$ & $\bar u d$ & \SI{139.57061 \pm 0.00024}{\mega\electronvolt}  \\
    Pion & $\pi^0$ & $(u \bar u - d \bar d)/\sqrt{2}$ & \SI{134.9770\pm0.0005}{\mega\electronvolt} \\
    Kaon & $K^-$ & $\bar u s$ & \SI{493.677\pm0.016}{\mega\electronvolt} \\
    Kaon & $K^0$ & $d \bar s$ & \SI{497.611\pm0.013}{\mega\electronvolt} \\
    Eta & $\eta$ & $(u \bar u + d \bar d - 2 s \bar s)/\sqrt{6}$ & \SI{547.862\pm0.017}{\mega\electronvolt} \\
  \end{tabular}
  \caption{List of mesons produced by a $\tau$-decay. Rare final states with
    branching Ratios smaller than 0.1 have been omitted. The list is taken from
    \cite{Davier2006} with corresponding rest masses taken from \cite{PDG2018}.}
  \label{table:lightMesons}
\end{table}
The hadronic $\tau$\-/decay provides one of the most precise ways to determine
the strong coupling \cite{Pich2016} and is theoretically accessible to high
precision within the framework of \textsc{qcd}.

The theory describing strong interactions is \textsc{QCD}. As the name
suggest\footnote{Chromo is the Greek word for colour.} \textsc{qcd} is
characterised by the colour charge and is a non-abelian gauge theory with
symmetry group $SU(3)$. Consequently every quark has next to its type one of the
three colours blue, red or green and the colour force is mediated through eight
gluons, which each being bi\-/coloured\footnote{Each gluon carries a colour and
  an anti\-/colour.}, interact with quarks and each other. The strength of the
strong force is given by the coupling constant $\alpha_s$, which depends on the
renormalisation\-/scale $\mu$. We often chosen the renormalisation\-/scale in a
way that the coupling constant $\alpha_s(q)$ depends on the energy $q^2$. Thus
the coupling varies with energy. It increases for low and decreases for high
energies\footnote{In contrast to the electromagnetic force, where $\alpha(q^2)$
  decreases!}. This behaviour has two main implications. The first one states,
that for low energies the coupling is too strong for isolated quarks to exist.
Until now we have not been able to observe an isolated quark and all experiments
can only measure quark compositions. These bound states are called
\textit{hadrons} and consist of two or three quarks \footnote{There exist also
  so\-/called \textit{Exotic hadrons}, which have more than three valence
  quarks.}, which are referred to as mesons\footnote{Composite of a quark and an
  anti\-/quark.} or baryons \footnote{Composite of three quarks or three
  anti\-/quarks.} respectively. This phenomenon, of quarks sticking together as
hadrons is referred to as \textit{confinement}. As the fundamental degrees of
freedom of \textsc{qcd} are given by quarks and gluons, but the observed
particles are hadrons we need to introduce the assumption of
\textit{quark\-/hadron duality} to match the theory to the experiment. This
means that a physical quantity should be similarly describable in the hadronic
picture or quark\-/gluon picture and that both descriptions are equivalent. As
we will see in our work quark\-/hadron duality is violated for low energies.
These so\-/called \textit{duality violations} have an impact on our strong
coupling determinations and can be dealt with either suppression or the
inclusion of a model \cite{Pich2006,Cata2008}. Throughout this work we will
favour and argument for the former approach. The second implication concerns
\textit{perturbation theory} (PT). The lower the energies we deal with, the
higher the value of the strong coupling and the contributions of
\textit{non\-/perturbative} (NP) effects. Currently there are three solutions to
deal with NP effects:
\begin{itemize}
\item \textbf{\textit{Chiral Perturbation Theory}} (\textsc{chpt}): Introduced
  by Weinberg \cite{Weinberg1978} in the late seventies. \textsc{chpt} is an
  effective field theory constructed with a Lagrangian symmetric under a
  chiral\-/transformation in the limit of massless quarks. It's limitations are
  based in the chiral symmetry, which is only a good approximation for the light
  quarks $u$, $d$ and in some cases $s$.
\item \textbf{\textit{Lattice QCD}} (\textsc{lqcd}): Is the numerical approach
  to the strong force. Based on the Wilson Loops \cite{Wilson1974} we treat QCD
  on a finite lattice instead of working with continuous fields. LQCD has
  already many applications but is limited due to its computational expensive
  calculations.
\item \textbf{\textit{QCD Sum Rules}} (\textsc{qcdsr}): Was also introduced in
  the late seventies by Shifman, Vainstein and Zakharov
  \cite{Shifman1978,Shifman1978a}. It relates the observed hadronic picture to
  quark\-/gluon parameters through a dispersion relation and the use of the
  \textit{Operator Product Expansion} (OPE), which treats NP effect through the
  definition of vacuum expectation values, the so\-/called \textit{QCD
    condensates}. It is a precise method for extracting the strong coupling
  $\alpha_s$ at low energies, although limited to the unknown higher order
  contributions of the OPE.
\end{itemize}


In this work we focus on the determination of the strong coupling $\alpha_s$
within the framework of \textsc{qcdsr} for $\tau$\-/decays which has been
exploited in the beginning of the nineties by Braaten, Narison and Pich
\cite{Braaten1991}. Within this setup we can measure $\alpha_s(m_\tau^2)$ at the
$m_\tau$ scale. As the strong coupling gets smaller at higher energies, so do
the errors. Thus if we obtain the strong coupling at a low scale we will obtain
high precision values at the scale of the Z\-/boson mass $m_Z$, which is the
standard scale to compare $\alpha_s$ values.

The \textsc{qcdsr} for the determination of $\alpha_s$, from low energies,
contain three major issues.
\begin{enumerate}
\item There are two different approaches to treat perturbative and
  non\-/perturbative contributions. In particular, there is a significant
  difference between results obtained using fixed\-/order (\textsc{fopt}) or
  contour improved perturbation theory (\textsc{cipt}), such that analyses based
  on \textsc{cipt} generally arrive at about 7\% larger values of
  $\alpha_s(m_{\tau^2})$ than those based on FOPT \cite{PDG2018}. There have
  been a variety of analyses on the topic been performed
  \cite{Pich2013,Caprini2009,Jamin2005} and we will favour the FOPT approach,
  but generously list our results for the \textsc{cipt} framework.

\item There are several prescriptions to deal with the NP-contributions of
  higher order OPE condensates. Typically terms of higher dimension have been
  neglected, even if they knowingly contribute. In this work we will include
  every necessary OPE term.

\item Finally there are known DV leading to an ongoing discussion of the
  importance of contributions from DV. Currently there are two main approaches:
  Either we neglect them, arguing that they are sufficiently suppressed due to
  \textit{pinched weights} \cite{Pich2016} or model DV with sinusoidal
  exponentially suppressed function \cite{Cata2008,Boito2011a,Boito2014}
  introducing extra fitting parameters. We will argue for the former method,
  implementing pinched weights that sufficiently suppress DV contributions such
  as having only a negligible effect on our analysis.
\end{enumerate}


In the first chapter of this work we want to summarise the necessary theoretical
background for working with the \textsc{qcdsr}. Starting with the basics of QCD
we want to motivate the \textit{renormalisation group equation} (\textsc{rge}),
which is responsible for the running of the strong coupling. We then continue
with the some aspects of the two\-/point function and its usage in the
dispersion relation, which connects the hadronic picture with the quark\-/gluon
picture. $\cdots$


\end{document}
% LocalWords:  SMTaxonomy tauDecay alphaSDetermination lightMesons lccc
