\message{ !name(index.tex)}\documentclass[../../index.tex]{subfiles}

\begin{document}

\message{ !name(index.tex) !offset(-3) }

\chapter{Introduction}
In particle physics we are concerned about small objects and their interactions.
Their dynamics are currently best described by the Standard Model (SM).

The SM contains two groups of fermionic, Spin 1/2 particles. The former group,
the Leptons consist of: the electron ($e$), the muon ($\mu$), the tau ($\tau$)
and their corresponding neutrinos $\nu_e$, $\nu_\mu$ and $\nu_\tau$. The latter
group, the Quarks contain: $u$, $d$ (up and down, the so called light quarks ),
$s$ (strange), $c$ (charm), $b$ (beauty or beauty) and $t$ (top or truth). The SM
furthermore differenciates between three fundamental forces (and its carriers):
the electromagnetic ($\gamma$ photon), weak ($Z$- or $W$-Boson) and strong ($g$
gluon) interactions. The before mentioned Leptons solely interact through the
electromagnetic and the weak force (also refered to as electroweak interaction),
whereas the quarks additionally interact through the strong force.

The strong force is also refered to as Quantumchromodynamics
(QCD). As the name suggest\footnote{Chromo is the greek word for color.} the
force is characterized by the color charge. Every quark has next to its type one
of the three colors blue, red or green. The color force is mediated through
eight gluons, which each being bi-colored\footnote{Each gluon carries a color
  and an anti-color.}, interact with quarks and each other. The strength of the
strong force is given by the coupling constant $\alpha_s$. The coupling
constants are a function of energy $E$ and $\alpha_s(E)$ increases with 
energy\footnote{In contrast to the electromagnetic force, where $\alpha(E)$
  decreases!}. This is exclusive for QCD and leads to \textit{asymptotic freedom} an
\textit{confinement}. The former phenomen describes the decreasing strong force
between quarks and gluons, which become asymptotically free at large
energies. The latter expresses the fact, that no isolated quark has been found
until today. Quarks appear confined as \textit{Hadrons}, the so called
\textit{Mesons}\footnote{Composite of a quark and an anti-quark.} and
\textit{Baryons}\footnote{Composite of three quarks or three anti-quarks.}.
As we measure \textit{Hadrons} in our experiments but calculate with quarks
within our theoretical QCD model we have to assume \textit{Quark-Hadron
  Duality}, which states that QCD is still valid for Hadrons for energies
sufficently heigh energies. There exist \textit{Duality Violations} (DV), which
will be investigated within this work.






\end{document}

\message{ !name(index.tex) !offset(-51) }
