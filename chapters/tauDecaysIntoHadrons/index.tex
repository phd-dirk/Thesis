\documentclass[../../index.tex]{subfiles}

\begin{document}
\chapter{$\tau$ decays into hadrons}

\cite{Tsai1971}
\begin{equation}
  R_\tau = 12 \pi \int_0^{m_\tau} = \frac{\dif s}{m_\tau^2}
  \left( 1 - \frac{s}{m_\tau^2} \right)
  \left[ \left( 1 + 2 \frac{s}{m_\tau^2} \right) \Ima \Pi^{(T)}(s) + \Ima \Pi^{(L)} \right]
\end{equation}

Cauchy's Theorem
\begin{equation}
  \int_{\mathcal{C}} f(z) \dif z = 0
\end{equation}

\begin{equation}
  \begin{split}
  \oint_{s=m_\tau} \Pi(s) &= \int_0^{m_\tau} \Pi(s + i \epsilon) + \int_{\mathcal{C}_2}\Pi(s) \dif s + \int_{m_\tau}^0 \Pi(s - i \epsilon) \dif s + \int_{\mathcal{C}_4} \Pi(s) \dif s \\
  &= \int_0^{m_\tau} \Pi(s+i \epsilon) - \Pi(s - i \epsilon) \dif s  + \int_{\mathcal{C}_2}\Pi(s) \dif s + \int_{\mathcal{C}_4} \Pi(s) \dif s \\
  &= \int_0^{m_\tau} \Pi(s + i \epsilon) - \overline{\Pi(s + i \epsilon)} + \int_{\mathcal{C}_2}\Pi(s) \dif s + \int_{\mathcal{C}_4} \Pi(s) \dif s \\
  &\overset{\lim \epsilon \to 0}{=} 2 i \int_0^{m_\tau} \Ima \Pi(s) \dif s + \oint_{s=m_\tau} \Pi(s) \dif s
  \end{split}
\end{equation}
where we made use of $\Pi(z) = \overline{\Pi(\overline z)}$, because $\Pi(s)$ is analytic and
$\Pi(z) - \overline{\Pi(z)} = 2 i \Ima \Pi(z)$

\begin{equation}
  \int_0^{m_\tau} \Pi(s) \dif s = \frac{i}{2} \oint_{s=m_\tau} \Pi(s) \dif s
\end{equation}

\begin{equation}
  R_\tau = 6 \pi i \oint_{s=m_\tau} \frac{\dif s}{m_\tau^2}
  \left( 1 - \frac{s}{m_\tau^2} \right)
  \left[ \left( 1 + 2 \frac{s}{m_\tau^2} \right) \Pi^{(T)}(s) + \Pi^{(L)} \right]
\end{equation}

\end{document}