\documentclass[../../index.tex]{subfiles}

\begin{document}
\chapter{Measuring the strong coupling}


\begin{wraptable}{l}{7cm}
  \caption{Timeline}\vskip -2.5ex
  \begin{tabular}{@{\,}r <{\hskip 5pt} !{\foo} >{\raggedright\arraybackslash}p{5cm}}
    \toprule
    \addlinespace[1.5ex]
    1991 & \cite{Braaten1991}: Systematic description, including \textsc{np} corrections to extract \(\alpha_s\) from \(R_\tau\). \\
    1992 & \cite{LeDiberder1992}: Introducing weights and fit methodology later used by \textsc{aleph} \cite{Aleph1993} and \textsc{opal} \cite{Opal1998} collaborations \\
    2011 & \cite{Boito2011a}: Include \textsc{dv} model to extract \(\alpha_s\) from \textsc{opal} data. \\
    2018 & \cite{Boito2018}: \(\alpha_s\) from \(e^+e^-\to \text{hadrons}\) up to \SI{2}{\giga\eV}.
  \end{tabular}
  \label{table:AlphasTauTimeline}
\end{wraptable}
The strong coupling has been measured since many years from hadronic \(\tau\)
decays. Until today most of the applied \textsc{qcdsr} to \(\tau\) decays are
based on the methodology developed in the early nineties by Braaten,
Pich and Narison \cite{Braaten1991}. They gathered the at this time available
perturbative and \textsc{np} contributions to extract the strong coupling from
comparing their theoretical results to the known inclusive hadronic \(\tau\)
decay ratio \(R_\tau\). Pich together with Le Dibgerger then formulated the fitting strategy of fitting
multiple moments of different weights to extract \(\alpha_s\) parallel to Wilson
coefficients of the \textsc{ope} \cite{LeDiberger1992}, which later has been
applied as standard in the \textsc{aleph} \cite{Aleph1993} as well as the
\textsc{opal} \cite{Opal1998} detectors. For the next ten year years the
methodology of extracting the strong coupling did not run through make major
changes until in the year 2011 when Boito, Cata, Golterman, Jamin, Osborn and
Peris \cite{Boito2011a} applied a duality model to include known \textsc{dv}
effects to the \textsc{qcd} analysis of \(\tau\) decays. The group around Boito
and Pich have different opinions on the importance of the newly introduced
duality model \cite{Pich2016,Boito2016,} and consequently we want to deliver a third,
opinion on the subject, favouring fits without the duality model. With new data
becoming available from \(e^+e^-\) annihilation the extraction of \(\alpha_s\)
has recently been extended to analyses up to \SI{2}{\giga\eV} \cite{Boito2018}.



\section{Fit Strategy}
The objective of this work is to extract \(\alhpa_s\) and argue about the
importance of \textsc{dv}. Apart from the two main objectives we want to analyse
the values of the \textsc{ope} Wilson coefficients of up to dimension ten.

Our fitting strategy will be in choosing weights of lower and higher pinching.
Lower pinched weights should be affected by \textsc{dv}, while higher pinched
weights should be protected from \textsc{dv}. As a result in comparing different
fits of lower and higher pinched weights it should be possible to argue about
strength of the \textsc{dv} that are (or are not) present.

Our hypothesis is that \textsc{dv} are tiny for fits of the combined vector and
axial\-/vector channel in combination with pinched weights. Consequently we can
extract parameters, like the strong coupling \(\alpha_s\) from \(\tau\) decays
to high precision without a \textsc{dv} model.

We will perform our analysis in the framework of \textsc{fopt}, but display our
final results also in \textsc{cipt}. Consequently to define a fit we have to
choose a weight \(\omega\) and a momentum \(s_0\). The only restriction from
choosing a weight is, that the weight has to be analytic, leaving us with a
variety of choices. For our strategy we have chosen three categories of weights,
each of them containing fits with three different weights. A table with an
overview of all used weights is given in \cref{table:fitCategories}
\begin{table}
  \centering
  \begin{tabular}{ccccc}
    & Symbol & Term & Expansion & \textsc{ope} Contributions \\
    \toprule
    \parbox[t]{2mm}{\multirow{3}{*}{\rotatebox[origin=c]{90}{\small Pinched}}} & \omega_\tau & \((1-x)^2(1+2x)\) & \(1 - 3x^2 + 2x^3\) & \(D6, D8\) \\
    & \omega_{cube} & \((1-x)^3(1+3x)\) & \(1 - 6x^2 + 8x^3 - 3x^4\) & \(D6, D8, D10\) \\
    & \omega_{quartic} & \((1-x)^4(1+3x)\) & \(1 - 10x^2 + 20x^3 - 15x^4 + 4x^5\) & \(D6, D8, D10, D12\) \\
    \midrule
    \parbox[t]{2mm}{\multirow{3}{*}{\rotatebox[origin=c]{90}{\small Monomial}}} & \omega_{M2} & \(1 - x^2\) & \(1-x^2\) & \(D6\) \\
    & \omega_{M3} & \(1 - x^3\) & \(1 - x^3\) & \(D8\) \\
    & \omega_{M4} & \(1 - x^4\) & \(1 - x^4\) & \(D10\) \\
    \midrule
    \parbox[t]{2mm}{\multirow{3}{*}{\rotatebox[origin=c]{90}{\small Pinched \(+ x\)}}} & \omega_{X2} & \((1 - x)^2\) & \(1 - 2x + x^2\) & \(D4, D6\) \\
    & \omega_{X3} & \((1 - x)^3\) & \(1 - 3x + 3x^2 - x^3\) & \(D4, D6, D8\) \\
    & \omega_{X4} & \((1 - x)^4\) & \(1 - 4x + 6x^2 - 4x^3 + x^4\) & \(D4, D6, D8, D10\) \\
    \bototmline
  \end{tabular}
  \caption{Displaying three categories of fits, each containing three weights
    with their corresponding mathematical expression and the \textsc{ope}
    contributions the fitted integral momentum will be sensitive to.}
  \label{table:fitCategories}
\end{table}
To test for the stability of the fitted values and have enough \textsc{dof} to
fit the higher \textsc{ope} contributions we furthermore fit every weight
for various momenta \(s_0\).



\section{Fits}
In the following we will give the results of each of the three previously
mentioned fit categories.

The first category contains the \textit{Pinched Weights without Monomial x}. The
chosen weights are double (\(\omega_\tau\)), triple (\(\omega_{cube}\))and
quadruple (\(\omega_{quartic}\)) pinched and do not contain a monomial term
\(x\). An \(x\) term would make the fits sensitive to the \(D=4\) \textsc{ope}
contribution and have a unreliable perturbative expansion \cite{Beneke2012}. The
higher the pinching, the higher the suppression of \textsc{dv}. Consequently if
we obtain stable values for \(\alpha_s\) from the different pinched fits we
should expect the \textsc{dv} to have no influence on the value of the strong
coupling. The different weights imply an increasing number of active
\textsc{ope} contributions \(D_6, D_8, D_{10}\) and \(D_{12}\), which can be
used to compare to the stability of higher order \textsc{ope} contributions and
to test for the convergence of the \textsc{ope}.

The second category contains the \textit{Single Pinched Monomial weights}. In
this case all of the weights are only single pinched and, as in the first category, do not carry a
monomial in \(x\). Consequently if \textsc{dv} affect the fits we should notice
different fitting results in comparison to the fits of the first category.
Furthermore the single pinched moments only carry two parameters, the strong
coupling and an \textsc{ope} Wilson coefficients. Thus we can further compare
the \(C_6, C_8\) and \(C_{10}\) Wilson coefficients and argue about the
stability of the fits.

The third and last category contains a similar pinching to the as compared to
the first category, but this time contains a monomial term in \(x\).
Consequently these fits are unreliable in the framework of \textsc{fopt} and we
have to apply the \define{bs}{Borel sum}. Following the logic of the second and
first category we then can compare the result to analyse the role of \textsc{dv}
compare the Wilson coefficients. 


\subsection{Pinched Weights without Monomial \(x\)}

\subsubsection{Kinematic weight: \(\omega_{\tau}(x) \equiv (1-x)^2(1+2x)\)}
We previously encountered the kinematic weight in \cref{eq:kinematicWeight}. It
is a polynomial weight function, defined as \(\omega_\tau(x) = (1-x)^2(1+2x)\),
double pinched, contains the unity and does not contain a term proportional to
\(x\). Consequently it is an optimal weight \cite{Beneke2012}. As a doubled
pinched weight it should have a good suppression of \textsc{dv} contributions
and its polynomial contains terms proportional to \(x^2\) and \(x^3\), which
makes it sensitive to the dimension six and eight \textsc{ope} contributions.
The fits have been performed within the framework of \textsc{fopt} for different
numbers of \(s_0\). The momentum sets are characterised by its lowest energy
\(s_{min}\). We fitted values down to \SI{1.5}{\giga\eV}. Going to lower
energies is questionable due to the coupling constant becoming large, which
implies a breakdown of \textsc{pt}. Furthermore it bares the risk to be affected
by the \(\rho(770)\) and \(a_1\) peaks in the vector and axial\-/vector spectral
function, which we cannot model within the framework of the \textsc{ope}. For
the three fitting parameters \(\alpha_s, C_6\) and \(C_8\) we have given the
results in \cref{table:fitWKinAlD6D8} and graphically in
\cref{fig:fitWKinAlD6D8}.
\begin{table}
  \centering
  \begin{tabular}{llllll}
    \toprule \\
    \(s_{min}\) & \#\(s_0\)s & \(\alpha_s(m_\tau^2)\) & \(c_6\) & \(c_8\) & \(\chi^2/dof\)  \\
    \hline \\
    % 1.500 & 23 & 0.3255(13) & -0.441(10) & -0.2909(34) & 2.00 \\
    % 1.525 & 22 & 0.3255(18) & -0.440(36) & -0.288(45) & 2.10 \\
    % 1.550 & 21 & 0.3265(16) & -0.478(36) & -0.343(50) & 1.81 \\
    % 1.575 & 20 & 0.3269(22) & -0.493(47) & -0.365(58) & 1.86 \\
    % 1.600 & 19 & 0.3272(23) & -0.506(51) & -0.384(64) & 1.94 \\
    % 1.625 & 18 & 0.3284(24) & -0.540(53) & -0.433(68) & 1.788 \\
    % 1.650 & 17 & 0.3283(24) & -0.550(57) & -0.448(74) & 1.90 \\
    % 1.675 & 16 & 0.3284(24) & -0.549(57) & -0.448(79) & 2.04 \\
    % 1.700 & 15 & 0.3281(24) & -0.538(63) & -0.430(87) & 2.19 \\
    % 1.750 & 14 & 0.3291(26) & -0.581(71) & -0.50(10) & 2.21 \\
    % 1.800 & 13 & 0.3293(27) & -0.589(77) & -0.51(11) & 2.43 \\
    % 1.850 & 12 & 0.3281(28) & -0.537(85) & -0.42(13) & 2.5 \\
    % 1.900 & 11 & 0.3272(29) & -0.493(93) & -0.35(15) & 2.65 \\
    % 1.950 & 10 & 0.3232(32) & -0.31(11) & -0.01(18) & 1.13 \\
    % 2.000 & 9 & 0.3234(34) & -0.32(12) & -0.03(21) & 1.31 \\
    2.100 & 8 & 0.3256(38) & -0.43(15) & -0.25(28) & 1.30 \\
    \rowcolor{primary}
    2.200 & 7 & 0.3308(44) & -0.72(20) & -0.85(38) & 0.19 \\
    \rowcolor{primary}
    2.300 & 6 & 0.3304(52) & -0.69(25) & -0.80(50) & 0.25 \\
    \rowcolor{primary}
    2.400 & 5 & 0.3339(70) & -0.91(39) & -1.29(83) & 0.10 \\
    2.600 & 4 & 0.3398(15) & -1.3(1.0) & -2.3(2.5) & 0.01  \\
    \bottomrule
  \end{tabular}
  \caption{Table of our fitting values of \(\alpha_s(m_\tau^2), C_6\) and
    \(C_8\) for the kinematic weight \(\omega(x)=(1-x)^2(1+2x)\) using
    \textsc{fopt} ordered by increasing \(s_{min}\). The errors are given in
    parenthesis after the observed value.}
  \label{table:fitWKinAlD6D8}
\end{table}
\begin{figure}
  \centering \includegraphics[width=\textwidth]{./images/fitWKinAlD6D8.eps}
  \caption{Fitting values of \(\alpha_s(m_\tau^2), C_6\) and \(C_8\) for the
    kinematic weight \(\omega(x)=(1-x)^2(1+2x)\) using \textsc{fopt} for
    different \(s_{min}\). The left graph plots \(\alpha_s(m_\tau^2)\) for
    different numbers of used \(s_0\)s. The right plot contains the dimension
    six and eight contributions to the \textsc{ope}. Both plots have in grey the
    \(\chi^2\) per \textsc{dof}.}
  \label{fig:fitWKinAlD6D8}
\end{figure}

We only display the fits for \(s_{min}\) larger than \SI{2.1}{\giga\eV}. We
noted a jump between the \(s_{min}=\SI{2.1}{\giga\eV}\) and
\(s_{min}=\SI{2.2}{\giga\eV}\) of the \(\chi^2/dof\) from \(0.19\) to \(1.3\).
We consequently discarded fits with a \(s_{min}<\SI{2.2}{\giga\eV}\), as fits
lower \(s_{min}\) behave more stable\footnote{As will be seen by comparing the
  kinematic weight with the cubic and quartic weight}. The values for the less
momenta are preferred by us due to two reasons. First below energies of
\SI{2.2}{\giga\eV} we have to face the problematic influence of increasing
resonances. Second, we will see, that the values obtained from the lower moment
fits are more compatible with our other fits series. We further discarded the
fit with four \(s_0\)s momenta, which has very small \(\chi^2/dof=0.01\). This
is due to the fact, that we have four \(s_0\)s momenta to fit three parameters,
which leaves us with too few \textsc{dof}.

The selected fits with \(8\-/10\) momenta have a small \(\chi^2\) per
\textsc{dof}. The fitted parameters, \(\alpha_s, c_6\) and \(c_8\) are in good
agreement with each other. For all fits we have a good convergence of the
\textsc{ope}. For later comparisons we will give the means for the strong
coupling, \(D=6\) and \(D=8\) contributions:
\begin{equation}
  \label{eq:wKinResult}
  \alpha_s(m_\tau^2) = 0.3317(33), \quad c_6 = -0.77(17) \quad \text{and} \quad
  c_8 = -0.98(35).
\end{equation}

We further tested the stability of the dimension six and eight contributions to
the \textsc{ope} within the same fit series but for a fixed value of the strong coupling
to our previous averaged result \(\alpha_s(m_\tau^2)=0.3179\). The values for
\(c_6\) and \(c_8\) are larger than the values given in our final results from
\cref{table:fitWKinAlD6D8}. This is explained with a smaller contribution from
the strong coupling, which has to be compensated by larger \textsc{ope} contributions.


\subsection{Cubic weight: \(\omega_{cube}(x) \equiv (1-x)^3(1+3x)\)}
\label{sec:cubicWeight}
To further consolidate the results from the kinematic weight, we test a weight
of higher pinching, which is should suppress \textsc{dv} more than a double
pinched weight. Consequently, if we obtain similar results to our previous fits
we could exclude \textsc{dv} effects for the kinematic weight. On the other
hand, any differences to the previous fit would indicate present \textsc{dv}.
Our \textit{cubic} weight will be triple pinched and optimal, as it does not
contain a \(x\) monomial. It is due to its polynomial structure sensitive to the
dimensions six, eight and ten contributions of the \textsc{ope}, which yields
one more parameter to fit than with the kinematic weight \(\omega_\tau\). The
fitting results can be seen in \cref{table:fitWCubicAlD6D8D10} and graphically
in \cref{fig:fitWCubeAlpha}.
\begin{table}
  \centering
  \begin{tabular}{lllllll}
    \toprule \\
    \(s_{min}\) & \#\(s_0\)s & \(\alpha_s(m_\tau^2)\) & \(C_6\) & \(C_8\) & \(C_{10}\) & \(\chi^2/dof\)  \\
    \hline \\
    % 1.800 & 13 & 0.3305(37) & -0.493(76) & -0.48(12) & -0.66(20) & 2.99 \\
    % 1.850 & 12 & 0.3303(37) & -0.482(68) & -0.456(100) & -0.62(17) & 3.35 \\
    % 1.900 & 11 & 0.3249(29) & -0.280(20) & -0.088(21) & 0.088(55) & 1.58 \\
    % 1.950 & 10 & 0.3237(26) & -0.232(25) & 0.005(42) & 0.275(93) & 1.67 \\
    2.000 & 9 & 0.3228(26) & -0.196(27) & 0.075(28) & 0.420(56) & 1.96 \\
    \rowcolor{primary}
    2.100 & 8 & 0.3302(40) & -0.52(11) & -0.58(22) & -1.00(45) & 0.43 \\
    \rowcolor{primary}
    2.200 & 7 & 0.3312(43) & -0.56(12) & -0.68(23) & -1.23(50) & 0.55 \\
    \rowcolor{primary}
    2.300 & 6 & 0.336(11) & -0.78(47) & -1.17(98) & -2.38(22) & 0.29 \\
    \rowcolor{primary}
    2.400 & 5 & 0.3330(96) & -0.63(47) & -0.82(10) & -1.51(26) & 0.48 \\
    \bottomrule
  \end{tabular}
  \caption{Table of our fitting values of \(\alpha_s(m_\tau^2), C_6, C_8\) and
    \(C_{10}\) for the cubic weight \(\omega(x)=(1-x)^3(1+3x)\) using
    \textsc{fopt} ordered by increasing \(s_{min}\). The errors are given in
    parenthesis after the observed value.}
  \label{table:fitWCubicAlD6D8D10}
\end{table}
\begin{figure}
  \centering \includegraphics[width=\textwidth]{./images/fitWCubeAlD6D8D10.eps}
  \label{fig:fitWCubeAlpha}
\end{figure}

As before we performed fits for \(s_0 \leq \SI{1.5}{\giga\eV}\), but could only
reach convergence for fits with energies larger or equal than
\SI{1.8}{\giga\eV}. As before the \(\chi^2\) makes a jump at
\(s_0=\SI{2.1}{\giga\eV}\) to values per \textsc{dof} of almost \(2\).
Consequently we excluded theses fits and focused on fits from five to eight
\(s_0\)s momenta.

The selected fits have a good \(\chi^2/dof\) and the fitted parameters,
\(\alpha_s, c_6, c_8\) and \(c_{10}\) are in agreement with each other, except
for the fit with six momenta. The fit with a \(s_{min}=\SI{2.3}{\giga\eV}\) has
the lowest \(\chi^2=0.29\) and error on \(\alhpa_s\), but takes slighlty
different values for the \textsc{ope} Wilson coefficients in comparison to the
other selected fits. The means for the strong coupling, \(D=6\) and \(D=8\)
contributions:
\begin{equation}
  \label{eq:wKinResult}
  \alpha_s(m_\tau^2)=0.33, C_6=-0.62, C_8=-0.81) \quad \text{quad} \quad C_{10}=-1.5.
\end{equation}
We furthermore found that the \textsc{ope} is converging, but not as fast as for
the kinematic weight. The values of \(\abs{\delta^{(8)}}\) is only half as large
as \(\abs{\delta^{(8)}}\). The values of the lower momentum count are in high
agreement with the ones obtained from the kinematic weight. The conclusions that
we take from the \textit{cubic weight} are that the kinematic weight, with its
double pinching, should sufficiently suppress any contributions from
\textsc{DV}s. If \textsc{DV} would have an effect on the kinematic weight, we
should have seen an improvement of the fits with the \textit{cubic} weight, due
to its triple pinching, which is not the case.


\subsection{Quartic weight: \($\omega(x) \equiv (1-x)^4(1+4x)$\)}
\label{sec:quarticWeight}
To include an even higher pinching of four and to compare the previously
obtained value for the dimension ten \textsc{OPE} contribution we performed fits
with the \textit{quartic weight} defined as \($\omega(x) \equiv (1-x)^4(1+4x)$\),
which also fulfils the definition of an optimal weight \cite{Beneke2012}. 
Unfortunately the fits only converged for \($s_{min}=\SI{2}{\giga\eV}$ (nine $s_0$\)s
moment combination). The results for , with a \($\chi^2$ per \textsc{dof} of $0.67$\) are given by:
\begin{equation}
  \begin{split}
    \alpha_s(m_\tau^2) &= 0.3290(11), \quad c_6=-0.3030(46), \quad c_8=-0.1874(28), \\
    c_{10} &= 0.3678(45) \quad \text{and} \quad c_{12}=-0.4071(77)
  \end{split}
\end{equation}
Due to the problematic of the fitting routing, which is caused by too many
\textsc{ope} contributions fitted simultaneously, we will discard the fitting
results for the quartic weight.

\subsection{Single Pinched Monomial Weights}
\subsubsection{\(1-x^2\)}
\subsubsection{\(1-x^3\)}
\subsubsection{\(1-x^4\)}

\subsection{Pinched Weights with monomial \(x\)}
\subsubsection{\((1-x)^2\)}
\subsubsection{\((1-x)^3\)}
\subsubsection{\((1-x)^4\)}



\section{Results}



In the following we will perform fits to determine \($\alpha_s$\) at the
\($m_\tau^2$\)\-/scale. The fits are separated corresponding to the used weight. Every
weight contains multiple fits for different \($s_0$\)\-/momenta. We will start with
the kinematic weight, which appears naturally in the inclusive \($\tau$\)\-/decay
ratio \cref{eq:inclusiveRatio} and has the best fitting characteristics of all
weights we have used.




\subsection{Third power monomial: \($\omega_{m3}(x) \equiv 1-x^3$\)}
To study the behaviour of the \textsc{DV} and the higher order \textsc{OPE}
contributions of dimension eight and ten we further included two optimal, single pinched weights.
The first one is defined as \($\omega_{m3}(x)\equiv 1-x^3$\) and contains a
single third power monomial and is consequently sensitive to dimension eight
contributions from the \textsc{OPE}. Our fitting results can be taken from
\cref{table:fitWM3AlD8}.
\begin{table}
  \centering
  \begin{tabular}{lllll}
    \toprule \\
    \($s_{min}$ & \#$s_0$s & $\alpha_s(m_\tau^2)$ & $c_8$ &  $\chi^2/dof$\)  \\
    \hline \\
    % 1.500 & 23 & 0.3160(28) & -0.523(65) & 2.4 \\
    % 1.525 & 22 & 0.3171(28) & -0.578(70) & 2.3 \\
    % 1.550 & 21 & 0.3173(29) & -0.587(76) & 2.42 \\
    % 1.575 & 20 & 0.3187(29) & -0.667(82) & 2.08 \\
    % 1.600 & 19 & 0.3189(30) & -0.679(87) & 2.19 \\
    % 1.625 & 18 & 0.3195(30) & -0.719(94) & 2.24 \\
    % 1.650 & 17 & 0.3205(30) & -0.783(99) & 2.1 \\
    % 1.675 & 16 & 0.3204(31) & -0.77(11) & 2.24 \\
    % 1.700 & 15 & 0.3206(31) & -0.79(11) & 2.39 \\
    % 1.750 & 14 & 0.3202(32) & -0.76(13) & 2.57 \\
    % 1.800 & 13 & 0.3217(33) & -0.88(14) & 2.41 \\
    % 1.850 & 12 & 0.3202(35) & -0.75(16) & 2.4 \\
    % 1.900 & 11 & 0.3202(36) & -0.75(18) & 2.67 \\
    % 1.950 & 10 & 0.3161(38) & -0.40(20) & 1.46 \\
    % 2.000 & 9  & 0.3148(39) & -0.28(22) & 1.47 \\
    % 2.100 & 8  & 0.3147(44) & -0.27(29) & 1.71 \\
    2.200 & 7  & 0.3214(49) & -1.01(39) & 0.41 \\
    2.300 & 6  & 0.3227(57) & -1.18(54) & 0.46 \\
    2.400 & 5  & 0.3257(67) & -1.58(74) & 0.39 \\
    2.600 & 4  & 0.325(10) & -1.54(1.53) & 0.58 \\
    2.800 & 3  & 0.326(21) & -1.69(4.03) & 1.17 \\
    \bottomrule
  \end{tabular}
  \caption{Table of our fitting values of \($\alpha_s(m_\tau^2)$\), and
    \($c_{8}$ for the single pinched third power monomial weight $\omega(x)=1-x^3$\) using FOPT ordered
    by increasing \($s_{min}$\). The errors are given in parenthesis after the observed value.}
  \label{table:fitWM3AlD8}
\end{table}
The \($\chi^2$ per \textsc{dof} is like in the $\omega_\tau$ and $\omega_{cubic}$\)
fits good for \($s_{min}\leq \SI{2.2}{\giga\eV}$\), but jumps to values
\($\chi^2/dof>1.4$ for smaller $s_{min}$\). This is, as before, explained through
resonances that appear in lower energies. Due to the good \($\chi^2$\) and the
internally compatible fitting values we averaged over all rows except the last
one of \cref{table:fitWM3AlD8}. The last row, at \($s_{min}=\SI{2.8}{\giga\eV}$\)
has only one \textsc{dof} and thus high errors. The averaged values are
thus given by 
\begin{equation}
  \alpha(m_\tau^2) = 0.32382(42) \qquad \text{and} \qquad c_8=-1.33(67).
\end{equation}
We note that the strong coupling is smaller as our expected values from the
kinematic weight \cref{eq:wKinResults}, but the dimension eight contribution
is in good agreement. The strong coupling from the monomial weight to third
order seems to be in better agreement with the 8\-/10 momenta used in the
kinematic fits, whereas the dimension eight contributions agrees more with the
4\-/7 momenta fits.

We have made use of a single pinched weight and discovered that the fitting
result is not completely compatible with our previous fitting results.
Consequently weights with a pinching less than two are affected by \textsc{DV}
and should not be used to determine the strong coupling.

\subsection{Fourth power monomial: \(\omega_{m4}(x) \equiv
  1-x^4\)}
We already analysed the cubic and quartic weights, which depend on the
dimension ten \textsc{ope} contribution, in \cref{sec:cubicWeight}
and \cref{sec:quarticWeight} correspondingly. Now, even with the visible
\textsc{dv} for
fourth power monomial \(\omega_{m4}\equiv 1-x^4\) to study another single pinched moment and the dimension
ten \textsc{ope} contribution. The results of the are given in \cref{table:wFitM4AlD10}. 
\begin{table}
  \centering
  \begin{tabular}{lllll}
    \toprule \\
    \($s_{min}$ & \#$s_0$s & $\alpha_s(m_\tau^2)$ & $c_{10}$ & $\chi^2/dof$\)  \\
    \hline \\
    % 1.500 & 23 & 0.3144(27) & -0.572(80) & 2.44 \\
    % 1.525 & 22 & 0.3155(27) & -0.655(90) & 2.34 \\
    % 1.550 & 21 & 0.3157(28) & -0.671(99) & 2.45 \\
    % 1.575 & 20 & 0.3171(28) & -0.80(11) & 2.1 \\
    % 1.600 & 19 & 0.3173(29) & -0.82(12) & 2.21 \\
    % 1.625 & 18 & 0.3180(29) & -0.88(13) & 2.24 \\
    % 1.650 & 17 & 0.3190(30) & -0.98(14) & 2.1 \\
    % 1.675 & 16 & 0.3189(30) & -0.97(15) & 2.24 \\
    % 1.700 & 15 & 0.3192(30) & -1.00(16) & 2.39 \\
    % 1.750 & 14 & 0.3188(32) & -0.96(19) & 2.58 \\
    % 1.800 & 13 & 0.3204(32) & -1.17(21) & 2.39 \\
    % 1.850 & 12 & 0.3190(34) & -0.95(26) & 2.4 \\
    % 1.900 & 11 & 0.3189(35) & -0.94(29) & 2.67 \\
    % 1.950 & 10 & 0.3149(37) & -0.31(34) & 1.47 \\
    % 2.000 & 9  & 0.3137(39) & -0.08(39) & 1.5 \\
    % 2.100 & 8  & 0.3136(43) & -0.07(54) & 1.75 \\
    2.200 & 7  & 0.3203(48) & -1.64(77) & 0.42 \\
    2.300 & 6  & 0.3216(56) & -2.01(1.13) & 0.47 \\
    2.400 & 5  & 0.3247(66) & -2.98(1.62) & 0.39 \\
    2.600 & 4  & 0.324(10) & -2.86(3.69) & 0.58 \\
    2.800 & 3  & 0.325(20) & -3.43(10.74) & 1.17 \\
    \bottomrule
  \end{tabular}
  \caption{Table of our fitting values of \($\alpha_s(m_\tau^2)$ and $c_{10}$\)
    for the single pinched fourth power monomial weight \($\omega(x)=1-x^4$\) using FOPT ordered
    by increasing \($s_{min}$\). The errors are given in parenthesis after the observed value.}
  \label{table:fitM4AlD10}
\end{table}
The fitting behaviour is very similar to the third power monomial
(\cref{table:wFitM3AlD8}) and we will directly cite our obtained results:
\begin{equation}
  \alpha_s(m_\tau^2) = 0.32277(40) \qquad \text{and} \qquad c_{10}=-2.4(3.6).
\end{equation}
As before the values for the strong coupling are lower than the ones obtained by
the kinematic weight fit. Furthermore the error on the tenth dimension
contribution of the \textsc{ope} are too huge, although the huge errors makes it
compatible with all previous results. All in all the usage of the single pinched
fourth power monomial weight is questionable and does not deliver any additional insights.

\subsection{Pich's Optimal Moments \cite{Pich2016}}
Next to the previously mentioned \textit{optimal weights} from Beneke and Jamin
\cite{Beneke2012} there are \textit{optimal moments} introduced by Pich
\cite{LeDiberder1992}. Combinations of these optimal moments have been widely
used by the \textsc{aleph} collaboration to perform QCD analysis on the Large electron-positron
collider (\textsc{lep}). These moments include the for \textsc{FOPT} problematic
proportional term in x \cite{Beneke2012}, thus we will perform additional fits
in the Borel-sum.

\begin{equation}
  \omega_{(n,m)}(x) = (1-x)^n\sum_{k=0}^m (k+1)x^k 
\end{equation}
\subsubsection{\($\omega(x) = (1-x)^2$\)}
\begin{table}[H]
  \centering
  \begin{tabular}{llllll}
    \toprule \\
    \($s_{min}$ & \#$s_0$s & $\alpha_s(m_\tau^2)$ & $aGGInv$ & $c_{6}$ & $\chi^2/dof$\)  \\
    \hline \\
    % 1.500 & 23 & 0.3276(13) & -0.0077(10) & 0.330(35) & 2.62 \\
    % 1.525 & 22 & 0.3278(14) & -0.0078(10) & 0.330(38) & 2.75 \\
    % 1.550 & 21 & 0.3299(16) & -0.0092(12) & 0.333(37) & 2.31 \\
    % 1.575 & 20 & 0.3308(25) & -0.0098(13) & 0.334(47) & 2.32 \\
    % 1.600 & 19 & 0.3317(28) & -0.0105(14) & 0.335(54) & 2.38 \\

    % 1.650 & 17 & 0.3345(34) & -0.0124(17) & 0.342(62) & 2.15 \\
    % 1.675 & 16 & 0.3349(25) & -0.0127(15) & 0.342(51) & 2.28 \\
    % 1.700 & 15 & 0.3348(33) & -0.0126(18) & 0.342(58) & 2.47 \\
    % 1.750 & 14 & 0.3372(43) & -0.0145(23) & 0.341(71) & 2.34 \\
    % 1.800 & 13 & 0.3378(31) & -0.0149(20) & 0.339(58) & 2.54 \\
    % 1.850 & 12 & 0.3365(38) & -0.0138(25) & 0.346(60) & 2.72 \\
    % 1.900 & 11 & 0.3355(40) & -0.0128(28) & 0.354(59) & 2.97 \\
    % 1.950 & 10 & 0.3296(47) & -0.0073(34) & 0.418(58) & 1.57 \\
    % 2.000 & 9  & 0.3299(50) & -0.0076(39) & 0.414(64) & 1.83 \\
    % 2.100 & 8  & 0.3331(54) & -0.0108(45) & 0.361(76) & 1.9 \\
    2.200 & 7  & 0.3401(57) & -0.0185(52) & 0.220(88) & 0.73 \\
    2.300 & 6  & 0.3383(68) & -0.0165(67) & 0.26(12) & 0.89 \\
    2.400 & 5  & 0.3450(93) & -0.0243(99) & 0.10(17) & 0.71 \\
    2.600 & 4  & 0.337(16) & -0.014(18) & 0.36(45) & 0.98 \\
    \bottomrule
  \end{tabular}
  \caption{Table of our fitting values of \($\alpha_s(m_\tau^2), aGGInv$ and $c_{6}$\)
    for the triple pinched optimal weight \($\omega^{(2,0)}(x)=(1-x)^2$\) using FOPT ordered
    by increasing \($s_{min}$\). The errors are given in parenthesis after the observed value.}
  \label{table:fitOpt30AlD4D6}
\end{table}

\subsubsection{\($\omega(x) = (1-x)^3$\)}
\begin{table}[H]
  \centering
  \begin{tabular}{lllllll}
    \toprule \\
    \($s_{min}$ & \#$s_0$s & $\alpha_s(m_\tau^2)$ & $aGGInv$ & $c_{6}$ & $c_{8}$ & $\chi^2/dof$\)  \\
    \hline \\
    1.900 & 11 & 0.34281(92) & -0.01473(73) & -0.103(22) & -0.534(46) & 1.52 \\
    1.950 & 10 & 0.34154(99) & -0.01304(61) & -0.050(17) & -0.389(44) & 1.42 \\
    2.000 & 9  & 0.33985(81) & -0.01124(43) & 0.002(10) & -0.242(26) & 1.59 \\
    2.100 & 8  & 0.3480(47) & -0.0201(36) & -0.264(89) & -1.03(28) & 0.31 \\
    2.200 & 7  & 0.3483(23) & -0.0204(41) & -0.27(15) & -1.05(40) & 0.41 \\
    2.300 & 6  & 0.3522(64) & -0.0249(62) & -0.42(18) & -1.51(57) & 0.29 \\
    2.400 & 5  & 0.3480(89) & -0.0199(100) & -0.25(33) & -0.96(10) & 0.39 \\
    \bottomrule
  \end{tabular}
  \caption{Table of our fitting values of \($\alpha_s(m_\tau^2), aGGInv, c_6$ and $c_{8}$\)
    for the optimal weight \($\omega^{(3,0)}(x)=(1-x)^3$\) using FOPT ordered
    by increasing \($s_{min}$\). The errors are given in parenthesis after the observed value.}
  \label{table:fitOpt30AlD4D6D8}
\end{table}

\newpage
\subsection{Comparison}
To create an overview of our previous results we have gathered the most
compatible rows by hand. These are shown in \cref{table:fitCombinations}, which
is composed of two parts:
\begin{itemize}
\item The upper three rows represent fits we found to have good properties for
  determining the strong coupling.
\item The lower five rows are problematic fits due to too many OPE
  contributions, too low pinching or to terms proportional to \($x$\).
\end{itemize}
\begin{table}
  \centering \resizebox{\textwidth}{!}{
    \begin{tabular}{cccccccc}
      \toprule \\
      weight & \($s_{min}$ & $\alpha_s(m_\tau^2)$ & $aGGInv$ & $c_6$ & $c_8$ & $c_{10}$ & $\chi^2/dof$\)  \\
      \toprule \\
      \($\omega_{kin}$\) & 2.2 & 0.3308(44) & - & -0.72(20) & -0.85(38) & - & 0.19 \\
      \($\omega_{cube}$\) & 2.1 & 0.3302(40) & - & -0.52(11) & -0.58(22) & -1.00(45) & 0.43 \\ 
      \($\omega_{3,0}^*$\) & 2.1 & 0.3239(30) & -0.2125(26) & -0.627(87) & -0.74(17) & - & 0.46 \\
      \hline \\
      \($\omega_{quartic}$\) & 2.0 & 0.3290(11) & - & -0.3030(46) & -0.1874(28) & 0.3678(45) & 0.67 \\
      \($\omega_{m3}$\) & 2.2 & 0.3214(49) & - & - & -1.01(39) & - & 0.41 \\
      \($\omega_{m4}$\) & 2.2 & 0.3203(48) & - & - & - & -1.64(77) & 0.42 \\
      \($\omega_{2,0}$\) & 2.2 & 0.3401(57) & -0.0185(52) & 0.220(88) & - & - & 0.73\\
      \($\omega_{3,0}$\) & 2.1 & 0.3480(47) & -0.0201(36) & -0.264(89) & -1.03(28) & - & 0.31\\
      \bottomrule
    \end{tabular}
  }
  \caption{Table of the best fits (selected by \($\chi^2/dof$\) and compatibility of
    the fitting values) for each weight including at least the strong coupling
    \($\alpha_s(m_\tau^2)$\) as a fitting variable. All fits have been performed
    using \textsc{fopt}, except weights marked with a star \($\omega^*$\), which
    have been fitted using the \textit{Borel sum}.}
  \label{table:fitCombinations}
\end{table}
We have found that the kinematic weight is in excellent agreement with the cubic
\($\omega_{cube}$ and Pich's optimal weight $\omega_{3,0}$\), fitted using the borel
model. The fitted parameters from the kinematic weight (\($\alpha_s, c_6$\) and
\($c_8$\)) are all within error ranges and thus compatible. One fact that has to be
investigated is the negative appearing sign for the gluon\-/condensate from the
borel\-/sum of \(\omega_{3,0}\).

% \begin{figure}[H]
%   \includegraphics[width=\textwidth]{./images/comparisonAlpha.png}
%   \caption{Comparison of the strong coupling for the best fits (selected by
%   \($\chi^2/dof$\) closest to one) for each weight including the strong coupling
%   as a fitting variable.}
% \end{figure}
% \begin{figure}[H]
%   \includegraphics[width=\textwidth]{./images/comparisonD6D8.png}
%   \caption{Comparison of the dimension six and eight values for the best fits
%   (selected by \($\chi^2/dof$\) closest to one) for each weight including the OPE
%   dimensions six and eight as fitting variables.}
% \end{figure}
% \begin{figure}[H]
%   \includegraphics[width=\textwidth]{./images/comparisonD10.png}
%   \caption{Comparison of the dimension ten values for the best fits (selected
%   by \($\chi^2/dof$\) closest to one) for each weight including the OPE dimensions
%   ten as fitting variables.}
% \end{figure}


\end{document}