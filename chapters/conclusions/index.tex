\documentclass[../../index.tex]{subfiles}

\begin{document}
\chapter{Conclusions}
We have performed a \textsc{qcd} analysis on hadronic \(\tau\) decays to
determine a value of \(\alpha_s\) at the \(m_\tau^2\) scale without including
\textsc{dv}. We have excluded \textsc{dv} to contrast the previous analysis of
Boito et al., which stated the necessity of incorporating a model describing
\textsc{dv}. To argument we employed a new set of weights to probe the
suppression of \textsc{dv}.

The the strong coupling we obtained at the \(m_\tau^2\) scale from our fits is
\begin{equation}
  \alpha_s(m_\tau^2) = 0.3261(51).
\end{equation}
Running this value to the \(m_z\) scale yields
\begin{tcolorbox}[ams equation,myformula]
  \alpha_s(m_Z^2) = 0.1194(50),
\end{tcolorbox}
which is comparable to the world average value of \(\alpha_s(M_Z^2) =
0.1181(11)\) taken from the \cite{PDG2018}.

For the dimension six and eight \textsc{ope} contributions we extracted values of
\begin{align}
  C_6 &= -0.68 \pm 0.20\\
  C_8 &=  -0.80 \pm 0.38.
\end{align}

For \textsc{dv} we found that in the framework of \textsc{fopt} in the
\textsc{V+A} channel no additional model is needed for double pinched weights.
Even for single pinched weights we obtained stable results.

We also performed fits using the \textsc{bs}, yielding comparable results to the
values obtained from \textsc{fopt}. In the debate of \textsc{fopt} vs
\textsc{cipt} we interpret this outcome in favour of the former and discourage
the usage of \textsc{cipt}.





% With new data becoming available from \(e^+e^-\) annihilation the extraction
% of \(\alpha_s\) has recently been extended to analyses up to \SI{2}{\giga\eV}
% \cite{Boito2018}.



\end{document}
% LocalWords:  lllll AlD
