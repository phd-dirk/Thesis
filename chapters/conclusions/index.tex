\documentclass[../../index.tex]{subfiles}

\begin{document}
\chapter{Conclusions}
We have performed a \textsc{qcd} analysis on hadronic \(\tau\) decays to
determine a value of \(\alpha_s\) at the \(m_\tau^2\) scale without including
\textsc{dv}. We have excluded \textsc{dv} to contrast the previous analysis of
Boito et al. \cite{Boito2011a,Boito2012,Boito2014}, which stated the necessity of
incorporating a model describing \textsc{dv}. To argue, we employed a new set
of analytic weights to probe the suppression of \textsc{dv}.

We compared seven selected fits of different weights with single, double and
triple pinching. All fits gave similar values for the strong coupling and
dimension six and eight \textsc{ope} contributions. The conclusion we take is
that \textsc{dv} are sufficiently suppressed in the framework of \textsc{fopt}
in the \textsc{V+A} channel, even for single pinched weights. To extract precise
values of the strong coupling from hadronic \(\tau\) decay data no additional
model is needed.

For the kinematic weight and the weights carrying a monomial term \(x\), we
performed fits using both, the \textsc{bs} and the \textsc{fopt} approach. The
fitted parameters of both frameworks show great compatibility. The fact that
both frameworks yield similar results argues in favour of \textsc{fopt}, as
\textsc{cipt} would give different values for the fitting parameters.
Consequently, we discourage the usage of \textsc{cipt} and favour the usage
\textsc{fopt} and further underline the opinion of Beneke et al.
\cite{Beneke2008} in the debate of \textsc{fopt} vs \textsc{cipt}.

The final value for the strong coupling we obtained at the \(m_\tau^2\) scale is
given by
\begin{equation}
  \alpha_s(m_\tau^2) = 0.3261(50).
\end{equation}
Running this value to the \(m_Z\) scale yields
\begin{tcolorbox}[ams equation,myformula]
  \alpha_s(m_Z^2) = 0.11940(60),
\end{tcolorbox}
which is comparable to the world average value of \(\alpha_s(M_Z^2) =
0.1181(11)\) taken from the \cite{PDG2018}. For dimension six and eight
\textsc{ope} contributions we extracted values of
\begin{align}
  \rho^{(6)} &= -0.68 \pm 0.20\\
  \rho^{(8)} &=  -0.80 \pm 0.38.
\end{align}

% With new data becoming available from \(e^+e^-\) annihilation the extraction
% of \(\alpha_s\) has recently been extended to analyses up to \SI{2}{\giga\eV}
% \cite{Boito2018}.



\end{document}
% LocalWords:  lllll AlD
