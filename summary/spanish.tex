\documentclass[11pt,a4paper]{article}

\begin{document}
\section*{Resumen}
En este trabajo realizamos un análisis de cromodinámica cuántica (QCD) en
desintegraciónes de taus en hadrones. QCD está describiendo la fuerza fuerte,
que dicta las interacciones de quarks. Los quarks son partículas elementales.
P.ej. un protón es una acumulación de tres quarks. La fuerza de las
interacciones está dada por la constante fuerte, que al contrario que indica su
nombre depende de la energía. En esta tesis medimos la constante fuerte de
desintegraciónes de tau en hadrons. El tau es el único leptón que tiene la masa
necesaria como para desintegrarse en hadrones. Sin embargo, nos permite medir la
constante fuerte para bajas energías. Como la constante fuerte disminuye para
mayor energías sus errores diminuyen tambien. El marco de extracción la
constante fuerte es referido como reglas de suma de QCD (QCDSR). Dentro de las
QCDSR tenemos que elegir entre la teoría de perturbación de orden fijo (FOPT) o
teoría de perturbación con contorno mejorado (CIPT). Ambos métodos son
igualmente válidos, pero tienen a valores diferentes. Para comprobrar la validez
de FOPT hacemos uso de la suma de Borel (BS). La BS puede ser utilizada para dar
la mejor suma posible de series asintóticas y divergentes como la que tratamos
en extraer la constante fuerte. Para bajas energías la constante fuerte es
grande y para altas energías la constante fuerte es pequeño. Esto lleva al
confinamiento. Los quarks aparecen siempre como partículas compuestas, llamados
Hadrones. Hasta hoy Nunca hemos medido un quark aislado. Esto es problemático ya
que QCD es una teoría que describe los quarks, pero loos experimentos miden los
hadrones. Por este problematica se ha introducido la dualidad de quark-hadron,
que establece que podemos describir cantidades físicas, ya sea en la imagen de
quark-gluon o en la imagen de hadrons, y que ambas descripciones son igualmente
válidas. Desafortunadamente el supuesto de la dualidad es a menudo violada. En
teoría podemos suprimir estas violaciones de dualidad (DV) mediante la
aplicación de los llamados pesos apretados. Cuanto mayor sea el apreto más
suprimidos son las DV. Realizaremos  para diferentes pesos al estado, que
incluso para apretos bajos, los DV están suficientemente suprimidos para
extraer la constante fuerte.
\end{document}
