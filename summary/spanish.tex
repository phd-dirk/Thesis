\documentclass[11pt,a4paper]{article}

\begin{document}
\section*{Resumen}
En este trabajo realizamos un análisis de la cromodinámica cuántica (QCD) en
desintegraciones del tau en hadrones. QCD describe la fuerza fuerte, que dicta
las interacciones entre quarks. Los quarks son partículas elementales. Por
ejemplo, un protón está constituido de tres quarks. La fuerza de las
interacciones está determinada por la constante fuerte que, al contrario de lo
que indica su nombre, depende de la energía. En esta tesis medimos la constante
fuerte a partir de las desintegraciones del tau en hadrons. El tau es el único
leptón que tiene la masa necesaria para permitir una desintegración en hadrones.
Sin embargo, nos permite medir la constante fuerte para bajas energías. Como la
constante fuerte disminuye para energías mayores, sus errores también
disminuyen. QCD sum rules (QCDSR) es el marco teórico para extraer la constante
fuerte. Dentro de QCDSR tenemos que elegir entre la teoría de perturbación de
orden fijo (FOPT) o la teoría de perturbación con contorno mejorado (CIPT).
Ambos métodos son igualmente válidos, pero dan lugar a valores diferentes. Para
comprobar la validez de FOPT, usamos la suma de Borel (BS). Para bajas energías,
la constante fuerte es grande y para altas energías es pequeña. Esto lleva al
confinamiento. Los quarks aparecen siempre como partículas compuestas, llamados
hadrones. Hasta hoy nunca hemos medido un quark aislado. Esto es un problema, ya
que QCD es una teoría que describe los quarks, mientras que los experimentos
miden los hadrones. Para solucionar este problema, se ha introducido la dualidad
quark-hadrón, que establece que podemos describir cantidades físicas tanto en la
imagen teórica de quarks y gluones como en la imagen experimental de los
hadrones, y que ambas descripciones son igualmente válidas. Desafortunadamente
el supuesto de la dualidad es a menudo violada. En teoría podemos suprimir estas
violaciones de dualidad (DV) mediante la aplicación de los llamados pesos con
pinchos. Cuanto mayor sea el pincho dado por el peso, más suprimidos son las DV.
Realizaremos fits con diferentes pesos para demostrar que incluso con pinchos
bajos, los DV están suficientemente suprimidos para extraer el valor de la
constante fuerte.
\end{document}
