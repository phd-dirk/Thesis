\documentclass[12pt,a4paper,openright]{scrreprt}
\makeatletter\@twosidetrue\makeatother

\usepackage[left=1.5in,right=1in,top=1in,bottom=1in,includefoot,headheight=13.6pt]{geometry}
\usepackage[utf8]{inputenc}
\usepackage[T1]{fontenc}



% font - Palatino
\usepackage[sc,osf]{mathpazo}   % With old-style figures and real smallcaps.
\usepackage[euler-digits,small]{eulervm}


\usepackage{subfiles}
\usepackage{amsmath}
\usepackage{amssymb}
\usepackage{amsfonts}

\usepackage{graphicx}
\usepackage{csvsimple}
\usepackage{siunitx}
\usepackage{array}
\usepackage{booktabs} % \toprule, \midrule
\usepackage{multirow}
\usepackage{bm} % bold math
\usepackage{float} % [H] for floats
\usepackage{commath} % derivatives
\usepackage[dvipsnames,table]{xcolor}
\usepackage{hyperref}
\hypersetup{colorlinks}
\usepackage[nameinlink]{cleveref} % \ref{eq:1} => eq. 1
\usepackage{simpler-wick}
\usepackage{subcaption} % subfigures
\usepackage[shortcuts]{extdash}
\usepackage{bm} % bold math
\usepackage{colortbl} % \rowcolor in tables
\usepackage{adjustbox}
\usepackage{wrapfig}
\usepackage{nomencl}
\usepackage{slashed} % Feynman slash notation
\usepackage{tikz}
\usepackage[most]{tcolorbox}
\tcbset{myformula/.style={colback=black!10, %yellow!10!white,
    colframe=white, %red!50!black,
    top=0pt,bottom=3pt,left=0pt,right=0pt,
    boxsep=6pt,
    arc=0pt,
    outer arc=0pt,
  }}
\usepackage[compat=1.0.0]{tikz-feynman}
\usepackage{cancel} % cross out to zero in math
\usepackage{tabularx}
\newcolumntype{Y}{>{\centering\arraybackslash}X}
\usepackage{tabu}
\usepackage{caption}
\usepackage{afterpage}

% Headers
\usepackage[Lenny]{fncychap}
\usepackage{titlesec}
% \titleformat{\section}[block]{\Large\bfseries\filcenter}{\thesection}{1em}{}
% \titleformat{\subsection}[block]{\large\bfseries\filcenter}{\thesubsection}{1em}{}
\titleformat{\subsubsection}[block]{\bfseries\filcenter}{}{1em}{}


% hyperlink colours
\newcommand\myshade{85}
\colorlet{mylinkcolor}{violet}
\colorlet{mycitecolor}{YellowOrange}
\colorlet{myurlcolor}{Aquamarine}
\hypersetup{
  linkcolor  = mylinkcolor!\myshade!black,
  citecolor  = mycitecolor!\myshade!black,
  urlcolor   = myurlcolor!\myshade!black,
  colorlinks = true,
}


% Timeline https://tex.stackexchange.com/questions/196794/how-can-you-create-a-vertical-timeline
\newcommand{\foo}{\makebox[0pt]{\textbullet}\hskip-0.5pt\vrule width 1pt\hspace{\labelsep}}
\newcommand{\specialcell}[2][c]{% force line break
  \begin{tabular}[#1]{@{}c@{}}#2\end{tabular}}

% following https://codeinthehole.com/guides/writing-a-thesis-in-latex/
\usepackage{fancyhdr}
\pagestyle{fancy}
\rhead{}
\lhead{\nouppercase{\textsc{\leftmark}}}
\renewcommand{\headrulewidth}{0pt}
\fancypagestyle{plain}{%
  \fancyhf{}% clears all header and footer fields
  \fancyfoot[C]{\thepage}%
}

% page layout
\parindent 0pt
\parskip 1ex
\renewcommand{\baselinestretch}{1.33}


\numberwithin{equation}{section}       % Tinker with equation numbering
\renewcommand{\bibname}{References}    % Alter appearance of table of contents slightly
% \renewcommand{\contentsname}{Contents}

% nomenclature definitions
\newcommand{\define}[2]{\textit{#2} (\textsc{#1})\nomenclature{\textsc{#1}}{#2}}

\creflabelformat{equation}{#2\textup{#1}#3} % cref no parenthesis for equation number

\DeclareMathOperator{\sfm2}{sfm2}




% References
\usepackage[
  style=numeric
]{biblatex}
\addbibresource{references.bib}

% Nomenclature
\renewcommand{\nomname}{List of Abbreviations\label{ch:nomenclature}}
% \let\oldDif\dif
% \renewcommand{\dif}[1]{\oldDif{#1}\,}


\DeclareMathOperator{\Rea}{Re}
\DeclareMathOperator{\Ima}{Im}
\DeclareMathOperator{\Tr}{Tr}
\newcommand*\anti[1]{\overline{#1}\,}
\newcommand*\intsum{\int\!\!\!\!\!\!\sum}
\newcommand*\fdif[1]{\frac{\dif^{\,4} #1}{(2\pi)^4}}

% Colour scheme
\definecolor{primary}{rgb}{0.78, 0.89, 1}


% relative graphic paths
\makeatletter
\let\org@subfile\subfile
\renewcommand*{\subfile}[1]{%
  \filename@parse{#1}% LaTeX's file name parser
  \expandafter
  \graphicspath\expandafter{\expandafter{\filename@area}}%
  \org@subfile{#1}%
}
\makeatother

\makenomenclature
\makeindex


\begin{document}

\pagenumbering{gobble} \pdfbookmark[0]{Titlepage}{title}
\begin{titlepage}
  \begin{center}
    \includegraphics[height=2.4cm]{./images/UAB_physics_department.eps} \hfill
    \includegraphics[height=2.4cm]{./images/logo_IFAE.eps}
    \par

    \vspace{0.5cm} \Large A PhD Thesis in Physics
    \makebox[\linewidth]{\rule{\textwidth}{2pt}}
    \par
    \vspace{0.2cm} \Huge The \textsc{qcd} Strong Coupling from Hadronic \(\tau\)
    Decays \makebox[\linewidth]{\rule{\textwidth}{2pt}}

    \vspace{1.2cm} \Large Dirk Hornung

    \vspace{0.9cm} \normalsize
    \begin{tkizpicture}
      % Using the layered layout
      \feynmandiagram [scale=1.3,transform shape][layered layout, horizontal=a
      to b] { a [particle=\(\tau\)] -- [fermion] b -- [fermion] f1
        [particle=\(\nu_{\tau}\)], b -- [boson, edge label'=\(W^{-}\)] c, c --
        [anti fermion] f2 [particle=\(\anti{q}\)], c -- [fermion] f3
        [particle=\(q\)], };
    \end{tkizpicture}

    \Large July 2019

    \vfill \makebox[\linewidth]{\rule{\textwidth}{1pt}} \large
    \textsc{Supervisor}: Matthias Jamin

    \afterpage{\null\newpage}
  \end{center}
\end{titlepage}


\begin{center}
  \Large The \textsc{qcd} Strong Coupling from Hadronic \(\tau\) Decays

  \makebox[\linewidth]{\rule{\textwidth}{1pt}}

  \vspace{2cm}

  \includegraphics[height=2.4cm]{./images/logo_UAB.eps} \hfill
  \includegraphics[height=2.4cm]{./images/logo_IFAE.eps}

  \vspace{1cm}

  \includegraphics[height=2.4cm]{./images/severo_ochoa.eps} \hfill
  \includegraphics[height=2.4cm]{./images/bist.eps}

  \vspace{3cm}
\end{center}

\begin{large}
  I, \textbf{Matthias Rudolf Jamin} hereby certify that the dissertation with title
  \textit{The \textsc{qcd} Strong Coupling from Hadronic \(\tau\) Decays} was
  written by \textbf{Dirk Hornung} under my supervision in partial fulfilment
  of the doctoral degree in physics.

  \vfill Bellaterra, 18. June 2019 \par
  \vspace{3cm}
  Supervisor and tutor \hfill Author \hspace{1.75cm} \\
  \textbf{Dr Matthias Rudolf Jamin} \hfill \textbf{Dirk Hornung}
  \vspace{1cm}
\end{large}


  

\subfile{chapters/abstract/index}
\subfile{chapters/acknowledgements/index}
\setcounter{tocdepth}{1}
\tableofcontents
\subfile{chapters/notation/index}
\newpage
\pagenumbering{arabic}
\subfile{chapters/introduction/index}
\subfile{chapters/qcdSumRules/index}
\subfile{chapters/tauDecaysIntoHadrons/index}
\subfile{chapters/measuringTheStrongCoupling/index}
\subfile{chapters/conclusions/index}
\appendix
\subfile{chapters/appendix/index}
\printnomenclature
\printbibliography[heading=bibintoc]

\end{document}
% LocalWords:  Titlepage LaTeX's rgb tocdepth arabic Im sfm cref colorlinks
% LocalWords:  myurlcolor urlcolor mycitecolor citecolor mylinkcolor linkcolor
% LocalWords:  YellowOrange math boxsep colframe colback myformula subfigures
% LocalWords:  smallcaps Palatino qcdSumRules tauDecaysIntoHadrons bibintoc
% LocalWords:  measuringTheStrongCoupling LocalWords
