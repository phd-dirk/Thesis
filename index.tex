\documentclass[12pt,a4paper]{report}

\usepackage[left=1.5in,right=1in,top=1in,bottom=1in,includefoot,headheight=13.6pt]{geometry}
\usepackage[utf8]{inputenc}
\usepackage[T1]{fontenc}


% font - Palatino
\usepackage[sc,osf]{mathpazo}   % With old-style figures and real smallcaps.
\usepackage[euler-digits,small]{eulervm}

\usepackage{subfiles}
\usepackage{amsmath}
\usepackage{amssymb}
\usepackage{amsfonts}

\usepackage{graphicx}
\usepackage{csvsimple}
\usepackage{siunitx}
\usepackage{array}
\usepackage{booktabs} % \toprule, \midrule
\usepackage{multirow}
\usepackage{bm} % bold math
\usepackage{float} % [H] for floats
\usepackage{commath} % derivatives
\usepackage[x11names,table]{xcolor}
\usepackage{hyperref}
\hypersetup{colorlinks,linkcolor=[rgb]{0.2,0.6,0.86}}
\usepackage[nameinlink]{cleveref} % \ref{eq:1} => eq. 1
\usepackage{simpler-wick}
\usepackage{subcaption} % subfigures
\usepackage[shortcuts]{extdash}
\usepackage{bm} % bold math
\usepackage{colortbl} % \rowcolor in tables
\usepackage{adjustbox}
\usepackage{wrapfig}
\usepackage[refpage]{nomencl}
\usepackage{slashed} % Feynman slash notation
\usepackage{tikz}
\usepackage[most]{tcolorbox}
\tcbset{myformula/.style={colback=black!10, %yellow!10!white,
    colframe=white, %red!50!black,
    top=0pt,bottom=3pt,left=0pt,right=0pt,
    boxsep=6pt,
    arc=0pt,
    outer arc=0pt,
  }}
\usepackage[compat=1.0.0]{tikz-feynman}
\usepackage{cancel} % cross out to zero in math
\usepackage{tabularx}
\newcolumntype{Y}{>{\centering\arraybackslash}X}
\usepackage{tabu}


% Timeline https://tex.stackexchange.com/questions/196794/how-can-you-create-a-vertical-timeline
\newcommand{\foo}{\makebox[0pt]{\textbullet}\hskip-0.5pt\vrule width 1pt\hspace{\labelsep}}
\newcommand{\specialcell}[2][c]{% force line break
  \begin{tabular}[#1]{@{}c@{}}#2\end{tabular}}

% following https://codeinthehole.com/guides/writing-a-thesis-in-latex/
\usepackage{fancyhdr}
\pagestyle{fancy}
\rhead{}
\lhead{\nouppercase{\textsc{\leftmark}}}
\renewcommand{\headrulewidth}{0pt}
\makeatletter
\renewcommand{\chaptermark}[1]{\markboth{\textsc{\@chapapp}\ \thechapter:\ #1}{}}
\makeatother

% custom chapter/section headers
\usepackage{sectsty}
\chapterfont{\large\sc\centering}
\chaptertitlefont{\centering}
\subsubsectionfont{\centering}

% page layout
\parindent 0pt
\parskip 1ex
\renewcommand{\baselinestretch}{1.33}


\numberwithin{equation}{section}       % Tinker with equation numbering
\renewcommand{\bibname}{References}    % Alter appearance of table of contents slightly
\renewcommand{\contentsname}{Contents}
\pagenumbering{roman}                  % Sets the pagenumbering to Roman nunerals to begin with

% nomenclature definitions
\newcommand{\define}[2]{\textit{#2} (\textsc{#1})\nomenclature{\textsc{#1}}{#2}}

\creflabelformat{equation}{#2\textup{#1}#3} % cref no parenthesis for equation number

% sum symbol superimposed on integral sign
\DeclareMathOperator*{\SumInt}{%
  \mathchoice%
  {\ooalign{$\displaystyle\sum$\cr\hidewidth$\displaystyle\int$\hidewidth\cr}}
  {\ooalign{\raisebox{.14\height}{\scalebox{.7}{$\textstyle\sum$}}\cr\hidewidth$\textstyle\int$\hidewidth\cr}}
  {\ooalign{\raisebox{.2\height}{\scalebox{.6}{$\scriptstyle\sum$}}\cr$\scriptstyle\int$\cr}}
  {\ooalign{\raisebox{.2\height}{\scalebox{.6}{$\scriptstyle\sum$}}\cr$\scriptstyle\int$\cr}}
}
\DeclareMathOperator{\sfm2}{sfm2}




% References
\usepackage[
  style=numeric
]{biblatex}
\addbibresource{references.bib}

% Nomenclature
\renewcommand{\nomname}{List of Abbreviations\label{ch:nomenclature}}
% \let\oldDif\dif
% \renewcommand{\dif}[1]{\oldDif{#1}\,}


\DeclareMathOperator{\Rea}{Re}
\DeclareMathOperator{\Ima}{Im}
\DeclareMathOperator{\Tr}{Tr}
\newcommand*\anti[1]{\overline{#1}\,}
\newcommand*\intsum{\int\!\!\!\!\!\!\sum}
\newcommand*\fdif[1]{\frac{\dif^4 #1}{(2\pi)^4}}

% Colour scheme
\definecolor{primary}{rgb}{0.78, 0.89, 1}


% relative graphic paths
\makeatletter
\let\org@subfile\subfile
\renewcommand*{\subfile}[1]{%
  \filename@parse{#1}% LaTeX's file name parser
  \expandafter
  \graphicspath\expandafter{\expandafter{\filename@area}}%
  \org@subfile{#1}%
}
\makeatother

\makenomenclature
\makeindex


\begin{document}
  \begin{titlepage}
    \begin{center}
      \includegraphics[height=2.4cm]{./images/logo_UAB.eps}
      \hfill
      \includegraphics[height=2.4cm]{./images/logo_IFAE.eps}
      \par

      \vspace{0.3cm}
      \Large
      A PhD Thesis in Physics
      \makebox[\linewidth]{\rule{\textwidth}{2pt}}
      \Huge
      The \textsc{qcd} Strong Coupling from Hadronic \(\tau\) Decays
      \makebox[\linewidth]{\rule{\textwidth}{2pt}}

      \vspace{1.2cm}
      \Large
      Dirk Hornung

      \vspace{0.9cm}
      \normalsize
      \begin{tkizpicture}
        % Using the layered layout
        \feynmandiagram [scale=1.3,transform shape][layered layout, horizontal=a to b] {
          a [particle=\(\tau\)] -- [fermion] b -- [fermion] f1 [particle=\(\nu_{\tau}\)], b -- [boson, edge label'=\(W^{-}\)] c,
          c -- [anti fermion] f2 [particle=\(\anti{q}\)],
          c -- [fermion] f3 [particle=\(q\)],
        };
      \end{tkizpicture}

      \Large
      July 2019

      \vfill
      \makebox[\linewidth]{\rule{\textwidth}{1pt}}
      \large
      \textsc{Supervisors}: Matthias Jamin, Santiago Peris Rodríguez

    \end{center}
  \end{titlepage}

  
  \pdfbookmark[0]{Titlepage}{title}

  \newpage
  \pdfbookmark[0]{Abstract}{abstract} % Sets a PDF bookmark for the abstract
  \chapter*{Abstract}


  \pdfbookmark[0]{Acknowledgements}{acknowledgements} % Sets a PDF bookmark for the acknowledements
  \chapter*{Acknowledgements}

  \pagenumbering{arabic}
  \subfile{chapters/notation/index}
  \subfile{chapters/introduction/index}
  \subfile{chapters/qcdSumRules/index}
  \subfile{chapters/tauDecaysIntoHadrons/index}
  \subfile{chapters/measuringTheStrongCoupling/index}
  \subfile{chapters/conclusions/index}
  \subfile{chapters/appendix/index}

  \printbibliography
  \printnomenclature
\end{document}